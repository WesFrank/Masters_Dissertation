\documentclass[12pt, oneside]{book}
\setcounter{tocdepth}{3}
\setcounter{secnumdepth}{3}
\topmargin 5mm
\textheight 215mm
\evensidemargin 5mm
\oddsidemargin 10mm
\textwidth 145mm
\headsep 10mm

\newtheorem{theorem}{Theorem}[subsection]
\newtheorem{corollary}[theorem]{Corollary}
\newtheorem{lemma}[theorem]{Lemma}
\newtheorem{proposition}[theorem]{Proposition}
\newtheorem{example}[theorem]{Example}
\newtheorem{definition}[theorem]{Definition}
\newtheorem{remark}[theorem]{Remark}
\newtheorem{claim}[theorem]{Claim}
\newtheorem{construction}[theorem]{Construction}

\newtheorem{innercustomthm}{Discussion}
\newenvironment{custom}[1]
  {\renewcommand\theinnercustomthm{#1}\innercustomthm}
  {\endinnercustomthm}

\newcommand{\proof}{{\noindent \it Proof:~}}
\newcommand{\qed}{\hfill ~$\Box$\\}
\newcommand{\ul}{\cal U}
\newcommand{\powerset}{\raisebox{.15\baselineskip}{\Large\ensuremath{\wp}}}


\def\Rad{\operatorname{Rad}}
\def\QN{\operatorname{QN}}
\def\inf{\operatorname{inf}}
\def\ker{\operatorname{ker}}
\def\coker{\operatorname{coker}}
\def\inter{\operatorname{int}}
\def\cl{\operatorname{cl}}
\def\CL{\operatorname{CL}}
\def\Cl{\operatorname{Cl}}
\def\alg{\operatorname{alg}}
\def\der{\operatorname{der}}
\def\spn{\operatorname{span}}
\def\dist{\operatorname{dist}}
\def\max{\operatorname{max}}
\def\Exp{\operatorname{Exp}}
\def\bsr{\operatorname{bsr}}
\def\Comp{\operatorname{Comp}}
\def\ran{\operatorname{ran}}
\def\dim{\operatorname{dim}}
\def\codim{\operatorname{codim}}
\def\ind{\operatorname{ind}}
\def\Left{\operatorname{left}}
\def\Right{\operatorname{right}}
\def\APP{\operatorname{APP}}
\def\m{\operatorname{m}}

\def\changemargin#1#2{\list{}{\rightmargin#2\leftmargin#1}\item[]}
\let\endchangemargin=\endlist 

\usepackage{latexsym,amssymb,amsmath}
\usepackage{comment, mathtools}
\usepackage{pgfplots}
\usepackage{tikz}
\usepackage{enumerate, enumitem}
\usepackage{xcolor}
\usepackage{geometry}
\usepackage{titlesec}
\usepackage{tabularx}
\usepackage{nomencl}
\usepackage{mathrsfs}
\usepackage{mathtools}
\usepackage{extarrows}
\usepackage{graphics}
\usepackage{mdframed}


\DeclareMathAlphabet{\mathpzc}{OT1}{pzc}{m}{it}


\usepackage[british]{babel}
\usepackage[useregional]{datetime2}
\DTMlangsetup[en-GB]{showdayofmonth=false}


\usepackage{graphicx,lipsum}
\pagestyle{empty}


%\usepackage{lipsum}
\usepackage{lmodern}
\usepackage{tcolorbox}



%\theoremstyle{plain}


\usepackage{hyperref}
\hypersetup{
colorlinks = true,
%colorlinks = false,
linkcolor = blue,
%linkbordercolor = blue,
citecolor = red,
%citebordercolor = red,
urlcolor = cyan,
}

%\usepackage[czech]{babel}
%\usepackage[english]{babel}

%\loadspellchecklist[en][wordlist.txt]
%\setupspellchecking[state=start]

\usepackage{transparent}
\usepackage[pages=some,scale=1,angle=0,opacity=0.7]{background}
\usepackage{graphicx}
\usepackage{lipsum}



\newcommand\BackImage[2][scale=1]{%
\BgThispage
\backgroundsetup{
  contents={\includegraphics{background.png}}
  }
}


\begin{document}


\pagestyle{plain}
\pagenumbering{arabic}



\begin{titlepage}
\BackImage[width=.5\textwidth]{example-image-a}% image on page 1

\begin{center}
{\bf THE SPECTRAL TOPOLOGY IN RINGS}\\
\smallskip
\center{by}\\
\vspace{0.25 in}
{\bf FRANCOIS WESSELS}\\
\vspace{0.25 in}
{\bf DISSERTATION}\\
\vspace{0.25 in}
{\bf submitted in the fulfillment of the requirements for the degree}\\
\vspace{0.25 in}
{\bf MAGISTER SCIENTIAE}\\
\medskip
{\bf in}\\
\medskip
{\bf MATHEMATICS}\\
\vspace{0.25 in}
{\bf to the}\\
\smallskip
{\bf FACULTY OF SCIENCE}\\
\vspace{0.25 in}
\vspace{0.25 in}
{\bf at the}\\
\medskip
{\bf UNIVERSITY OF JOHANNESBURG}\\
\vspace{0.25 in}
\vspace{0.25 in}
{\bf SUPERVISOR: DR A SWARTZ}\\
\vspace{0.25 in}
{\bf CO-SUPERVISOR: PROF H RAUBENHEIMER}\\
\vspace{0.25 in}
\end{center}



\vspace{0.25 in}
\begin{center}
{\bf \today}\\
\end{center}

\end{titlepage}


\pagenumbering{roman}

\cleardoublepage
\setcounter{page}{1}


\tableofcontents





\chapter*{Declaration}
\addcontentsline{toc}{chapter}{\numberline{}Declaration}%


By submitting this dissertation, I declare that the entirety of the work contained herein is my own original work and that I am the owner of the copyright thereof (unless explicitly otherwise stated) and that I have not previously submitted it or part thereof for obtaining any other qualification(s).\\[1cm]

\noindent Date: \today \\

\begin{flushleft}
\includegraphics[scale=0.15]{Handtekening.png}\\
\line(1,0){150}\\
FJ Wessels\\[5cm]
\end{flushleft}


\noindent Copyright \textcopyright{} \today \space University of Johannesburg\\
\noindent All rights reserved

\chapter*{Dedication}
\addcontentsline{toc}{chapter}{\numberline{}Dedication}%

\section*{To my Teachers:}
\addcontentsline{toc}{section}{\numberline{}To my Teachers}%

\begin{flushleft}

\begin{table}[h]
\begin{tabularx}{\textwidth}{l X}
Mnr. Haasbroek: & Wie aan ons die stappe van langde{\"e}ling vir die eerste keer verduidelik het deur om van komieklike "karate-moves" gebruik te maak \newline [Laerskool Villieria - 2003]\\[0.2cm]

Mnr. Naud{\'e}: & Wie aan ons die wetenskaplike metode vir die eerste keer verduidelik het en my belangstelling vir ondersoek en navorsing geprikkel het\newline [Laerskool Tini Vorster - 2004 tot 2005] \\[0.2cm]

Mev. Horn: & My English teacher and guardian from Ho{\"e}rskool Hugenote who encouraged me to rise above my circumstances \newline [Ho{\"e}rskool Hugenote - 2006 to 2010]\\[0.2cm]

Mev. Vermaak: & Thank you for your encouragement in expressing my most creative thoughts \newline [Ho{\"e}rskool Hugenote - 2009 to 2010]\\[0.2cm]

Mev. Loock: & Wie bo haar eie omstandighede gestyg het om ander te leer hoe om dieselfde te bereik \newline [Ho{\"e}rskool Hugenote - 2009 to 2010]\\[0.2cm]

Mev. Roodt: & Wie imagin{\^e}re getalle en Calculus aan ons bekend gestel het \newline [Ho{\"e}rskool Hugenote - 2008] \\[0.2cm]

Mev. Nortj{\'e}: & Wie Euklidiese meetkunde op 'n begrypbare grondslag vir die eerste keer aan ons oorgedra het \newline [Ho{\"e}rskool Hugenote - 2009] \\[0.2cm]

\end{tabularx}
\end{table}

\end{flushleft}

\pagebreak


\newpage
\section*{To my Lecturers:}
\addcontentsline{toc}{section}{\numberline{}To my Lecturers}%

\begin{flushleft}

\begin{table}[h]
\begin{tabularx}{\textwidth}{l X}

Prof. Pretorius: & Wie ho{\"e}r-orde konsepte van Calculus op 'n eenvoudige, begrypbare wyse aan 'n gehoor kon oordra \newline [Universiteit van Pretoria - 2012]\\[0.2cm]

Prof. Sauer: & Wie die streng wiskundige metodes van parsi{\"e}le differensiaalvergelykings die eerste keer aan ons verduidelik het \newline [Universiteit van Pretoria - 2014]\\[0.2cm]

Prof. Cinti: & Who played a great influence in my learning of the methods and techniques of partial differential equations via correspondence [University of Johannesburg - 2018]\\[0.2cm]

Dr. Braadvedt: & Who has been a great coordinator, facilitator, mentor and go-to person. I wish I took your course on measure theory. [University of Johannesburg - 2018 to 2019]\\[0.2cm]

Prof. Brits: & Wie die vak Funksionele Analise voor my o{\"e} op die skryfbord ontsyfer het - 'n ongelooflike ervaring en voorreg om u klas te kon bywoon [University of Johannesburg - 2018]\\[0.2cm]

Prof. Raubenheimer:	& To have witnessed his mathematical collaborative brilliance first-hand has been a great privilege for me throughout my learning experiences [University of Johannesburg] \\[0.2cm]


Dr. Swartz:  		& I am the tool, he the mechanic\\[0.2cm]
{}					& Together to invent the engine\\[0.2cm]			
{}					& I the builder, he the engineer\\[0.2cm]		
{}					& Together to construct a machine\\[0.2cm]						
{}					& I the magic, he the wizard\\[0.2cm]
{}					& To produce the spell of truth\\[0.2cm]

\end{tabularx}
\end{table}

\end{flushleft}



\newpage
\section*{To those who understand little of my interest in mathematics but much of my journey to success:}
\addcontentsline{toc}{section}{\numberline{}Family \& Friends}%

\begin{flushleft}

\begin{table}[h]
\begin{tabularx}{\textwidth}{l X}

Vanessa: & (Mother) The source of nurture in my life. I am proud to be your son\\[0.2cm]

Louis: & (Father) Who, despite having difficulties of your own, taught me the importance of balance in life\\[0.2cm]

Ingrid: & (Sister) Hoewel ons hemelsbre{\"e}d verskil, kon ons net sowel 'n tweeling gewees het met al die liefde wat ons deel\\[0.2cm]

Marincy: & (Ouma) My liefde vir u strek verder as die landlyn Kaap toe\\[0.2cm]

Tiny \& Shmally: & My kitty meauw meauws\\[0.2cm]

\end{tabularx}
\end{table}

\end{flushleft}


\begin{center}
\textbf{\begin{large}
Ek het julle oneindig-maal lief
\end{large}}
\end{center}


\begin{flushleft}

\begin{table}[h]
\begin{tabularx}{\textwidth}{l X}

Darryn Jacobs: & As ek aan jou dink, dink ek aan 'n broer \\[0.2cm]

Sean Lubane: & To my dearest friend who, despite having been temporarily homeless during senior year, showed persistence and determination in finishing what he started \\[0.2cm]

Jerome \& Amber Bastick: & Ware vriendskap, die jare saam julle is vir my baie kosbaar \\[0.2cm]

Bryce Mapole: & Whose leadership qualities and inspirational character influenced my contribution to Africa \\[0.2cm]

Celine de Villiers: & The first person to believe in my capabilities \\[0.2cm]

Dawie, Leticia \& Louis: & Dankie vir julle bystand gedurende baie moeilike tye \\[0.2cm]

Gagie: & Your positivity shines bright \\[0.2cm]

M.F. Hancock Jr: & A true hound of a dog, a friend whose hand I never shook but who touched me deeply \\[0.2cm]

E. Schilling \& V. Moses: & Wie vir my die fundza-lushaka "oorbruggings" geleentheid onderhandel het \\[0.2cm]

To My Future Wife: & Please answer your DM's \\[0.2cm]

\end{tabularx}
\end{table}

\end{flushleft}


\begin{center}
\textbf{\begin{Large}
I dedicate this dissertation to you
\end{Large}}
\end{center}



\chapter*{Acknowledgements}
\addcontentsline{toc}{chapter}{\numberline{}Acknowledgements}%


We would like to mention a kind word of appreciation and gratitude towards the help of the University of Johannesburg's project managers and Department of Mathematics' staff members.  To those who have lead us through our learning experiences and guided us to enjoy the subject of interest, the journey to higher mathematics is rather humbling and priceless. 
\par \medskip
\noindent To my supervisor Dr. A Swartz and co-supervisor Prof. H Raubenheimer, a dearest word of thank you and appreciation. From the authors of the books and articles we've studied, to the students with whom we co-studied, a thank you for your motivation. 
\par \medskip
\noindent A sincere blessing to the departments' administrative staff for their help and friendliness over correspondence and three years of fulfilment in a beautiful setting of learning experiences. I emphasise my gratitude towards my project supervisor Dr. A Swartz for arranging the allocation of funds towards a portion of my student fees. A thousand thank you's to the entire mathematics department at the University of Johannesburg and all of their administrative staff for making this journey exciting.

\newpage

\pagenumbering{arabic}

\cleardoublepage
\setcounter{page}{1}



\chapter*{Introduction}
\addcontentsline{toc}{chapter}{\numberline{}Introduction}%

A careful study of Hilbert \cite{DH1} and Cauchy's \cite{ALC} original work reveals that the 
word \textit{characteristique} in French translates to "eigenschaften" in German ("eienskap" in Afrikaans) which may be interpreted as \textit{intrinsic} or \textit{property} in the sense that it associates 
a(n) (set of) eigenvalue(s) $\lambda$ [thought of as the seed] belonging to an operator $T$ [thought of as the stem]. 
It is due to this operator-eigenvalue tug'o war that most of the modern theory has become so well developed to model physical problems leading to insightful results. 
The word "integralgleichung" is used frequently throughout the text which translates to English as \textit{integral equation}. 
In his article Hilbert discusses the symmetric kernel on page 52, characteristic determinant equations on page 53, linear combinations and algebraic solutions to the integral equations on page 57, convergent power series solution of 
Fredholm determinant expression on page 58, orthogonality and eigenfunction solutions of integral equations on page 67, essentially the foundation for classical operator theory.


\bigskip \noindent Long before the power of vector space algebra was postulated into existence and became common practice, the eigenvalue method was one of the tools used to study systems of linear equations as roots of determinant equations, but made some of its first appearance in the studies of differential and integral equations originating from variational calculus during the celestial era of mathematics, giving birth to operator theory. Operator theory can be chronologically tracked through Leonard Euler (1707 - 1783), Jean le Rond d'Alembert (1717 - 1783), Joseph Louis Lagrange (1736 - 1813), Daniel (1700 - 1782) \& the three Bernoulli brothers (1710 - 1790, 1744 - 1807, 1759 - 1789), Pierre Simon Laplace (1749 - 1827), Joseph Fourier (1768 - 1830), Johann Peter Gustav Lejeune Dirichlet (1805 - 1859), who all played a big part in pursuit of an even bigger picture - \textit{Modelling, analysing, solving and generalising functional equations}.

\bigskip \noindent Sometimes eigenvalues are referred to as the roots of characteristic equations, those equations deeply associated with its homogeneous auxiliary equations and in modern mathematics related to the spectrum, the set of values in a scalar field associated with algebraic invertibility of an operator expression of the form $T - \lambda I$. In fact, eigenvalues arise naturally in the generalization of operator equations as the following examples illustrate:


\begin{example}
\normalfont
\noindent Consider the homogeneous linear system of differential equations: 
\begin{center}
$\frac{d}{dt} \overline{x}(t)=A \overline{x}(t)$
\end{center}
\noindent where $\overline{x}(t)$ is a column of functionals dependent on $t$ and $A$ is the coefficient matrix for the system. It is an elementary exercise to verify upon substitution that 
$\overline{x}(t) = v(t) e^{\lambda t}$ reduces the equation to 
$\lambda \overline{v} = A \overline{v}$ not surprisingly, an ordinary eigenvalue problem which may be written more recognizably as $(A - \lambda I) \overline{v}= \overline{0}$ with the remaining task to compute the eigenvalues $\lambda$ and eigenvectors $\overline{v}$.
\qed
\end{example}

\begin{example}
\normalfont
\noindent Consider the classical homogeneous Fredholm integral equation: 
\begin{center}
$g(x)=\mu \displaystyle \int_a^b K(x,t) g(t) \ dt$
\end{center}
\noindent Let $\mu =\frac{1}{\lambda}$ and define the integral transform by 
$(Tg)(x):=\displaystyle \int^b_a K(x,t) g(t) \ dt$ 
so that the above may be written more compactly as $g(x)=\frac{1}{\lambda} T g(x)$. 
Multiplying $\lambda$ throughout gives $\lambda g(x) = T g(x)$ which can be rearranged as
$T g(x) - \lambda g(x)=0$ and finally factoring $g(x)$, the equation is recognizable as an ordinary eigenvalue problem $(T-\lambda I) g(x) = 0$.
\qed
\end{example}

\begin{example}
\normalfont
\noindent Consider the operator equation $(Tu)(x,t)=(Su)(x,t)$ for an unknown two variable function $u(x,t)$ dependent on position $x$ along the closed interval $[0,l]$ and time $t$,  described on the domain $\mathscr{D}(u)=\{ (x,t): x \in [0,l], t \geq 0 \}$.
\vskip 0.3cm
\noindent If we assign $T = \frac{\partial}{\partial t}$ and $S = \alpha \frac{\partial }{\partial x}$ then we recover the familiar transport equation.
\vskip 0.3cm
\noindent If we assign $T = \frac{\partial}{\partial t}$ and $S = \alpha^2 \frac{\partial^2 }{\partial x^2}$ then we recover the familiar heat equation.
\vskip 0.3cm
\noindent If we assign $T = \frac{\partial^2}{\partial t^2}$ and $S = \alpha^2 \nabla^2$ then we recover the familiar wave equation.
\vskip 0.3cm
\noindent We wish to find a product solution of the form $u(x,t)=f(x)g(t)$ so that the equation becomes separable $\frac{Tg}{g} = \frac{Sf}{f}$ in which case this reduces to the recognizable ordinary eigenvalue problems:
\begin{center}
$Tg= \lambda g$ \quad and \quad $Sf = \lambda f$
\end{center}
\noindent but as it turns out, separability of operator equations is related to the spectral theorem. The applicability of the technique is intrinsic to the coordinate system in which the equation is modelled and depends on the symmetry properties of the equation.
\qed
\end{example}
%\vskip 0.3cm
\noindent In each example above, the equations were reducible to an ordinary linear eigenvalue problem. The foundations of the solvability theory for such problems had already mostly been explored by researchers such as the three prodigies, Johann Carl Friedrich Gauss (1777 - 1855) 
(in his \textit{Disquisitiones Arithmeticae}), \'Evariste Galois (1811 - 1832), and Niels Hendrik Abel (1802 - 1829). 
This is the insight which motivated mathematicians to study operator equations collectively as a subject.
However, a great deal of the behaviour of operator equations is due to the underlying structure space on which the equation is defined. For example continuity, convergence, invertibility, integrability and
 differentiability are all in one way or another, dependent, not only on the form of the equation, but also largely on the underlying algebraic and topological aspects of the space.
\vskip 0.3cm
\noindent The demand for clear insight about the underlying structure space for operator equations further stimulated research in the fields of topology and algebra. The story diverts to set theory and the axiomatization of mathematics. Georg Ferdinand Ludwig Philipp Cantor's (1845 - 1918) set theory had been digested by the mathematics community and axiomatized by Ernst Zermelo (1871 - 1953) in his 1908 paper \textit{Untersuchengen \"uber die Grundlagen der Mengenlehre} ["mengenlehre" in German translates to \textit{set theory} in English] while Giuseppe Peano fully axiomatized linear spaces in his 1888 book, titled \textit{Calcolo}. Cantor had already loosely defined the notions of open, closed and derived sets, influencing Ren\'e-Louis Baire (1874 - 1932), \'Emile Borel (1871 - 1956) and Henri Lebesgue (1875 - 1941) to write his phenomenal piece, titled \textit{Sur l'approximation des fonctions}, thereby extending the ideas of George Friedrich Bernhard Riemann (1826 - 1866) with strong reliance on set theory, consequently giving rise to the subjects of measure and integration theory as it is taught in current times.
\vskip 0.3cm
\noindent Set theory shed new light on the concept of a function which inspired Maurice Ren\'e Fr\'echet (1878 - 1973) to fabricate the axioms of a metric space, in doing so, extending Leibniz and Newton's classical calculus of functions to topology, roughly phrased from the mouth of contemporary mathematical dialect. Felix Hausdorff (1868 - 1942) extended much of topology and set theory by introducing the concepts of neighbourhoods, neighbourhood systems and partially ordered sets, associated with Zorn's lemma, in his 1914 piece \textit{Grundzüge der Mengenlehre}. By now, the \textit{duality principle} seeped its way through to function spaces found in the works of Hans Hahn (1879 - 1934), Frigyes Riesz (1880 - 1956), Stefan Banach (1892 - 1945), Juluisz Schauder (1899 - 1943), Giulio Ascoli (1843 - 1896) and Cesare Arzel\'a (1847 - 1912) who all played roles on elaborating the concepts of Bernard Bolzano (1781 - 1848), Karl Theodor Wilhelm Weierstrass (1815 - 1897) and other contemporaries to sequences of functions (boundedness, continuity, convergence, compactness), representations of linear functionals, functional domain extension, normed spaces, completeness of spaces, etc...
\vskip 0.3cm
\noindent The power of the \textit{duality principle} was once again emphasised when Kazimierz Kuratowski (1896 - 1980) defined his closure algebra, a dual approach for constructing a topology on any collection, in his 1922 paper \cite{KK}, a discovery that would become extremely important to the marriage of algebra and topology. In the words of Herman Weyl (1885 - 1955) - "...the angel of topology and the devil of abstract algebra fight for the soul... " A substantial amount of modern mathematical content for algebraic topology is due to the team of writers, under the name Nicolas Bourbaki (1934 - 1935) in their series of textbooks. 
\vskip 0.3cm
\noindent For the past couple of decades, attempts have been made (Oscar Zariski (1899 - 1986)) to find an analytic expression which generates a topology on a ring. The first question that would come to mind for the majority of people is "why do we want such a structure?"
\vskip 0.3cm
\noindent To answer this question we heavily rely on the motto \textit{modelling, analysing, solving and generalising functional equations} to argue our point of view that it is convenient to know the topological and algebraic features of an operator all at once, since this will allow us to decide whether it may have certain desirable properties such as being (left/right (nearly)) invertible, open, continuous, quasinilpotent, etc...
\vskip 0.3cm
\noindent This dissertation celebrates the work of Robin Harte and Dragana Cvetkovi\'c-ili\'c who recently succeeded in the task of defining a topology on a ring via a Kuratowski closure operation, leading to new insights for operators - the main theme of this dissertation. 
Below is a brief overview of the content of various chapters.
\vskip 0.3cm
\noindent In Chapter 1 we define basic concepts that we need to understand the content that follows.
\vskip 0.3cm
\noindent In Chapter 2 we prove that the construction of the authors is a Kuratowski closure operation. We also discuss some properties of the closure operation. 
\vskip 0.3cm
\noindent In Chapter 3 we compare the spectral topology with the norm topology on a Banach algebra. We also look at additional properties, including the structure of neighbourhoods of the spectral topology.
\vskip 0.3cm
\noindent In Chapter 4 we look at how the spectral closure enables us to define a concept 
of quasinilpotent that applies to a general ring.
\vskip 0.3cm
\noindent In Chapter 5 we look at how the spectral closure intervenes in concepts of generalized
 invertibility.
\vskip 0.3cm
\noindent In Chapter 6 we look at how the spectral closure intervenes with Fredholm Theory relative to 
a Banach algebra homomorphism.
\vskip 0.3cm
\noindent In Chapter 7 we look at how the spectral closure intervenes in the concepts of 
Bass Stable Rank of a ring. 
\vskip 0.3cm
\noindent In Chapter 8 we give a brief summary of what was achieved, as well as highlight some questions raised by the study.
\vskip 2cm
\begin{center}
\maltese
\end{center}

\chapter{Rings, topologies, Banach algebras} \label{chapter 1}

\section{Introduction}

This chapter provides an overview of all the prerequisites needed for our discussions in coming chapters. 
The main mathematical structures we will encounter are rings, topologies, topologocal rings and 
Banach algebras. In this chapter these structures are defined and discussed in the detail necessary to make the theory that follows understandable. Groups and vector spaces are considered to be understood and are not defined. 

\section{Rings}

\begin{definition}[\cite{G1}, p. 95] \label{RingDefinition}
\normalfont
\noindent A {\sl ring} is a nonempty set $R$ on which there are defined two binary operations, + and $\cdot$, called addition and multiplication respectively, which satisfy the following axioms:
\begin{itemize}
\item [R1.] \quad with respect to +, $R$ is an abelian group.
\item [R2.] \quad  $\cdot$ is associative: 
$a \cdot (b \cdot c) = (a \cdot b) \cdot c$ for all $a, b, c \in R$. 	
\item [R3.] \quad The following distributive laws are satisfied (for all $a, b, c \in R$)
\vskip 0.3cm
$a \cdot (b + c) = a \cdot b + a \cdot c $ \qquad  and \qquad $(b + c)\cdot a = b \cdot a + c \cdot a.$
\end{itemize}
\end{definition} 
\noindent If $R$ is a ring with operations of + and $\cdot$ we will also use the triple 
$\langle R, +, \cdot \rangle$ to represent the ring structure.
\vskip 0.3cm
\noindent If $R$ is a ring and $a \cdot b = b \cdot a$ for all $a, b \in R$, then $R$ is called 
{\sl commutative}.
\vskip 0.3cm
\noindent If $R$ is a ring then by R1 it has a unique {\sl additive identity}, 
which we denote by $0$, which satisfies the condition that $0 \cdot a = a \cdot 0 = 0$ for all 
$a \in R$.
\vskip 0.3cm
\noindent Let $R$ be a ring and suppose there exists $1 \in R$ 
with the property that for all $a \in R$
\begin{center}
$1 \cdot a = a = a \cdot 1$.
\end{center}
\noindent We will call such an element an {\sl identity element}, {\sl multiplicative identity} or 
simply an {\sl identity} of the ring. 
A ring with an identity is referred to as a {\sl ring with unity} or a {\sl unital ring}. 
It is not necessary for a ring to have an identity, but if it does then the identity is unique. 
This is easily seen to be the case as follows. Suppose $1$ and $1'$ are identities of the ring. 
Then it is clear (from the properties of 1 and $1'$) that we must have 
\begin{center}
$1 = 1 \cdot 1' = 1'$.
\end{center}
\vskip 0.3cm

\noindent If a ring $R$ contains an identity with respect to multiplication, we will write $1_R$ 
if we need to emphasize the ring $R$. In this dissertation every ring will have an identity. 
From this point the expression `Let $R$ be a ring' will mean `Let $R$ be a ring with identity'. 

\vskip 0.3cm
\noindent If $\langle R, +, \cdot \rangle$ is a ring then we will usually suppress the 
multiplication sign, so that for $a, b \in R$, $ab$ will have the same meaning as $a \cdot b$. 
The symbol for multiplication will be used only if we feel readability will be enhanced by its use.
\vskip 0.3cm
\noindent Let $R$ be a ring. An element $a \in R$, is called {\sl left} 
(respectively {\sl right}) {\sl invertible} if there exists $b \in R$ such that $ba = 1$ 
(respectively $ab = 1$). The element $b$ is called a {\sl left} (respectively {\sl right}) {\sl inverse}
 of $a$.

\vskip 0.3cm

\noindent If an element $a \in R$  is both left and right invertible, then the left and right inverses for $a$ coincide, as we shall now illustrate. If $a$ is left invertible and right invertible then there exist $b, c \in R$ such that $ba = 1$ and $ac = 1$. So
\begin{equation} \label{InversesAreUnique}
b = b \cdot 1 = b \cdot (a \cdot c) = (b \cdot a) \cdot c = 1 \cdot c = c.
\end{equation}

\noindent In this last case  $a$ is called {\sl invertible} and its left (and right) inverse is called an {\sl inverse}. This argument also shows that the inverse of an element, if it exists, must be unique. This is so because an inverse is both a left inverse and a right inverse. 
The inverse of $a$ is denoted by $a^{-1}$. The sets of left invertible, right  invertible and invertible elements in $R$ are denoted by $R^{-1}_l$, $R^{-1}_r$ and $R^{-1}$ respectively.

\begin{example} [\cite{G1}, p. 97]
\normalfont
\noindent Let $R = \{0\}$, and let us define addition and multiplication by:
\begin{center}
$0 + 0 = 0$ \quad and \quad $0 \cdot 0 = 0.$
\end{center}
\noindent Then the axioms R1 - R3 are satisfied in a trivial way. $R$ is a ring called the 
{\sl trivial ring}. Note that $R$ is commutative with identity 0. Note also that in $R$, we have 
$1_R = 0_R$. \qed
\end{example}
In what follows we will always assume that an arbirary ring is not the trivial ring. 
\vskip 0.3cm
\noindent An element $a$ of a ring $R$ is said to be {\sl nilpotent} if there exists some 
$n \in \mathbb{N}$ such that $a^{n}=0$.
\begin{remark}
\normalfont
\noindent Suppose $G, H \subseteq R$ and $K \subseteq R^{-1}$ for a ring $R$. 
We will use the following notations:
\begin{enumerate}[label=(\alph*)] 
\item $K^{-1} = \{k^{-1} : k \in K \}$.
\item $G + H = \{g + h : g \in G, h \in H \}$.
\item $G \cdot H = \{ g \cdot h : g \in G, h \in H \}$.
\item $a + G = \{a + g : g \in G \}$.
\end{enumerate}
\end{remark}
\qed
\noindent The following lemma lists a number of properties of invertible elements which we 
will need to refer to.
\begin{lemma} \label{RInverseStructure1}
\normalfont
Let $R$ be a ring. Then

\begin{enumerate}[label=(\alph*)]
\item If $a,b \in R^{-1}$ then $ab \in R^{-1}$.
\item If $ab \in R^{-1}$ then $a \in R_r^{-1}$ and $b \in R_l^{-1}$.
\item If $ab \in R^{-1}$ and $b \in R^{-1}$ then $a \in R^{-1}$.
\item If $a \in R^{-1}$ and $b \not \in R^{-1}$ then $ab \not \in R^{-1}$.
\end{enumerate}
\end{lemma}
\qed

\vspace{-0.5cm}
\begin{remark}
\normalfont
\noindent We can say more about the structure of the set of invertibles in a ring. 
If $\langle R, +, \cdot \rangle$ is a ring, then $\langle R^{-1}, \cdot \rangle$ is a group. 
We will refer to this group as the {\sl group of units} or {\sl the invertible group} of the ring $R$. \qed


\end{remark}
\noindent Next, we define types of rings encountered throughout the dissertation:

\begin{definition}[\cite{F}, p. 252]
\normalfont
Let $R$ be a ring and suppose that $a, b \in R$. Suppose also that $a \neq 0$ and $b \neq 0$, but that 
$a \cdot b = 0$. Then we call $a$ and $b$ {\sl divisors of zero}. In particular, $a$ is a {\sl left} divisor of zero and $b$ is a {\sl right} divisor of zero.

\end{definition}

\begin{definition}[\cite{F}, p. 254]
\normalfont
\noindent An {\sl integral domain} is a commutative ring with unity containing no divisors 
of zero.
\end{definition}

\begin{example}[\cite{G}, p. 249] \label{ZIsAnIntegralDom}
\normalfont
\noindent The ring of integers is an integral domain. \qed
\end{example}

\begin{proposition} [\cite{G1}, p. 101]
\normalfont
\noindent Let $R$ be an integral domain, $a, b, c \in R, a\neq 0$. Suppose that $a\cdot b = a\cdot c$. Then $b = c.$ \qed
\end{proposition}

\begin{definition} \label{dfnDivisionRing}
\normalfont
A {\sl division ring} is a ring with the property that every nonzero element is invertible.

\end{definition}


\begin{definition}
\textup{A {\sl Boolean ring} is a ring $R$ with the property that $r^2=r$ for every $r \in R$.} 
\end{definition}
\begin{remark} \label{BooleanInvertibles}
\normalfont
If $R$ is a Boolean ring, then $R^{-1}=\{1\}$. To see this suppose $r \in R^{-1}$. Then 
\begin{center}
$r=r \cdot 1=r \cdot (r \cdot r^{-1})=r^2 \cdot r^{-1}=r \cdot r^{-1}=1.$
\end{center}
\qed
\end{remark}

\noindent It is well known that $\mathbb{Z}$, together with the usual addition and multiplication of integers is a ring. In what follows we will assume some of the basic properties of this ring. 
This includes the fact that it is a ring with unity and some of the order theoretic properties of the ring. Developing these from scratch would take us too far from the main focus of this dissertation. The details of these properties of the ring of integers is developed in Chapter 2 of \cite{G1}.

\begin{example} \label{InvertiblesInZ}
\normalfont
$\mathbb{Z}$ with usual addition $+$ and multiplication $\cdot$ is a ring. We show that $\mathbb{Z}^{-1} = \{-1, 1\}$. Consider arbitrary $a, b \in \mathbb{Z}$ such that $a b=1$. We show that either $a = b = 1$ or $a = b = -1$.
\vskip 0.3cm
\noindent Clearly $a\not = 0$ and $b \not =0$. We prove the statement above using a proof by cases argument, based on the possible values for $a$. These values are:
$$a > 1, \qquad a < -1, \qquad a = 1, \qquad a = -1.$$ 
Suppose that $a > 1$. If $b > 1$ then $a b > 1$, contradicting $a b = 1$. If $b < -1$ then $a b < -1$, contradicting $a b = 1$. If $b = 1$, then $a b > 1$, again contradicting $a b = 1$. Finally, if $b = -1$, then $a b < -1$, again leading to a contradiction.
\vskip 0.3cm
\noindent The argument for $a < -1$ follows a similar pattern, with signs and inequalities reversed.
\vskip 0.3cm
\noindent Suppose that $a=1$. Then we must have $b=1$, for suppose $b \not = 1$. Then $a  b=1 \cdot b = b \not = 1$, contradicting that $a b=1$.
\vskip 0.3cm
\noindent Finally, suppose that $a=-1$. Then necessarily $b=-1$ since otherwise $a  b=(-1) \cdot b = -b \not = 1$.\qed

\end{example}

\begin{example}[\cite{G1}, p. 100]
\normalfont
\noindent Let $R$ and $S$ be any two rings, and let $R \times S$ denote the Cartesian product of $R$ 
and $S$ as sets. Let $(r, s), (r', s') \in R \times S$. Then we define addition and multiplication 
in $R \times S$ by
\begin{align*}
(r, s) + (r', s') &= (r + r', s + s'),\\
(r, s)(r', s') &= (r r', s s')
\end{align*}
With + and $\cdot$ defined as above we have that $\langle R \times S, +, \cdot \rangle$ is a ring. 
This structure is sometimes referred to as the {\sl direct sum} of $R$ and $S$ and denoted by 
$R \oplus S$. Checking that $R \oplus S$ is a ring is a routine exercise. 
So, instead of checking every condition (R1 - R3) we look at only the details relating to the zero element 
and invertibility in the product ring, $R \oplus S$, since those details will be pertinent for the
discussions to come. 
\vskip 0.3cm
\noindent Let $0_R$ and $0_S$ be the zero elements in $R$ and $S$ respectively, and let $1_R$ and $1_S$ 
be the multiplicative identities in $R$ and $S$ respectively. 
\vskip 0.3cm
\noindent First we show that $(0_R, 0_S)$ is the zero element in the product ring. To see this, 
let $(r, s) \in R \oplus S$. Then
\begin{center}
$(r, s) \cdot (0_R, 0_S) = (r \cdot 0_R, s \cdot 0_S) = (0_R, 0_S)$
\end{center}
and 
\begin{center}
$(0_R, 0_S)  \cdot  (r, s) = ( 0_R \cdot r , 0_S  \cdot s ) = (0_R, 0_S)$.
\end{center}
Also,
\begin{center}
$(0_R, 0_S) + (r, s)  = (0_R + r, 0_S + s) = (r, s)$
\end{center}
and
\begin{center}
$(r, s) + (0_R, 0_S)   = (r + 0_R, s + 0_S) = (r, s).$
\end{center}
Hence $(0_R, 0_S)$ is the zero element in $R \oplus S$. 
Next,
\begin{center}
$(r, s) \cdot (1_R, 1_S) = (r \cdot 1_R, s \cdot 1_S) = (r, s)$
\end{center}
and
\begin{center}
$(1_R, 1_S) \cdot (r, s)  = (1_R \cdot r, 1_S \cdot s) = (r, s).$
\end{center}
Hence $(1_R, 1_S)$ is the multiplicative identity in $R \oplus S$. 
\vskip 0.3cm
\noindent Next, let $R^{-1}$ and $S^{-1}$ be the sets of invertibles in $R$ and $S$ respectively. Then
\begin{center}
$(R \oplus S)^{-1} = R^{-1} \times S^{-1}$.
\end{center}
To see this, let $(r, s) \in (R \oplus S)^{-1}$. Then there must be $(r', s') \in R \oplus S$ such that 
\begin{center}
$(r', s') \cdot (r, s) = (1_R, 1_S)$
\end{center}
and 
\begin{center}
$(r, s) \cdot (r', s') = (1_R, 1_S)$.
\end{center}
\noindent These equations imply that $r' = r^{-1}$ and $s' = s^{-1}$. 
Hence $(r, s) \in R^{-1} \times S^{-1}$, which means $(R \oplus S)^{-1} \subseteq R^{-1} \times S^{-1}$.
\vskip 0.3cm
\noindent Conversely, let $(r, s) \in R^{-1} \times S^{-1}$. Then $r \in R^{-1}$ and $s \in S^{-1}$. 
Hence the inverses $r^{-1}$ and $s^{-1}$ exist. The equations
\begin{center}
$(r, s) (r^{-1}, s^{-1}) = (1_R, 1_S)$
\end{center} 
and
\begin{center}
$(r^{-1}, s^{-1}) (r, s)  = (1_R, 1_S)$
\end{center}
prove that $(r, s) \in (R \oplus S)^{-1}$. Hence $R^{-1} \times S^{-1} \subseteq (R \oplus S)^{-1}$. 
We have proved that 
\begin{center}
$(R \oplus S)^{-1} = R^{-1} \times S^{-1}$.
\end{center}
\qed
\end{example}
\vspace{-1cm}
\begin{lemma}[\cite{CH1}, p. 86] \label{Jacobson}
\normalfont
Let $R$ be a ring and $a,b \in R$. Then 
\begin{equation} \label{JacobsonEqn}
1-ab \in R^{-1} \iff 1-ba \in R^{-1}
\end{equation}
\end{lemma}
\qed

\begin{remark}\label{JacobsonRemV2}
\normalfont
\noindent Let $R$ be a ring and $a, b \in R$. We note that condition \eqref{JacobsonEqn} in Lemma \ref{Jacobson} is equivalent to the condition
\begin{equation} \label{JacobsonEqnV2}
1-ab \notin R^{-1} \iff 1-ba \notin R^{-1}
\end{equation}
\end{remark}
\qed
\begin{comment}
\proof Suppose that condition (1.1) holds, and suppose that $1 - ab \notin R^{-1}$. 
Then $1 - ba \notin R^{-1}$, since $1 - ba \in R^{-1}$ would imply (by (1.1)), that 
$1 - ab \in R^{-1}$, contradicting our initial assumption. Since $a$ and $b$ were arbitrary elements of 
$R$, this proves that 
condition (1.1) implies condition (1.2). To see the reverse implication uses a similar argument and we 
do not give those details as it would be repetitive.
\end{comment}
\subsection{Ideals}
\begin{definition}[\cite{G1},  p. 114]
\normalfont
\noindent Let $R$ be a ring. A subset $S$ of $R$ is called a {\sl subring} of $R$ if $S$ is a ring 
with respect to the operations of addition and multiplication inherited from $R$. 
\end{definition}

\begin{theorem} [\cite{G1}, p. 115]\label{SubringTest}
\normalfont
\noindent Let $S$ be a nonempty subset of a ring $R$. Then $S$ is a subring of $R$ if and only if 
for $a, b \in S$, we have $a - b \in S$ and $ab \in S$. \qed
\end{theorem}

\begin{definition}[\cite{G1},  p. 117] 
\normalfont
\noindent Let $R$ be a ring and let $I$ be a subring of $R$. We say that $I$ is a 
{\sl left (right) ideal} of $R$ if for all $r \in R$ and $a \in I$ we have 
that $ra \in I \text{\space} (ar \in I)$. 
An ideal $I$ which is both a left and right ideal of $R$, is simply called an {\sl ideal} of $R$.

\end{definition}

\noindent For any ring $R$, $R$ itself is an ideal of $R$, called the {\sl improper ideal}. 
All other ideals of $R$ are {\sl proper ideals}. For any ring $R$, the singleton set $\{ 0 \}$ is 
an ideal, called the {\sl trivial ideal}. All other ideals of $R$ are {\sl non-trivial ideals}.

\begin{lemma} \label{ProperIdeal}
\normalfont
\noindent Let $R$ be a commutative ring with identity and $I$ an ideal of $R$. 
If $I \cap R^{-1} \neq \emptyset$ then $I = R$.
\end{lemma}
\proof \space Let $R$ and $I$ be as described and suppose that $a \in I \cap R^{-1}$. 
We have that $I \subseteq R$. We show that $R \subseteq I$. Since $a \in R^{-1}$, $a$ has an inverse, 
$a^{-1}$ in $R$. Since $I$ is an ideal we have $1 = a^{-1}a \in I$. Let $r \in R$ be arbitrary. Then 
$r = r \cdot 1 \in I$, hence $R \subseteq I$. Hence $I = R$. \qed
\begin{definition}[\cite{F}, p. 322] 
\normalfont
A {\sl maximal left (right) ideal} of a ring $R$ is a left (right) ideal $I$ of $R$ different 
from $R$ such that there is no proper left (right) ideal of $R$ properly containing $I$. 
An ideal which is both a maximal left ideal and a maximal right ideal is simply called a 
maximal ideal.

\end{definition}

\begin{proposition}[\cite{G1}, p. 118] \label{QuotientRings}
\normalfont
\noindent Let $R$ be a ring and $I$ an ideal of $R$. Denote by $R / I$ the set of {\sl cosets} 
of the form $a+I$ where $a \in R$. Define addition and multiplication of cosets in $R / I$  by: 
\begin{center}
$(a+I)+(b+I)=(a+b)+I$ \qquad and \qquad $(a+I) \cdot (b+I)=a \cdot b + I$.
\end{center}

\noindent With these definitions $R / I$ is a ring called the {\sl quotient ring} of 
$R$ modulo $I$. 
\qed
\begin{remark}
\normalfont
The ring $R / I$ partitions the ring $R$ into a set of pairwise disjoint equivalence classes. 
The set $a + I$ is the equivalence class containing the element $a$. We will also use 
the equivalent notation $[a]$ for the equivalence class of $a$. Hence $[a] = a + I$. 
If $[a], [b]$ are elements in $R / I$, then $[a] = [b] \iff a - b \in I$.
\qed
\end{remark}

\end{proposition}

\begin{theorem}[\cite{R1}, p. 347] \label{Krull} \textup{(Krull's Theorem)} \textup{Let $R$ be a ring with unity and let $I$ be a proper (left/right) ideal of $R$ (respectively). Then there is a maximal (left/right) ideal of $R$ containing $I$ (respectively).} \qed

\end{theorem}


\subsection{The Jacobson radical}

We define the notions relating to an important ideal, called the {\sl Jacobson radical}. 

\begin{definition}[\cite{L1}, p. 50] 
\normalfont
Let $R$ be a ring and $\mathcal{M}_l$ be the collection of all maximal left ideals of $R$. 
The {\sl Jacobson radical} of $R$ is the intersection of all maximal left ideals of $R$: 
\begin{center}
$ \Rad R\ = \underset{I \in \mathcal{M}_l}{\bigcap} I$
\end{center}
\end{definition}

\noindent Strictly speaking, $\Rad R$ is the left Jacobson radical. 
Similarly, the right Jacobson radical of $R$ is defined by intersecting the maximal right ideals $I \in \mathcal{M}_r$ of $R$. 
It turns out, fortuitously, that the left and right Jacobson radicals coincide, so the distinction is, after all, unnecessary.
\vskip 0.3cm
\noindent The following proposition is used to characterize the Jacobson radical of a ring $R$ in terms of its group of invertible elements, $R^{-1}$.
\begin{proposition}[\cite{L1}, Lemma 4.1 - p. 50] \label{RadCharacter}
\normalfont
Let $R$ be a ring. Then
\begin{center}
$\Rad R=\{a \in R: 1-Ra \subseteq R^{-1}\}$ 
\end{center}
\qed
\end{proposition}

\vskip -3cm
\begin{definition}[\cite{L1}, p. 52] \label{dfnSemisimpleRing}
\normalfont
\noindent Let $R$ be a ring. If $\Rad R=\{0\}$, then $R$ is said to be a {\sl semisimple} ring.

\end{definition}

\begin{example} \label{ZisSemisimple}
\normalfont
\noindent $\langle \mathbb{Z}, +,  \cdot \rangle$ is semisimple. To see why this is the case,
 recall that $x \in \Rad \mathbb{Z}$ if and only if $1-xr \in \mathbb{Z}^{-1}$ for all 
$r \in \mathbb{Z}$. From Example \ref{InvertiblesInZ} we have $\mathbb{Z}^{-1}=\{-1,1\}$ 
so that $x\in \Rad \mathbb{Z}$ if and only if 
for all $r \in R$ we have that $1-xr=-1$ or $1-xr=1$, which simplifies to $xr=0$ or $xr=2$. 
Upon substitution of $r=\pm 1$ we get $x=2$ and $x=-2$, an impossibility, or $x=0$ necessarily.
\qed
\end{example}


\begin{definition}[\cite{L1}, p. 279] \label{dfnLocalRing}
\normalfont
\noindent A ring with a unique maximal ideal is called {\sl local}.
\end{definition}
\normalfont
\noindent It is not true in general that a ring has a unique maximal ideal. 
For example the ring of integers $\mathbb{Z}$ has infinitely many maximal ideals. If $I = p\mathbb{Z}$  where $p$ is prime, then $I$ is a maximal ideal in $\mathbb{Z}$ (\cite{G}, page 269, Exercise 9).

\begin{proposition} \label{chrLocalRing}
\normalfont
Let $R$ be a local ring. Then $R=R^{-1} \cup \Rad R$.

\end{proposition}

\proof \space The fact that $R^{-1} \cup \Rad R \subseteq R$ is obvious. It remains to show that $R \subseteq R^{-1} \cup \Rad R$. Consider arbitrary $a \in R$. If $a \in R^{-1}$, we are done. So suppose that $a \not \in R^{-1}$. 
Then either $a \not \in R_l^{-1}$ or $a \not \in R_r^{-1}$. Without loss of generality, suppose that $a \not \in R_r^{-1}$. 
Then $a R$ is a proper right ideal of $R$. By Theorem \ref{Krull}, $a R$ is contained in a maximal right ideal. But $\Rad R$ is the unique maximal right ideal, hence $a=a \cdot 1 \in a R \subseteq \Rad R$ and the result follows. The case in which $a \not \in R_l^{-1}$ is similar. \qed

\begin{proposition} \label{UniqueMaxIdeal}
\normalfont
\noindent Let $R$ be a commutative ring and let $I = R \setminus R^{-1}$. 
If $I$ is an ideal then $I$ is the unique maximal ideal of $R$.
\end{proposition}
\proof \space
\noindent Suppose that $R$ is a commutative ring with identity and that 
$I = R \setminus R^{-1}$ is an ideal. Notice that since $1 \notin I$, 
we know that $I$ is a proper ideal. To see that $I$ is maximal, suppose that there exists 
an ideal $J$ such that $I \subseteq J \subseteq R$. Suppose also that 
$I \neq J$. Hence there exists $a \in J$ such that $a \notin I.$ But then $a \in R^{-1}$. 
Hence $J \cap R^{-1} \neq \emptyset$. By Lemma \ref{ProperIdeal} we have that $J = R$, and so 
$I$ must be maximal.
\vskip 0.3cm
\noindent To see that $I$ is unique, suppose that $J$ is another maximal ideal of $R$. 
Then $J$ must be a proper ideal, so that $J \cap R^{-1} = \emptyset$. 
That means that $J \subset R \setminus R^{-1} = I$. 
Since $J$ is a maximal ideal it cannot be properly contained in any other proper ideal, 
hence $J = I$, and $I$ is unique. \qed

\begin{example}[\cite{S}, p. 3] \label{exmFormalPowerSer}
\normalfont
Let $R$ be a commutative ring. Denote by $R[[x]]$ the set of all infinite expressions of the form 
\begin{equation} \label{FormPowSerAsSum}
a_0+a_1x+a_2x^2+ \cdots + a_i x^i + \cdots = \displaystyle \sum_{i = 0}^{\infty} a_ix^i
\end{equation}
with coefficients $a_i$ coming from the ring $R$. In
 \eqref{FormPowSerAsSum}, $x$ is an indeterminate, i.e. $x$ is not in $R$ and is not a solution 
 of any algebraic equation with coefficients in $R$. Also, even though the 
 symbol $'+'$ is used, the expression does not really represent addition. 
The expression in \eqref{FormPowSerAsSum} is actually a convenient way to express the infinite sequence 
\begin{equation} \label{FormPowSerAsSeq}
(a_n)_{n \in \mathbb{N}}=(a_0, a_1, a_2, \dots, a_i, \dots).
\end{equation}
\noindent The alternative notation \eqref{FormPowSerAsSum} is sometimes preferable. We will use both notations. 
\vskip 0.3cm
\noindent If $(a_n)$ and $(b_n)$ are two elements of $R[[x]]$ then $(a_n) = (b_n)$ if and only if 
$a_i = b_i$ for all $i \in \mathbb{N}$.
\vskip 0.3cm
\noindent Let $\displaystyle \sum_{i = 0}^{\infty} a_ix^i, \displaystyle \sum_{i = 0}^{\infty} b_ix^i 
\in R[[x]]$. Then we define on $R[[x]]$ operations of addition and multiplication as: 
\begin{center}
$\displaystyle \sum_{i = 0}^{\infty} a_ix^i + \displaystyle \sum_{i = 0}^{\infty} b_ix^i
= \displaystyle \sum_{i = 0}^{\infty} (a_i + b_i) x^i$
\end{center}
and
\begin{center}
$\big(\displaystyle \sum_{i = 0}^{\infty} a_ix^i\big) \cdot 
\big( \displaystyle \sum_{j = 0}^{\infty} b_jx^j \big) = \displaystyle \sum_{k = 0}^{\infty} c_k x^k$
\end{center}
where, for each integer $k \geqslant 0$, 
\begin{equation} \label{Theck}
c_k = a_0b_k + a_1b_{k-1} + \cdots + a_kb_0
\end{equation}
With these operations $R[[x]]$ becomes a ring, called the {\sl ring of formal power series over R}.
In this ring, the additive and multiplicative identities are given by
\begin{center} 
$(0,0,0, \dots)$ and $(1,0,0,\dots)$ 
\end{center}
respectively. 
\qed
\end{example}

\begin{lemma}[\cite{L1}, p. 8] \label{leadingCoefficient}
\normalfont
\noindent Let $R[[x]]$ be the ring of all formal power series over $R$ in the indeterminate $x$ as
described in Example \ref{exmFormalPowerSer}. Then an arbitrary formal power series 
$\displaystyle \sum_{i = 0}^{\infty} a_ix^i$ is invertible in $R[[x]]$ if and only if its 
leading coefficient $a_0$ is invertible in $R$. 
\end{lemma}

\proof \space  

\noindent Suppose $\displaystyle \sum_{i = 0}^{\infty} a_ix^i$ is invertible in $R[[x]]$. 
Then there exists $\displaystyle \sum_{j = 0}^{\infty} b_jx^j \in R[[x]]$ such that 
\begin{equation} \label{Invertible1}
\big(\displaystyle \sum_{i = 0}^{\infty} a_ix^i\big) \cdot 
\big( \displaystyle \sum_{j = 0}^{\infty} b_jx^j \big) = \displaystyle \sum_{k = 0}^{\infty} c_k x^k
\end{equation}
and
\begin{equation} \label{Invertible2}
\big(\displaystyle \sum_{j = 0}^{\infty} b_jx^j\big) \cdot 
\big( \displaystyle \sum_{i = 0}^{\infty} a_ix^i \big) = \displaystyle \sum_{k = 0}^{\infty} c_k x^k
\end{equation}
where $c_0 = 1$ and $c_k = 0$ for $k > 0.$ From \eqref{Theck} we must have that 
$1 = c_0 = a_0 \cdot b_0$ and $1 = c_0 = b_0 \cdot a_0$. This means that $a_0 \in R^{-1}$, as required.
\vskip 0.3cm
\noindent Conversely, suppose that $\displaystyle \sum_{i = 0}^{\infty} a_ix^i \in R[[x]]$ 
and that $a_0 \in R^{-1}$. Let $\displaystyle \sum_{j = 0}^{\infty} b_jx^j \in R[[x]]$, 
such that $b_0 = a_0^{-1}$, and $b_j = 0$ for $j > 0$. Then it is easy to see that the inverse of 
$\displaystyle \sum_{i = 0}^{\infty} a_ix^i$ is $\displaystyle \sum_{j = 0}^{\infty} b_jx^j$.
%\ref{Invertible1} and \eqref{Invertible2}

\qed

\begin{example}
\normalfont
\noindent The ring $R=\mathbb{C}[[z]]$ is a local ring.
\end{example}

\proof \space We can write $\mathbb{C}[[z]] = (\mathbb{C}[[z]])^{-1} \bigcup \mathbb{C}[[z]] 
\setminus (\mathbb{C}[[z]])^{-1}.$ The ring $\mathbb{C}$ is a division ring, and so by Lemma
\ref{leadingCoefficient} we can write 
\begin{center}
$\mathbb{C}[[z]] \setminus (\mathbb{C}[[z]])^{-1} 
= \Big\{ \displaystyle \sum_{i = 0}^{\infty} a_iz^i \in \mathbb{C}[[z]] : a_0 = 0 \Big\} $
\end{center}
\noindent We will show that 
$I =  \Big\{ \displaystyle \sum_{i = 0}^{\infty} a_iz^i \in \mathbb{C}[[z]] : a_0 = 0 \Big\} $ 
is an ideal in $\mathbb{C}$. So, let 
$\displaystyle \sum_{i = 0}^{\infty} a_iz^i, 
\displaystyle \sum_{j = 0}^{\infty} b_jz^j \in I$. 
Then $a_0 = b_0 = 0$, hence $a_0 - b_0 = 0$, and $a_0b_0 = 0$. This means that
$\displaystyle \sum_{i = 0}^{\infty} a_iz^i - \displaystyle \sum_{j = 0}^{\infty} b_jz^j \in I$
and 

$\big(\displaystyle \sum_{i = 0}^{\infty} a_iz^i\big) \cdot \big( \displaystyle \sum_{j = 0}^{\infty} b_jz^j \big) \in I$. 
Hence, by Theorem \ref{SubringTest}, $I$ is a subring of $\mathbb{C}[[z]]$.
\vskip 0.3cm
\noindent To see that $I$ is an ideal, let 
$\displaystyle \sum_{i = 0}^{\infty} a_iz^i \in \mathbb{C}[[z]]$ and  
$\displaystyle \sum_{j = 0}^{\infty} b_jz^j \in I$. Then $b_0 = 0$ and so $a_0 b_0 = 0$ and $b_0 a_0 = 0$.
This means that  
$\big(\displaystyle \sum_{i = 0}^{\infty} a_iz^i\big) \cdot \big( \displaystyle \sum_{j = 0}^{\infty} b_jz^j \big) \in I$ and $\big( \displaystyle \sum_{j = 0}^{\infty} b_jz^j \big) \cdot  \big(\displaystyle \sum_{i = 0}^{\infty} a_iz^i\big) \in I$. 
Hence $I$ is an ideal. By Proposition \ref{UniqueMaxIdeal}, $I$ is the unique maximal ideal in 
$\mathbb{C}$, and so $\mathbb{C}$ is a local ring.
 \qed
\begin{proposition}
\normalfont
\noindent If $R$ is a local ring, then the quotient ring $R / \Rad R$ is a division ring. 
\end{proposition}

\proof \space Suppose $R$ is a local ring. Then by Proposition \ref{chrLocalRing} we can write $R$ 
as $R = R^{-1} \cup \Rad R$. For $r \in R$ we denote its equivalence class as $[r] = r + \Rad R$. 
Suppose $[r] \neq [0]$. Then $r = r - 0 \notin \Rad R$. Hence $r \in R^{-1}$, so there exists 
$s \in R$ such that $rs = sr = 1$. But then $[r][s] = [rs] = [1]$ and $[s][r] = [sr] = 1$. 
This shows that every nonzero element of the ring $R / \Rad R$ is invertible. 
Hence $R / \Rad R$ is a division ring. \qed
\begin{comment}
\begin{remark}
\textup{We mention for the sake of completeness, without proof, since it is beyond the scope of this dissertation, that the converse holds only if $R$ is commutative.}
\end{remark}
\end{comment}
\subsection{Homomorphisms}

\begin{definition}[\cite{G1}, p. 121] 
\normalfont
\noindent let $R$ and $S$ be rings. A {\sl ring homomorphism from R to S} is a function 
$f : R \rightarrow S$ such that for $a, b \in R$, we have 
\begin{itemize}
\item [H1.] \quad $f(a+b)=f(a)+f(b)$,
\item [H2.] \quad $f(a \cdot b)=f(a) \cdot f(b)$.
\end{itemize}
\end{definition}
\begin{definition}[\cite{G1}, p. 125]
\normalfont
\noindent Let $R$ and $S$ be rings and $f: R \rightarrow S$ be a ring homomorphism. Then we define the {\sl kernel} of $f$ as 
\begin{center}
$\ker f = f^{-1}(\{0\}) = \{ r \in R : f(r) = 0 \}$.
\end{center}
\end{definition}
\begin{proposition}[\cite{G1}, p. 124] \label{RingHomomOnto}
\normalfont
\noindent Let $R$ and $S$ be rings, and let $f$ be a homomorphism from $R$ onto $S$. 
If $R$ has an identity 1, then $S$ has an identity $f(1)$. \qed
\end{proposition}
\begin{proposition}[\cite{G1}, p. 125] \label{KerIsAnIdeal}
\normalfont
\noindent Let $R$ and $S$ be rings and let $f : R \rightarrow S$ be a ring homomorphism. 
Then the kernel of $f$ is a two sided ideal of $R$. \qed
\end{proposition}

\begin{remark}
\normalfont
\noindent Let $R$ be a ring and $I$ a two sided ideal of $R$. The map
\begin{align*}
\pi_I : R &\rightarrow R/I\\
a &\mapsto a + I
\end{align*}
is a homomorphism (\cite{G}, p. 282) called the {\sl canonical homomorphism} from $R$ to $R/I$.
\vskip 0.3cm
\noindent The canonical homomorphism from $R$ to $R/\Rad R$ will be denoted by $\pi$. Hence for 
$a \in R$, $\pi(a) = a + \Rad R = [a]$.
\qed
\end{remark}

\section{Topologies}

\subsection{General notions}
\begin{definition}[\cite{W2}, p. 23] \label{topology}
\normalfont 
\noindent A {\sl topology} on a set $X$ is a collection $\tau$ of subsets of $X$, called the 
{\sl open sets}, satisfying:
\begin{itemize}
\item [T1.] \quad Any union of elements of $\tau$ belongs to $\tau$,
\item [T2.] \quad Any finite intersection of elements of $\tau$ belongs to $\tau$,
\item [T3.] \quad $\emptyset$ and $X$ belong to $\tau$.
\end{itemize}
\end{definition}

\noindent $\langle X,\tau \rangle$ is called a topological space, abbreviated $X$, 
when the topology $\tau$ is understood.

\vskip 0.3cm

\noindent Given two topologies $\tau_1$ and $\tau_2$ on the same underlying set $X$, we say $\tau_1$ is {\sl weaker}, {\sl smaller} or {\sl coarser} than $\tau_2$, 
alternatively $\tau_2$ is {\sl stronger}, {\sl larger} or {\sl finer} than $\tau_1$ if and only if $\tau_1 \subseteq \tau_2$.
\begin{definition}[\cite{W2}, p. 27] 
\normalfont
If $X$ is a topological space and $A \subseteq X$, then the {\sl interior} of $A$ in $X$ is the set $\inter(A)=\bigcup \{ G \subseteq X: G\ \text{open in}\ X\ \text{and}\ G \subseteq A \}$.

\end{definition}

\noindent An element belonging to $\inter(A)$ is called an {\sl interior point} of $A$.

\vskip 0.3cm
\noindent We will encounter the following examples of topological spaces.

\begin{example} 
\normalfont
Let $X$ be any set and let $\tau = \{\emptyset, X \}$. Then $\tau$ is a topology on $X$, called the 
{\sl trivial (indiscrete)} topology. Clearly it is coarser than any other topology on $X$. \qed
\end{example}
\vspace{-1cm}
\begin{example} 
\normalfont
Let $X$ be any set and let $\tau$ be the collection of all subsets of $X$. Then $\tau$ is a topology on 
$X$, called the {\sl discrete} topology on $X$. It is the strongest topology on $X$.\qed
\end{example}
\vspace{-.5cm}
\begin{definition}[\cite{W2}, p. 24]
\normalfont
\noindent Let $\langle X, \tau \rangle$ be a topological space. Then $A \subseteq X$ is said to be 
{\sl closed} if its {\sl complement} is open, i.e. if $X \setminus A \in \tau$. 
\end{definition}

\noindent The following theorem is a consequence of De Morgan's laws in conjunction
with the obvious duality between the notions of open set and closed set.

\begin{theorem}[\cite{W2}, p. 24] \label{defnClosedSets}
\normalfont
If $\mathscr{F}$ is the collection of closed sets in a topological space $X$, then 
\begin{itemize}
\item [F1.] \quad Any intersection of members of  $\mathscr{F}$ belongs to  $\mathscr{F}$,
\item [F2.] \quad Any finite union of members of  $\mathscr{F}$ belongs to  $\mathscr{F}$,
\item [F3.] \quad $X$ and $\emptyset$ both belong to  $\mathscr{F}$.
\end{itemize}
\noindent Conversely, given a set $X$, and a family  $\mathscr{F}$ of subsets of $X$ that satisfies
conditions F1, F2 and F3, the collection of complements of members of  $\mathscr{F}$ is a topology 
on $X$ in which the family of closed sets is just  $\mathscr{F}$. \qed
\end{theorem}
\begin{definition}[\cite{W2}, p. 25] 
\normalfont
If $\langle X, \tau \rangle$ is a topological space and $A \subseteq X$, then the {\sl closure} of $A$ in $X$ is the set 
\begin{center}
$\cl_{\tau}(A) = \bigcap\ \{K \subseteq X : K\ \text{closed in}\ \langle X, \tau \rangle \
 \textup{and}\ A \subseteq K \}$.
\end{center}
\end{definition}
\noindent If the topology is understood, we will omit the subscript $\tau$ and write 
$\cl(A)$ to mean $\cl_{\tau}(A)$.
\vskip 0.3cm
\noindent By property F1 from Theorem \ref{defnClosedSets}, for $A \subseteq X$, 
the set $\cl_{\tau}(A)$ is closed. It is the smallest closed set containing $A$ in the sense 
that it is contained in every closed set containing $A$.
\vskip 0.3cm
\noindent In Definition \ref{topology}, we defined a topology by specifying the open sets. 
Often topologies are defined by using a {\sl closure operation}, described next.

\begin{definition} [\cite{W2}, p. 25] \label{defnKuratowski}
\normalfont
\noindent Let $X$ be a set and let
\begin{center}
$k:\powerset(X) \rightarrow \powerset(X)$ 
\end{center}
be an operation assigning to each $A \subseteq X$ the subset $k(A) \subseteq X$ satisfying the 
following properties for all $A,B \subseteq X$ and $\emptyset$:

\begin{itemize}
\item [K1.] \quad $A \subseteq k(A)$ 
\item [K2.] \quad $k(k(A))=k(A)$  
\item [K3.] \quad $k(A \cup B) = k(A) \cup k(B)$ 
\item [K4.] \quad $k(\emptyset)=\emptyset$.
\end{itemize}

\noindent Then $k$ is called a {\sl Kuratowski closure operation} on $X$.

\end{definition}

\begin{theorem}[\cite{W2}, p. 25] \label{thmKuratowski}
\normalfont
Suppose $X$ is a set and $k:\powerset(X) \rightarrow \powerset(X)$ is a Kuratowski closure operation 
on the set $X$. For $A \subseteq X$ we define:
\begin{itemize}
\item [K5.]  \quad $A$ is closed in $X$ if and only if $k(A) = A.$
\end{itemize}
The result is a topology on $X$, whose closure operation is just the closure operation we 
started with.
\end{theorem}
\proof \space
Suppose that $X$ is a set and that $k:\powerset(X) \rightarrow \powerset(X)$ is a Kuratowski closure
 operation on $X$. We define
\begin{center}
$\mathscr{F}_0 = \{A \subseteq X : k(A) = A \}$
\end{center}
\noindent We show that $\mathscr{F}_0$ satisfies the conditions of Theorem \ref{defnClosedSets} and hence defines a topology on $X$. 
\vskip 0.3cm
\noindent First we show that $A \subseteq B \implies k(A) \subseteq k(B)$. If $A \subseteq B$ 
then $B = A \cup (B \setminus A)$. From K3 we have that $k(B) = k(A) \cup k(B \setminus A)$, 
from which it follows that $k(A) \subseteq k(B)$. Hence we have shown that
\begin{equation}\label{SubSets}
A \subseteq B \implies k(A) \subseteq k(B)
\end{equation}
Next, suppose that $F_{\lambda} \in \mathscr{F}_0$ for each $\lambda \in \Lambda$. 
Then for each $\lambda \in \Lambda$ we have:
\begin{equation}\label{AppSubSets}
\bigcap\limits_{\lambda \in \Lambda} F_{\lambda} \subseteq F_{\lambda}
\end{equation}
We apply \eqref{SubSets} above to \eqref{AppSubSets} to give us
\begin{equation} \label{IntSubsets}
k\bigg(\bigcap\limits_{\lambda \in \Lambda} F_{\lambda}\bigg) \subseteq k(F_{\lambda})
\end{equation}
From \eqref{IntSubsets} and the fact that  $k(F_{\lambda}) = F_{\lambda}$ for all $\lambda \in \Lambda$
we have
\begin{equation} \label{Inclusion1}
k\bigg(\bigcap\limits_{\lambda \in \Lambda} F_{\lambda}\bigg) 
\subseteq \bigcap\limits_{\lambda \in \Lambda}k(F_{\lambda}) = \bigcap\limits_{\lambda \in \Lambda} F_{\lambda}
\end{equation}
From K1 we have that
\begin{equation} \label{Inclusion2}
\bigcap\limits_{\lambda \in \Lambda} F_{\lambda} \subseteq k\bigg(\bigcap\limits_{\lambda \in \Lambda} F_{\lambda}\bigg)
\end{equation}
From \eqref{Inclusion1} and \eqref{Inclusion2} we have that
\begin{center}
$\bigcap\limits_{\lambda \in \Lambda} F_{\lambda} = k\bigg(\bigcap\limits_{\lambda \in \Lambda} F_{\lambda}\bigg)$
\end{center}
which shows that F1 is satisfied by $\mathscr{F}_0$.
\vskip 0.3cm
\noindent Next, suppose that $F_1, \cdots, F_n \in \mathscr{F}_0$. Then by K3 and induction we have that:
\begin{align*}
k(F_1 \cup \cdots \cup F_n) 
&= k(F_1) \cup \cdots \cup k(F_n)\\
&= F_1 \cup \cdots \cup F_n.
\end{align*}
Hence $\mathscr{F}_0$ satisfies F2.
\vskip 0.3cm
\noindent Finally, by K4 we have that $\emptyset \in \mathscr{F}_0$ and by K1 we have 
that $X \in \mathscr{F}_0$, so that $\mathscr{F}_0$ satisfies F3. Since the collection $\mathscr{F}_0$
satisfies the conditions of Theorem \ref{defnClosedSets}, $\mathscr{F}_0$ defines a topology on $X$, 
in which $\mathscr{F}_0$ is the collection of closed sets.
\vskip 0.3cm
\noindent It remains to show that the closure of a set in the generated topology is simply the 
closure operation we initially started with. 
What we need to show is that if $A \subseteq X$, then $k(A)$ is the smallest closed set containing $A$. 
\vskip 0.3cm
\noindent First we note that by K2, we have that $k(k(A)) = k(A)$, hence we know that $k(A)$ is a 
closed set in the generated topology. From K1, we have that $k(A)$ is a closed set that contains $A$. 
Next, let $K$ be any element from $\mathscr{F}_0$ containing $A$. 
Then from \eqref{SubSets} 
\begin{center}
$A \subseteq K \implies k(A) \subseteq k(K) = K$. 
\end{center}
Hence $k(A)$ is the smallest element of $\mathscr{F}_0$ containing $A$.
\qed

\begin{example} [\cite{W2}, p. 26] \label{cofiniteTopology}
\normalfont
\noindent 
Let $X$ be an infinite set and for $A \subseteq X$ we define \\
$k_0 : \powerset(X) \rightarrow \powerset(X)$ as:
\[ 
k_0(A) =
\begin{cases} 
      A & \text{if A is finite}\\
      X & \text{if A is infinite} 
   \end{cases}
\]

\noindent The properties K1 to K4 can be verified for the operation $k_0$, and it is a 
Kuratowski closure operation on $X$. The generated topology is called the 
{\sl co-finite topology}. It has as closed sets, all sets $A$ such that $k_0(A) = A$, hence the closed sets are $X, \emptyset$ and $A \subseteq X$ 
such that $A$ is finite. The open sets are $X, \emptyset$ and $A \subseteq X$ such 
that $X \setminus A$ is finite.
\qed
\end{example}
\vspace{-1cm}

\begin{comment}
\begin{definition}[\cite{W2}, p. 41] 
\normalfont
If $\langle X, \tau\rangle$ is a topological space and $A \subseteq X$, then the collection $\tau' = \{ G \cap A : G \in  \tau \}$ is a topology for $A$, called the {\sl relative topology} for $A$. 
The fact that a subset of $X$ is being given this topology, is signified by referring to it as a 
{\sl subspace} of $X$.

\end{definition}
\end{comment}
\begin{definition}[\cite{W2}, p. 38] 
\normalfont
\noindent If $\langle X, \tau \rangle$ is a topological space, a {\sl base} for $\tau$ is a collection $\mathscr{B} \subseteq \tau$ such that 
\begin{center}
$ \tau = \Big \{ \bigcup_{B \in \mathscr{C}} B : \ \mathscr{C} \subseteq \mathscr{B} \Big \} $
\end{center}
\end{definition}

\begin{definition}[\cite{W2}, p. 39] 
\normalfont
If $\langle X, \tau \rangle$ is a topological space, a {\sl subbase} for $\tau$ is a collection 
$\mathscr{D} \subseteq \tau$ such that the collection of all finite intersections of elements from 
$\mathscr{D}$ forms a base for $\tau$.

\end{definition}



\begin{definition}[\cite{W2}, p. 31] 
\normalfont
If $X$ is a topological space and $x \in X$, a {\sl neighbourhood} of $x$ is a set $U$ which contains 
an open set $V$ containing $x$. The collection $\mathcal{N}_x$ of all neighbourhoods of $x$ is the 
{\sl neighbourhood system} at $x$. 
\end{definition}

\begin{definition}[\cite{W2}, p. 35] \label{dfnClusterPoint}
\normalfont
\noindent
An {\sl accumulation point (cluster point)} of a set $A$ in a topological space $X$ is a point 
$x \in X$ such that each neighbourhood of $x$ contains some point of $A$, other than $x$. The set $\der A$ is the set of accumulation points of $A$.
\end{definition}
\begin{theorem} [\cite{W2}, p. 35] \label{ClosureITODerivedSet}
\normalfont
\noindent Let $\langle X, \tau \rangle$ be a topological space, and let $A \subseteq X$. Then 
\begin{center}
$\cl_{\tau}(A) = A \cup \der_{\tau} A$. 
\end{center}
\end{theorem}
\qed
\noindent When the topology is understood, the symbol $\overline{A}$ is often used to 
represent the closure of $A$.
\begin{definition}[\cite{W2}, p. 70]
\normalfont
\noindent A sequence $(x_n)$ in a topological space $X$ is said to {\sl converge} to $x \in X$ 
written $x_n \xlongrightarrow[n]{\infty} x$ if and only if for each neighbourhood $U$
of $x$, there exists $N \in \mathbb{N}$ such that $n \geq N \implies x_n \in U$.
\end{definition}

\begin{lemma} \label{2TopologiesOnOneSet}
\normalfont
Let $X$ be a set with topologies $\tau$ and $\sigma$ such that $\cl_{\tau}(A) \subseteq \cl_{\sigma}(A)$ for every $A \subseteq X$. Then $\sigma \subseteq \tau$.

\end{lemma}

\proof \space Let $A \in \sigma$. Then $X \setminus A$ is closed in $X$ with respect to $\sigma$ 
giving $X \setminus A = \cl_{\sigma}(X \setminus A)$. 
By assumption $\cl_{\tau}(X \setminus A) \subseteq \cl_{\sigma}(X \setminus A)= X \setminus A$, 
hence $\cl_{\tau}(X \setminus A)=X \setminus A$. 
Hence $A \in \tau$, and so $\sigma \subseteq \tau$. \qed
\begin{comment}
\begin{proposition}[\cite{W2}, p. 42] 
\normalfont
\noindent Let $A$ be a subspace of a topological space $X$. 
Then $E \subseteq A \implies \cl_A(E)=A \cap \cl_X(E)$. \qed
\end{proposition}
\end{comment}

\subsection{Product and quotient topologies}

\begin{definition} [\cite{W2}, p. 52] 
\normalfont
Let $\{ {\langle X_i , \tau_i \rangle\ : i \in I} \}$ be any collection of topological spaces, 
with index set $I$. 
Denote by $X = \prod_i  X_i = \{ t = \langle x_i\ : i \in I  \rangle : x_i \in X_i\}$ the 
{\sl Cartesian product} of $X_i's$, the set of all functions $t$ defined on the index 
set $I$ such that the value of the function at a particular index $i$ is an element of $X_i$.

\end{definition}

\noindent The map $p_j : \prod_i X_i \to  X_j$ defined by $p_j(t)=x_j$ returns the 
$j^{\text{th}}$ coordinate for the tuple $t= \langle x_i\ : i \in I \rangle$ and is called 
the $j^{\text{th}}$ {\sl projection map}.

\begin{definition}[\cite{W2}, p. 53] 
\normalfont
The {\sl product topology} on $X=\prod_i X_i$ is obtained by taking as a base for the open sets, sets of the form $\prod_i U_i$, where

\begin{itemize}

\item [P1.] \quad $U_i$ is open in $X_i$, for each $i \in I$.

\item [P2.] \quad  For all but finitely many coordinates, $U_i=X_i$.

P1 can be replaced by

\item [P$1'$.] \quad  $U_i \in \mathscr{B}_i$ where for each $i$, $\mathscr{B}_i$ is a (fixed) base for the topology of $X_i$.
\end{itemize}
\vskip 0.3cm
\noindent Also, notice that the set $\prod_i U_i$, where $U_i=X_i$ except for $i=i_1, \dots, i_n$ can be written

$$ \prod_i U_i = p_{i_1}^{-1}(U_{i_1}) \cap \dots \cap p_{i_n}^{-1}(U_{i_n})$$

\noindent Thus the product topology is precisely that topology which has for a subbase the
 collection $\{ p_i^{-1}(U_i): i \in I, U_i\ \text{is open in}\ X_i \}$. Again, the sets $U_i$ 
can be restricted to come from some fixed base (in fact, in this case, subbase) in $X_i$.

\end{definition}

\noindent Wherever necessary, we will denote the product topology by $\tau^{\times}$ and the product topological space by $\langle X, \tau^{\times} \rangle$, where $X=\prod_i X_i$. 
Hereafter, $X=\prod_i X_i$ is always assumed to be endowed with the product topology if each $X_i$ is a topological space.


\begin{proposition}[\cite{W2}, p. 54] 
\normalfont
The product topology $\tau^{\times}$ on $X$ is the weakest topology on $X$ with respect to 
which all the projection functions $p_i : X \rightarrow  X_i$ defined by 
$p_{j_0}(\langle x_i : i \in I \rangle) =x_{j_0}$ for each $j_0 \in I$ are continuous. \qed
\end{proposition}

\begin{definition}[\cite{W2}, p. 59] 
\normalfont
Let $\langle X, \tau \rangle$ be a topological space, $Y$ a set and $g : X \rightarrow Y$ be 
an onto map.
Then the collection $\tau_g$ of subsets of $Y$ defined by
\begin{center}
$\tau_g = \{G \subseteq Y : g^{-1}(G)$ is open in $ X \}$
\end{center}
is a topology on $Y$, called the {\sl quotient topology} induced on $Y$ by $g$. When $Y$ is given 
some such topology, it is called a {\sl quotient space} of $X$, and the inducing map is called 
the {\sl quotient map}. 
\end{definition}

\begin{theorem}[\cite{W2}, p. 59] \label{QuotientTopology}
\normalfont
If $X$ and $Y$ are topological spaces and $f:X \rightarrow Y$ is continuous and either open 
or closed, then the topology $\tau$ on $Y$ is the quotient topology $\tau_f$.
\qed
\end{theorem}


\subsection{Separation}

\begin{definition}[\cite{W2}, p. 85] \label{defnT0}
\normalfont
\noindent A topological space $X$ is a $T_0$-{\sl space} (or the topology on $X$ is $T_0$) if and only if whenever $x$ and $y$ are distinct points in $X$, there is an open set containing one and not the other.

\end{definition}

\begin{definition}[\cite{W2}, p. 86] \label{defnT1}
\normalfont
\noindent A topological space $X$ is a $T_1$-{\sl space} if and only if whenever $x$ and $y$ are distinct points in $X$, there is a neighbourhood of each not containing the other.
\end{definition}


\begin{definition}[\cite{W2}, p. 86]\label{defnT2}
\normalfont
\noindent A topological space $X$ is a $T_2$-{\sl space} or a {\sl Hausdorff} space if and only if whenever $x$ and $y$ are distinct points in $X$, there are disjoint open sets $U$ and $V$ in $X$ with $x \in U$ and $y \in V$.

\end{definition}


\begin{remark} \label{T0T1T2}
\normalfont
\noindent It follows from Definitions \ref{defnT0}, \ref{defnT1} and \ref{defnT2} that 
every $T_2$ space is a $T_1$ space and every $T_1$ space is a $T_0$ space, 
but the converse statements are not true in general. 
\vskip 0.3cm
\noindent For example, the set $X=\{a,b\}$ equipped with the topology $\tau=\{\emptyset, \{a\}, X\}$ is 
a $T_0$ space which is not $T_1$. 
\vskip 0.3cm
\noindent Similarly an infinite set equipped with the co-finite topology is a $T_1$ space, 
since singleton sets are closed. Such a topology is not $T_2$, since no two nonempty 
open sets are disjoint.

\end{remark}

\begin{proposition}[\cite{W2}, p. 86] \label{SeparationEquiv}
\normalfont
\noindent If $X$ is a topological space, then the following are equivalent:
\begin{enumerate}[label=(\alph*)]
\item $X$ is $T_1$,
\item each one-point set in $X$ is closed,
\item each subset of $X$ is the intersection of the open sets containing it. \qed
\end{enumerate}
\end{proposition}

\subsection{Continuity}

\begin{definition}[\cite{W2}, p. 44] \label{defnContinuity}
\normalfont
\noindent Let $X$ and $Y$ be topological spaces. 
A function $f: X \rightarrow Y$ is said to be continuous at $x_0$ if and only if for each neighbourhood $V$ of $f(x_0)$ in $Y$ there is a neighbourhood $U$ of $x_0$ in $X$ such that $f(U) \subseteq V$. 
We say that $f$ is {\sl continuous on} $X$ if and only if $f$ is continuous at each $x_0 \in X$.
\end{definition}

\begin{proposition}[\cite{W2}, p. 44] \label{charContinuity}
\normalfont
\noindent Let $X$ and $Y$ be topological spaces and 
\begin{center}
$f: X \rightarrow Y$
\end{center}
Then the following are equivalent:
\begin{enumerate}[label=(\alph*)]
\item $f$ is continuous,
\item for each open set $H$ in $Y$, $f^{-1}[H]$ is open in $X$,
\item for each closed set $K$ in $Y$, $f^{-1}[K]$ is closed in $X$,
\item for each $E \subset X$, $f[\cl_X E]\subset \cl_Y f[E]$. \qed
\end{enumerate}
\end{proposition}

\begin{definition}[\cite{W2}, p. 46]
\normalfont
\noindent If $X$ and $Y$ are topological spaces, a function $f$ from $X$ to $Y$ is a {\sl homeomorphism}
 if and only if $f$ is one to one, onto, continuous and has a continuous inverse, $f^{-1}$.  
 In this case, we say that $X$ and $Y$ are homeomorphic.

\end{definition}
\vskip 0.3cm

\noindent Evidently, a continuous map $f: X \to Y$ is a homeomorphism if and only if there is a continuous map $g: Y \to X$ such that the compositions $g \circ f$ and $f \circ g$ are the identity maps on $X$ and $Y$ respectively.
%\begin{comment}
\subsection{Compactness}

\begin{definition}[\cite{W2}, p. 104]
\normalfont
\noindent Let $X$ be a topological space. A {\sl cover} (or {\sl covering}) of $X$ is a collection 
$\mathcal{A}$ of subsets of $X$ whose union is $X$. A subcover of a cover $\mathcal{A}$, 
is a subcollection $\mathcal{A}'$ of $\mathcal{A}$ which is a cover. 
An {\sl open cover} is a cover consisting of open sets.
\end{definition}

\begin{definition}[\cite{W2}, p. 116]
\normalfont
\noindent Let $X$ be a topological space. $X$ is {\sl compact} if and only if every open cover of $X$ 
has a finite subcover.
\end{definition}
%\end{comment}
\section{Topological groups and topological rings}

\begin{definition}[\cite{W1}, p. 13] 
\normalfont
\noindent A topology $\tau$ on a group $G$, denoted multiplicatively, is a {\sl group topology} and $G$, furnished with $\tau$, is a {\sl topological group} if the following conditions hold:
\begin{itemize}
\item [TG1.] \quad $(x,y) \mapsto xy$ is continuous from $G \times G$ to $G$,
\item [TG2.] \quad  $x \mapsto x^{-1}$ is continuous from $G$ to $G$,
\end{itemize}
\noindent where $G$ is given topology $\tau$ and $G \times G$ carries the Cartesian product 
topology $\tau^{\times}$ determined by $\tau$.

\end{definition}


\begin{definition}[\cite{W1}, p. 1] \label{dfnTopRing}
\normalfont
\noindent A topology $\tau$ on a ring $R$ is a {\sl ring topology} and $R$, furnished with $\tau$, 
is a {\sl topological ring} if the following conditions hold:
\begin{itemize}
\item [TR1.] \quad  $(x,y) \mapsto x+y$ is continuous from $R \times R$ to $R$,
\item [TR2.] \quad  $x \mapsto -x$ is continuous from $R$ to $R$,
\item [TR3.] \quad  $(x,y) \mapsto xy$ is continuous from $R \times R$ to $R$,
\end{itemize}

\noindent where $R$ is given topology $\tau$ and $R \times R$ carries the Cartesian product 
topology $\tau^{\times}$ determined by $\tau$.

\end{definition}


\begin{remark}
\normalfont
\noindent Notice that every topological ring is also a topological group with respect to addition and 
we call $-: A \rightarrow A$ defined by $x \mapsto -x$ the inversion map.
\qed
\end{remark}

\begin{theorem}[\cite{W1}, p. 14] \label{thmTranslationThm}
\normalfont
\noindent Let $G$ be a topological group and let $a \in G$. The functions 
$x \rightarrow -x$, $x \rightarrow a + x$ and $x \rightarrow x + a$ are homeomorphisms from $G$ to $G$. 
Consequently, for any $X \subseteq G$, we have 
$\overline{-X} = - \overline{X}$, 
$\overline{a + X} =  a + \overline{X}$, $\overline{X + a} = \overline{X} + a$. Also, for any open 
(closed) subset $P \subseteq G$ we have that $, -P, a + P$ are open (closed).

\qed
\end{theorem}
\vspace{-1cm}
\begin{lemma}[\cite{W1}, p. 15]  \label{MinVANbhood}
\normalfont
\noindent Let $G$ be a topological group. If $V$ is a neighbourhood of zero, so is $-V$. \qed

\end{lemma}




\vspace{-1cm}

\section{Banach algebras}

\begin{definition} [\cite{K}, p. 59]
\normalfont
Let $V$ be a vector space over a field $\mathbb{K}$. 
A {\sl norm} on $V$ is a map $\|\cdot\| : V \rightarrow R^+$ that satisfies the following properties,
for all $x, y \in V$ and $\alpha \in \mathbb{K}$:
\begin{itemize}
\item[N1.] \quad  $\|x \| \geqslant 0$,
\item[N2.] \quad  $\|x \| = 0 \iff x = 0$,
\item[N3.] \quad  $\| \alpha x \| = |\alpha| \|x\|$,
\item[N4.] \quad  $\| x + y \| \leqslant \|x\| + \|y\|$.
\end{itemize}

\noindent A vector space $V$ with a norm $\|\cdot\|$ defined on it is called a {\sl normed} vector space,
 denoted by $\langle V, \|\cdot\| \rangle$.  One now constructs (or induces) a {\sl metric} using the norm, by defining the distance between $a, b \in V$ as
\begin{center}
$d(a, b) = \|a - b\|$.
\end{center}
\noindent A {\sl Banach space} is a normed vector space which is complete in the metric induced by the 
norm. 
\end{definition}
\begin{definition}
\normalfont
\noindent An {\sl algebra} is a vector space $A$ over a field $\mathbb{K}$, with a multiplication
operation such that for all $x, y, z \in A$  and  $\lambda \in \mathbb{K}$:
\begin{itemize}
\item[BA1.] \quad  $x(yz) =  (xy)z$,
\item[BA2.] \quad  $(x + y)z = xz + yz$,
\item[BA3.] \quad  $x(y + z) = xy + xz$,
\item[BA4.] \quad  $\lambda(xy) = (\lambda x)y = x(\lambda y)$.
\end{itemize}
\end{definition}
\noindent If, in addition, $A$ is a Banach space for a norm $||\cdot||$ and satisfies the norm inequality 
$||xy|| \leq ||x||\cdot||y||$, for all $x, y \in A$, we say that $A$ is a {\sl Banach algebra.}
\vskip 0.3cm
\noindent If the field $\mathbb{K}$ is either the set of real numbers $\mathbb{R}$ or complex numbers 
$\mathbb{C}$, the Banach algebra is called a {\sl real} or {\sl complex} Banach algebra, respectively.
\vskip 0.3cm
\noindent An {\sl identity} of $A$ is an element $1 \in A$ such that for all $x \in A$ we have that 
$1x = x1 = x $. We use $1_A$ to represent the identity, when emphasizing the Banach algebra $A$. 
If a Banach algebra has an identity, it is unique and the Banach algebra is called {\sl unital}. 
The proof that the identity is unique is essentially the same as the proof that the identity in a ring 
is unique. Left and right invertible (and invertible) elements in a Banach algebra are defined as they 
are in a ring.
\vskip 0.3cm
\noindent We make a normed space $\langle X, || \cdot || \rangle$ into a topological space, via the concept of an open ball as follows. For $x_0 \in X$ and $\epsilon \in \mathbb{R}^+$, we define the {\sl open ball of radius $\epsilon$, centered at $x_0$} as 
\begin{center}
$B(x_0, \epsilon):=\{x \in X: || x-x_0 || < \epsilon\}$. 
\end{center}
\noindent Next, we define an arbitrary set in the space $\langle X, || \cdot || \rangle$ to be open if it contains a ball about each of its points. 

\begin{theorem}[\cite{A}, p. 35] \label{oneMinXInvert}
\normalfont
Suppose that $A$ is a Banach algebra with identity $1$. If $x \in A$ and $||x|| < 1$ 
then $1 - x \in A^{-1}$ and
\begin{center}
 $\displaystyle{ (1-x)^{-1}=\sum_{k=0}^{\infty} x^k}$ \quad where $x^0=1$.
\end{center}\qed
\end{theorem}
%\vskip 0.3cm


\begin{theorem}[\cite{A}, p. 36] \label{InvertiblesIsOpen}
\normalfont
Suppose that $A$ is a Banach algebra and that $a$ is invertible. If $||x - a|| < \dfrac{1}{||a^{-1}||}$,
then $x$ is invertible. Moreover the mapping $x \mapsto x^{-1}$ is a homeomorphism from $A^{-1}$ 
onto $A^{-1}$.\qed
\end{theorem} 

\begin{remark}
\normalfont
Theorem \ref{InvertiblesIsOpen} proves that the set of invertible elements in a Banach algebra is an 
open set in the topology induced by the norm. \qed
\end{remark}




\begin{proposition}[\cite{CH2}, p. 3549] \label{IntersectLInvAndRInv}
\normalfont
\noindent In a Banach algebra $A$, we have: 
\begin{center}
$\cl_{\| \cdot \|}(A_l^{-1}) \cap A_r^{-1}=A^{-1}=A_l^{-1} \cap \cl_{\| \cdot \|}(A_r^{-1})$
\end{center}
\qed
\end{proposition}

\subsection{Spectral theory}

\begin{definition}[\cite{H1}, p. 341] 
\normalfont
Let $A$ be a unital complex Banach algebra. The {\sl spectrum} of $a \in A$ is defined 
as the set 
\begin{center}
$\sigma(a) = \{ \lambda \in \mathbb{C}: \lambda -a \not \in A^{-1} \}$.
\end{center}
\end{definition}
%\noindent It is easily verified that $\sigma(0)=\{0\}$ and $\sigma(1)=\{1\}$.
\begin{definition}[\cite{H1}, p. 352] 
\normalfont
\noindent Let $A$ be a unital complex Banach algebra. 
The {\sl spectral radius} of $a \in  A$ is defined as 
\begin{center}
$\rho(a) = \max \{ |\lambda|: \lambda \in \sigma(a) \}$. 
\end{center}
\end{definition}
It follows easily that $\rho(a)=0$ if and only if $\sigma(a)=\{0\}$. In fact, we have the characterization:
\begin{theorem}[\cite{A}, Theorem 3.2.8 p. 38] \label{SpectrumCompact}
\normalfont
\noindent Let $A$ be a unital complex Banach algebra and let $a \in A$. Then
\begin{enumerate}[label=(\alph*)] 

\item $\lambda \rightarrow (\lambda 1 - a))^{-1}$ is analytic on $\mathbb{C} \setminus \sigma(a)$ 
and goes to zero at infinity,
\item $\sigma(a)$ is compact and non empty,
\item $\rho(a) = \displaystyle{\lim_{n \to \infty}} \| a^{n} \|^{\frac{1}{n}}$. 
\end{enumerate}
\qed
\end{theorem}

\begin{definition}[\cite{H1}, p. 251] 
\normalfont
\noindent If $A$ is a normed algebra and $a \in A$, then $a$ is {\sl quasinilpotent} if 
$\displaystyle{\| a^n \|^{\frac{1}{n}} \xlongrightarrow[n]{\infty} 0}$. The set of quasinilpotent 
elements in $A$ is $\QN_{\| \cdot \|}(A)$.
\end{definition}


\begin{proposition}[\cite{H1}, p. 354] \label{QNs}
\normalfont
\noindent An element $a$ in a Banach algebra is quasinilpotent if and only if $\sigma(a)=\{0\}$.
\qed
\end{proposition}




\begin{definition}[\cite{G2}, p. 72]  
\normalfont
\noindent An element $a$ of a Banach algebra $A$, is called a {\sl left (right) topological zero divisor} if and only if there exists a sequence $(x_n)$ in $A$ such that $\|x_n\|=1$ for all $n \in \mathbb{N}$ and $a x_n \rightarrow 0 \quad (x_na \rightarrow 0)$. An element which is both a left and right topological zero divisor, is said to be a {\sl (two-sided) topological zero divisor}.

\end{definition}

\begin{proposition} [\cite{Muller}, p. 6] \label{boundaryOfInv}
\normalfont
\noindent Let $A$ be a unital Banach algebra with invertible group $A^{-1}$.
If $a \in A$ is any element such that $a \in \partial A^{-1}$ then $a$ is a topological divisor of zero.
\qed
\end{proposition}

\begin{example} 
\normalfont
\noindent Any quasinilpotent element in a Banach algebra is a topological zero divisor.
\vskip 0.3cm

\proof \space Suppose $a \in \QN_{\| \cdot \|}(A)$. Then from Proposition \ref{QNs} we have that 
$\sigma(a)=\{0\}$. It follows that $a \not \in A^{-1}$. Let $(\lambda_n)$ be any sequence in 
$\mathbb{C}$ such that $\lambda_n \neq 0$ for all $n \in \mathbb{N}$ and 
$\displaystyle{\lim_{n \to \infty}} \lambda_n=0$. Then clearly $(a - \lambda_n)$ is a sequence in $A^{-1}$ 
such that $\displaystyle{  \lim_{n \to \infty} (a - \lambda_n})=a$. 
Hence it follows that $a \in \partial A^{-1}$. By Proposition \ref{boundaryOfInv} $a$ is a topological divisor of zero.
\qed

\end{example}


\begin{example}[\cite{A}, p. 19] 
\normalfont
\noindent We construct a counterexample to show that, in the infinite dimensional case, quasinilpotence does not imply nilpotence. Define the unit square $S=[0,1] \times [0,1] \subseteq \mathbb{R}^2$ with Lebesgue measure $\eta$. Then the Riemann-Liouville operator on Hilbert space $\mathcal{L}^{2}(S,\eta)$ is defined by: $$(V^{\alpha}f)(x)= \frac{1}{\Gamma(\alpha)} \int_{0}^{x} (x-t)^{\alpha-1} f(t) dt $$ where $\alpha$ is a complex number with positive real part
\vskip 0.3cm

\noindent If we let $\alpha=1$, we recover the Volterra operator: 
\begin{center}
$(V f)(x)=\int_{0}^{x}f(t) dt$
\end{center}

with the kernel on $S$ defined by  $K(x,t)= \begin{cases} 
							      						1 & \text{if} \quad x \geq t \\
      													0 & \text{if} \quad x < t
							   							\end{cases}$

\vskip 0.3cm
\noindent By the famous Arzel{\'a}-Ascoli theorem, we have that since $V$ is equicontinuous and bounded, consequently compact, it must therefore be quasinilpotent. However, by direct calculation, we can verify that if $f=1$ identically, then $V$ is not nilpotent:
\vskip 0.3cm
\noindent $\displaystyle{ V(x)=\int_0^x 1\ dt = t\ \Big|_0^x =x-0 = x \not = 0} \qquad \implies \qquad V^{\alpha}f \not = 0$ for all $\alpha \in \mathbb{N}$

\end{example}



\subsection{Basic Operator Theory}

\noindent In this section we briefly discuss some basic notions in Operator Theory. Our aim is to have
the notions of Fredholm and Weyl operators available to us for later discussions. 
\vskip 0.3cm

\noindent Let $X$ and $Y$ be Complex Banach spaces. By an {\sl operator} $T : X \rightarrow Y$ 
we mean a bounded linear mapping. The set of all bounded linear mappings is denoted by 
$\mathcal{BL}(X, Y)$. We denote the range of $T$ by $\ran T$.
\vskip 0.3cm
\noindent If $X = Y$ then $\mathcal{BL}(X,Y)$ is denoted by $\mathcal{BL}(X)$. 
\vskip 0.3cm
\noindent Let $T \in \mathcal{BL}(X)$. Then we say $T$ has {\sl finite rank} if $\dim (T(X)) < \infty.$
$T$ is called a {\sl compact} operator on $X$ if $\overline{T(U)}$ is compact, where $U$ is the 
closed unit ball of $X$. The finite rank operators in $\mathcal{BL}(X)$ form an ideal, denoted by 
$\mathcal{F}(X)$, and the compact operators form a closed ideal, denoted by $\mathcal{K}(X)$.
\vskip 0.3cm
\begin{example} [\cite{BMSW}, p. 3]
\normalfont Let $X$ be a Banach space. Denote by $l_{\infty}(X)$ the linear space of all bounded
sequences $(x_n)$ of elements $x_n \in X$ with the supremum norm:
\begin{center}
$\|(x_n)\| = \sup \{\|x_n\| : n \in \mathbb{N} \}.$
\end{center}
\noindent $\m(X)$ is the linear subspace of $l_{\infty}(X)$ consisting of those sequences 
every subsequence of which contains a convergent subsequence. It is elementary to check that 
$l_{\infty}(X)$ is a Banach space and that $\m(X)$ is a closed subspace of $l_{\infty}(X)$. 
\vskip 0.3cm
\noindent Let $\hat{X}$ denote the quotient space $l_{\infty}(X) / \m(X)$ and if $T \in \mathcal{BL}(X)$
let $\hat{T}$ denote the operator on $\hat{X}$ defined by
\begin{center}
$\hat{T} ((x_n) + \m(X)) = (Tx_n) + \m(X).$
\end{center}
\qed
\end{example}
\begin{definition} [\cite{Muller}, p. 149]
\normalfont
\noindent Let $X, Y$ be Banach spaces, and let $T \in \mathcal{BL}(X, Y)$. We say that
\begin{enumerate}[label=(\alph*)]
\item $T$ is {\sl upper semi-Fredholm} if $\ran T$ is closed and $\dim \ker T < \infty$;
\item $T$ is {\sl lower semi-Fredholm} if $\codim \ran T = \dim Y/ \ran T < \infty$;
\item $T$ is {\sl Fredholm} if $\dim \ker T < \infty$ and $\codim \ran T < \infty$.
\end{enumerate} 
\end{definition}

\begin{definition}[\cite{Muller}, p. 150]
\normalfont
\noindent Let $X, Y$ be Banach spaces and let $T \in \mathcal{BL}(X, Y)$. Suppose also that $T$ has 
closed range. Then we define:
\begin{center}
$\alpha(T) = \dim \ker T$ \qquad and \qquad $\beta (T) = \codim \ran T$. 
\end{center}
We also define  the {\sl index} of $T$ as:
\begin{center}
$\iota (T) = \alpha(T) - \beta(T)$.
\end{center}
\end{definition}

\noindent We denote the class of Fredholm operators acting between Banach spaces $X$ to $Y$ by 
$\Phi(X,Y)$
and if $X=Y$ we simply write $\Phi(X)$. Clearly if $T \in \Phi(X,Y)$, then $\iota (T) \in \mathbb{Z}$. 
A Fredholm operator for which $\iota(T)=0$ is called {\sl Weyl}. 
We denote the class of Weyl operators acting between Banach spaces $X$ to $Y$ by $\Omega(X,Y)$ and if $X=Y$ we simply write $\Omega(X)$. It's obvious that $\Omega(X, Y)  \subseteq  \Phi(X, Y).$


\begin{definition}[\cite{TL}, p. 251]
\normalfont
\noindent Let $X$ and $Y$ be Banach spaces. Given $T \in \mathcal{BL}(X,Y)$, we say that an operator 
$S \in \mathcal{BL}(Y,X)$ is a pseudoinverse or generalized inverse of $T$ if and only if $T=TST$. 
The operator $T$ is called relatively regular.

\end{definition}

\begin{comment}
\begin{proposition}[\cite{TL}, p. 254 - Theorem 13.2] 
\normalfont
\noindent Let $X$ and $Y$ be Banach spaces. If $T \in \Phi(X,Y)$ then $T$ has a pseudoinverse 
in $\mathcal{BL}(X,Y)$. \qed
\end{proposition}


\begin{proposition}[\cite{TL}, p. 255 - Theorem 13.3] 
\normalfont
\noindent Let $X$ and $Y$ be Banach spaces.
\begin{enumerate}[label=(\alph*)]
\item If $T \in \Phi(X,Y)$ and $S$ is a pseudoinverse of $T$ then $S \in \Phi(Y,X)$ and \\ $i(S)=-i(T)$.
\item If $T \in \Phi(X,Y)$, then $i(T)=0$ if and only if $T$ has a bijective pseudoinverse.
\end{enumerate}
\qed
\end{proposition}
\end{comment}
\noindent The following result is called Atkinson's Theorem

\begin{theorem} [\cite{BMSW}, p. 3]
\normalfont Let $X$ be a Banach space over $\mathbb{C}$. For $T \in \mathcal{BL}(X)$ the 
following statements are equivalent
\begin{enumerate}[label=(\alph*)]
\item $T \in \Phi(X)$,
\item $T + \mathcal{F}(X) \in \big(\mathcal{BL}(X)/\mathcal{F}(X)\big)^{-1}$,
\item $T + \mathcal{K}(X) \in \big(\mathcal{BL}(X)/\mathcal{K}(X)\big)^{-1}$,
\item $\hat{T} \in \mathcal{BL}(\hat{X})^{-1}$.
\end{enumerate}
\end{theorem}
\vskip 2cm
\begin{center}
\maltese
\end{center}
\chapter{The spectral topology in rings}

\section{Introduction}

In this chapter we present an example of a Kuratowski closure operation, called the 
{\sl spectral closure}, which gives rise to a topology on a ring, called the 
{\sl spectral topology}. We discuss some of the properties of this topology. We look at some 
examples of the topologies on different types of rings. We also look at the product spaces and 
quotient spaces of rings with the spectral topology. 


\subsection{Algebraic closure}
In \cite{CH2} the authors of that article (Harte and Cvetkovic-Ili\'{c}) made a first attempt at defining an operation that  defines a topology on a ring.
\begin{definition}[\cite{CH2}, p. 3547] 
\normalfont
Let $R$ be a ring and $K \subseteq R$. We define the {\sl algebraic closure} of $K$ by:
\begin{center}
$\cl_{\alg}(K) = \{ a \in R:\ $ for all $b \in R$ there exists $a' \in K:\  1-b(a-a') \in R^{-1}\}$
\end{center}
\end{definition}

\begin{proposition}\label{AlgClosureAndRad}
\normalfont
\noindent Let $R$ be a ring and $a \in R$. Then $\cl_{\alg}(\{a\}) = a + \Rad R.$
\end{proposition}
\proof
We show that $\cl_{\alg}(\{a\}) \subseteq a + \Rad R$ and $a + \Rad R \subseteq \cl_{\alg}(\{a\})$.
\vskip 0.3cm
\noindent To see the first inclusion, let $c \in \cl_{\alg}(\{a\})$. Then for every $b \in R$ there exists
$a' \in \{a \}$ such that $1 - b(c - a') \in R^{-1}$. Since $\{a\}$ is a singleton, this is equivalent to 
saying that: For every $b \in R$ we have that $1 - b(c - a) \in R^{-1}$. 
Hence $1 - R(c - a) \subseteq R^{-1}$. 
This means that $c - a \in \Rad R$. So there exists $d \in \Rad R$ s.t. $c - a = d$. 
Hence $c = a + d$, or $c \in a + \Rad R$ as required.
\vskip 0.3cm
\noindent To prove the second inclusion, let $b \in a + \Rad R$. Then there exists $c \in \Rad R$ 
such that $b = a + c$. This means that $c = b - a \in \Rad R$. Hence $1 - R(b - a) \subseteq R^{-1}$. 
Hence, for every $d \in R$, we have that $1 - d(b - a) \in R^{-1}$. 
This means that $b \in \cl_{\alg}(\{a\})$. Hence the second inclusion holds and the proposition is proved.
\qed

\begin{remark}
\normalfont
The algebraic closure has almost all the properties of a topological closure, failing only at 
K3 (see \cite{CH2}, page 3548). 
The definition in the next section is essentially a generalization of the the definition of
algebraic closure. As we shall see it satisfies all the conditions of a closure operator.

\end{remark}

\section{The spectral topology on a ring}

\subsection{The spectral closure}

\begin{definition}[\cite{CH3}, p. 268] 
\normalfont Let $R$ be a ring, $K \subseteq R$. The {\sl spectral closure} of $K$ is:
\vskip 0.3cm
\noindent $\CL(K)=\{ a \in R:  \forall\ \text{finite}\ J \subseteq R\  \exists\ a' \in K:  1-J(a-a') \subseteq R^{-1}\}$ \hfill $(1)$ 

\vskip 0.3cm

\noindent Our first concern is to show the equivalence of $(1)$ with $(2)$:

\vskip 0.3cm

\noindent $\CL(K)=\{ a \in R:  \forall\ \text{finite}\ J,L \subseteq R\ \exists\ a' \in K:  1-J(a-a')L \subseteq R^{-1}\}$ \hfill $(2)$ 

\end{definition}

\noindent For the purpose of the argument, denote the set in $(1)$ by $\CL_1(K)$ 
and the set in $(2)$ by $\CL_2(K)$. We show that $\CL_1(K) \subseteq \CL_2(K)$ and $\CL_2(K) \subseteq \CL_1(K)$. 

\medskip

\noindent For the first inclusion, let $a \in \CL_1(K)$ and let $H,L \subseteq R$ such that 
$H,L$ are finite. 
Then the product set $H \cdot L $ is finite. Since $a \in \CL_1(K)$ there exists $a' \in K$ 
such that $1-H \cdot L (a-a') \subseteq R^{-1}$. This means that $1-h \cdot l (a-a') \in R^{-1}$
 for all $h\in H, l \in L$. By Lemma \ref{Jacobson}, this means that $1- l (a-a')h \in R^{-1}$ 
 for all $h\in H, l \in L$. Hence $1-L(a-a')H \subseteq R^{-1}$, which means that 
 $a \in \CL_2(K)$. Hence $\CL_1(K) \subseteq \CL_2(K).$

\vskip 0.3cm

\noindent For the reverse inclusion, let $a \in \CL_2(K)$. Then for all finite $J,L \subseteq R$
 there exists $a' \in K$ such that $1-J(a-a')L \subseteq R^{-1} $. 
In particular, consider $L=\{1\}$. Then $1-J(a-a')=1-J(a-a')\{1\}\subseteq R^{-1}$. Hence $a \in \CL_1(K)$, so that $\CL_2(K) \subseteq \CL_1(K)$. 
\vskip 0.3cm
\noindent Combining the two inclusions gives $\CL_1(K)=\CL_2(K)$.
\vskip 0.3cm
\noindent If we want to emphasize that the spectral closure of a subset $K$ is taken in a
specific ring $R$, we indicate this explicitly by use of a subscript, as in $\CL_R(K)$, but omit the subscript whenever no confusion is possible.

\subsection{Basic properties of the spectral closure}

\begin{lemma}\label{closureOfSubsets}
\normalfont
Let $R$ be a ring, and let $A \subseteq B \subseteq R$. Then $\CL(A) \subseteq \CL(B).$
\end{lemma}
\proof \space
Suppose $A \subseteq B$ and let $a \in \CL(A).$ Also, suppose that $J \subseteq R, J$ finite. 
Since $a \in \CL(A)$ there exists $a' \in A$ such that $1 - J(a - a') \subseteq R^{-1}.$ 
But $A \subseteq B$, so $a' \in B$. Hence $a \in \CL(B)$, as required. \qed

\begin{proposition} \label{InvertiblesTimesClosure}
\normalfont
\noindent Let $R$ be a ring. Then $\CL(R^{-1}) = R^{-1}\CL(R^{-1})$.
\end{proposition}
\proof \space
To see that $\CL(R^{-1}) \subseteq R^{-1}\CL(R^{-1})$, let $a \in \CL(R^{-1})$. 
Then $a = 1\cdot a \in R^{-1}\CL(R^{-1})$, and hence $\CL(R^{-1}) \subseteq R^{-1}\CL(R^{-1})$.
\vskip 0.3cm
\noindent To see the reverse inclusion, let $a \in R^{-1}\CL(R^{-1})$. 
Then $a = a_1 a_2$, where $a_1 \in R^{-1}$ and $a_2 \in \CL(R^{-1})$. 
Let $J \subseteq R, J$ finite 
and arbitrary. Then $Ja_1$ is also finite, and since $a_2 \in \CL(R^{-1})$ there exists 
$a'_2 \in R^{-1}$ such that $1 - Ja_1(a_2 - a'_2) \in R^{-1}$, hence 
$1 - J(a_1a_2 - a_1a'_2) \in R^{-1}$. This gives
$1 - J(a - a_1a'_2) \in R^{-1}$. By part (a) of Lemma \ref{RInverseStructure1}, we have that 
$a_1a'_2 \in R^{-1}$. Hence $a \in \CL(R^{-1})$, or  $R^{-1}\CL(R^{-1}) \subseteq \CL(R^{-1})$. 
Hence the result follows.
\qed

\begin{proposition}[\cite{CH3}, p. 268] \label{closureOfSingleton}
\normalfont
Let $R$ be a ring. Then 
\begin{center}
$\CL(\{a\})=a+\Rad R\ \text{for all}\ a \in R$.
\end{center}

\end{proposition}

\proof \space 
We show that 
\begin{center}
$ a+\Rad R \subseteq \CL(\{a\}) \ \text{for all}\ a \in R$. \hfill(1)
\end{center}
and 
\begin{center}
$\CL(\{a\}) \subseteq a+\Rad R\ \text{for all}\ a \in R$. \hfill(2)
\end{center}
\noindent To see $(1)$, let $b \in a+\Rad R.$ Then $b = a + c$ for some 
$c \in \Rad R$. Let $J \subseteq R, J$ finite. Then
\begin{align*}
1 - J(b - a) &\subseteq 1 - R(b - a) \qquad \ (\text{since $J \subseteq R)$}\\
&\subseteq R^{-1} \qquad \qquad \qquad (\text{by Proposition \ref{RadCharacter}}).
\end{align*}
This last containment shows that $b \in \CL(\{a\})$ as required.
\vskip 0.3cm
\noindent To prove $(2)$, let $b \in \CL(\{a\}).$ Let $x \in R$. Then $\{x\} \subseteq R, \{x\}$ finite.
Since $b \in \CL(\{a\})$ we must have that
\begin{center}
$1 - \{x \} (b - a) \subseteq R^{-1}$.
\end{center}
\noindent Since this last containment holds for all $x \in R$, 
we must have that $1 - R(b - a) \subseteq R^{-1}$.  By Proposition \ref{RadCharacter} we have that $b - a \in \Rad R$. Hence
 \begin{center}
 $b - a = r$, for some $r \in \Rad R$, or $b \in a + \Rad R$,
 \end{center}
 as required.
\noindent \qed


\begin{proposition}[\cite{CH3}, p. 268] \label{ctyOfAddAndMult}
\normalfont
Let $R$ be a ring and $K,H \subseteq R$. Then the spectral closure is compatible with the ring operations:

\begin{enumerate}[label=(\alph*)]
\item $\CL(K)+\CL(H) \subseteq \CL(K + H)$
\item $\CL(K) \cdot \CL(H) \subseteq \CL(K \cdot H)$
\end{enumerate}

\end{proposition}

\proof 
\begin{enumerate}[label=(\alph*)]
\item Suppose that $x \in \CL(K)$ and $y \in \CL(H)$. We show that $x+y \in \CL(K+H)$.
\vskip 0.3cm
\noindent Let $J \subseteq R, J$ finite. Since $x \in \CL(K)$ there exists $x' \in K$ such that 
$1-J(x-x') \subseteq R^{-1}$. Let $G=1-J(x-x')$. Then $G \subseteq R^{-1}$ and $G$ is finite. 
Then $G^{-1} J$ is also finite, and since $y \in \CL(H)$ there exists 
$y' \in H$ such that $1 - G^{-1} J(y-y') \subseteq R^{-1}$. 
Hence we have, for $j \in J$:
\vskip -.7cm
\begin{align*}
1-j[(x+y)-(x'+y')]
&= 1-j[ x + y - x' - y' ] \\
&= 1-j[ x - x' + y - y' ] \\
&= 1-j[ (x - x') + (y - y') ] \\
&= 1-j(x-x')-j(y-y').
\end{align*}
Hence
\begin{align*}
1-j[(x+y)-(x'+y')]
&\in G-J(y-y')\\
&\subseteq G[1 - G^{-1} J(y-y')] \\
&\subseteq R^{-1} \qquad \text{(using part (a) of Lemma \ref{RInverseStructure1}).}
\end{align*}
\noindent Hence $1-J[(x+y)-(x'+y')] \subseteq R^{-1}$. It is clear that $x'+y' \in K+H $ so that (a) 
is proved.

\item Suppose $x \in \CL(K)$ and $y \in \CL(H)$. We show that $x y \in \CL(K H)$. Let $J \subseteq R, J$ finite. Since $x \in \CL(K)$ there exists $x' \in K$ such that $1-J(x-x')y \subseteq R^{-1}$. Let $D=1-J(x-x')y$. Then $D \subseteq R^{-1}$, and since $D$ is finite, 
we have $D^{-1}$ is also finite. Also $\{x' \} \subseteq R$ and is finite. Hence $D^{-1}  J  x' 
\subseteq R$ and is finite. Since $y \in \CL(H)$ we have that there exists $y' \in H$ such that $1- D^{-1} J  x' (y-y') \subseteq R^{-1}$. Hence for $j \in J$:
\begin{align*}
1-j(xy-x'y')
&= 1-j(xy)+j(x'y')\\
&= 1-j(xy)+j(x'y)-j(x'y)+j(x'y')\\
&= 1-jxy + jx'y - jx'y +jx'y'\\
&= 1-j(xy - x'y) - j(x'y - x'y')\\
&= 1-j(x-x')y - j x'(y-y').
\end{align*}
Hence
\begin{align*}
1-j(xy-x'y')
&\in D-J x'(y-y')\\
&\subseteq D[1-D^{-1} J x'(y-y')]\\ 
&\subseteq R^{-1} \qquad \text{(using part (a) of Lemma \ref{RInverseStructure1}).}
\end{align*}
\noindent Hence $1-J(xy-x'y')	\subseteq R^{-1}$. It is clear that $x'y' \in K H$, and so 
(b) is proved \qed

\end{enumerate}


\begin{lemma}  \label{ngtnIsCts}
\normalfont
Let $R$ be a ring. Then for $V \subseteq R$, we have:

\begin{enumerate}[label=(\alph*)]

\item $a \in \CL(V) \implies -a \in \CL(-V)$,

\item $\CL(-V)=-\CL(V)$.

\end{enumerate}

\end{lemma}


\proof \space To prove (a), suppose that $a \in  \CL(V)$ and let $J$ be an arbitrary finite subset of $R$. 
Let $J' =-J$. Then $J'$ is also finite. Since $a \in \CL(V)$, there exists 
$a' \in V$ such that $1-J'(a-a') \subseteq R^{-1}$. Then $1+J'(-a+a') \subseteq R^{-1}$. Hence 
\begin{equation} \label{eqnNgtnIsCts}
1+J'(-a-(-a')) \subseteq R^{-1}
\end{equation}

\vskip 0.3cm
\noindent Let $b=-a'$. Then $b \in -V$. From \eqref{eqnNgtnIsCts} we have that 
\begin{center}
$ 1-J(-a-b) \subseteq R^{-1}$
\end{center}
Since $J$ was arbitrary and finite, we have that $-a \in \CL(-V)$, as desired.
\vskip 0.3cm
\noindent To prove (b) we show that $\CL(-V) \subseteq - \CL(V)$ and $- \CL(V) \subseteq \CL(-V)$. 
For the first inclusion, suppose that $a \in \CL(-V)$. 
From part (a), we have that $-a \in \CL(-(-V)) = \CL(V)$. Hence $-a \in \CL(V)$, 
so $a \in -\CL(V)$. Hence the first inclusion is proved. For the reverse inclusion, suppose that $a \in - \CL(V)$. Then $-a \in \CL(V)$ so by part (a) we have that $a \in \CL(-V)$, as desired. Hence the second inclusion is proved and $\CL(-V)= - \CL(V)$. \qed

\subsection{The spectral closure on products of rings}

\noindent In the next result we extend the spectral closure to the direct sum of two 
rings.
\begin{proposition} \label{ClosureOfProducts}
\normalfont
Let $R$ and $S$ be rings. and let $E_1 \subseteq R$ and $E_2 \subseteq S$. Then 
\begin{center}
$\CL_{R\oplus S} (E_1 \oplus E_2) = \CL_R(E_1) \oplus \CL_S(E_2)$
\end{center}
\end{proposition}
\proof \space
We prove the statement by showing that
\begin{center}
$(e_1, e_2) \in \CL_{R\oplus S}(E_1 \oplus E_2) \iff (e_1, e_2) \in \CL_R(E_1) \oplus \CL_S(E_2).$
\end{center}
So, $(e_1, e_2) \in \CL_{R\oplus S}(E_1 \oplus E_2) \iff$
\vskip 0.3cm
for finite  $J \oplus K \subseteq R \oplus S$ there exists $(e'_1, e'_2) \in E_1 \oplus E_2$ such that\begin{center}
$(1, 1) - (J \oplus K) \big[(e_1, e_2) - (e'_1, e'_2)\big] \subseteq (R\oplus S)^{-1} = R^{-1} \oplus S^{-1} \iff $
\end{center}
$J \subseteq R, J$ finite and $e'_1 \in E_1$ and $1 - J(e_1 - e'_1) \subseteq R^{-1}$
\vskip 0.3cm
and
\vskip 0.3cm
\noindent $K \subseteq S, K$ finite and $e'_2 \in E_2$ and $1 - K(e_2 - e'_2) \subseteq S^{-1} \iff $
\begin{center}
$(e_1, e_2) \in \CL(E_1) \oplus \CL(E_2)$
\end{center}
\qed

\section{The spectral topology}

\begin{theorem}[\cite{CH3}, p. 268] \label{TheSpectralTopology}
\normalfont
On a ring $R$, the spectral closure generates a topology, called {\sl the spectral topology}, defined as
\begin{center}
$\tau=\{ K \subseteq R: \CL(R \setminus K) = R \setminus K \}$.
\end{center}
\end{theorem}

\proof \space We show that 
$\CL(\cdot): \powerset(R) \rightarrow \powerset(R)$ satisfies the Kuratowski closure axioms, K1 through 
K4 of Definition \ref{defnKuratowski}.

\vskip 0.3cm

\noindent To show that K1 holds, suppose that $K \subseteq R$. Let $J \subseteq R$, $J$ finite and 
let $x \in K$. Then $1-J(x-x)= 1 - \{0 \} = \{1\} \subseteq R^{-1}$. Hence $x \in \CL(K)$. 
So $K \subseteq \CL(K)$, and K1 holds.

\vskip 0.3cm

\noindent To show that K2 holds, notice that from K1 and Lemma \ref{closureOfSubsets} we have 
$\CL(K)\subseteq \CL(\CL(K))$. So it remains to show the reverse inclusion, that 
$\CL(\CL(K)) \subseteq \CL(K)$. 
To do so, let $a \in \CL(\CL(K))$. We show that $a \in \CL(K)$. Let $J \subseteq R, J$ finite. 
Then there exists $a' \in \CL(K)$ such that $1-J(a-a') \subseteq R^{-1}$. Let $G = 1-J(a-a')$. 
Then $G \subseteq R^{-1}$. Since $J$ is finite, $G$ is finite, and since $G \subseteq R^{-1}$, the set 
$G^{-1}$ is well defined. Also, since $G$ is finite, so is $G^{-1}$, and so is $G^{-1}J$.
\vskip 0.3cm 
\noindent Since $a' \in \CL(K)$ and $G^{-1}J$ is finite, there exists $a'' \in K$ such that 
\begin{center}
$1-G^{-1} J(a'-a'') \subseteq R^{-1}$. 
\end{center}
Next we have for $j \in J$:
\begin{align*}
1-j(a-a'') 
&= 1-j(a-a'+a'-a'')\\
&= 1-j(a-a')-j(a'-a'').
\end{align*}
Hence
\begin{align*}
1-j(a-a'') 
&\in G-J(a'-a'')\\
&\subseteq G(1 - G^{-1} J(a'-a'')).
\end{align*}		
			
\noindent Since $G \subseteq R^{-1}$ and $1-G^{-1} \cdot J (a'-a'') \subseteq R^{-1}$, by part (a) of
Lemma \ref{RInverseStructure1}, we have that $G(1-G^{-1} \cdot J(a'-a'')) \subseteq R^{-1}$, 
which means that $1-J(a-a'') \subseteq R^{-1}$. Hence $a \in \CL(K)$, giving the 
reverse inclusion, that $\CL(\CL(K)) \subseteq \CL(K)$, and so K2 holds.

\vskip 0.3cm

\noindent To prove K3 holds, we show 
$\CL(K) \cup \CL(H)\subseteq \CL(K \cup H)$ and $\CL(K \cup H) \subseteq \CL(K) \cup \CL(H)$.
\vskip 0.3cm
\noindent Since $K \subseteq K \cup H$ and $H \subseteq K \cup H$, we have from 
Lemma \ref{closureOfSubsets} that $\CL(K) \subseteq \CL(K \cup H)$ and $\CL(H) \subseteq \CL(K \cup H)$. 
If two sets are both subsets of the same set, then so is the union of the two sets, 
giving $\CL(K) \cup \CL(H) \subseteq \CL(K \cup H)$.

\vskip 0.3cm

\noindent For the reverse inclusion, suppose that  $x \in \CL(K \cup H)\setminus \CL(H)$. 
Then we have that $x \in \CL(K \cup H)$ and $x \not \in \CL(H)$. Since $x \not \in \CL(H)$, there exists 
$L \subseteq R, L$ finite, such that for all $h \in H$ we have 
\begin{equation} \label{xNotinCLH}
1-L(x-h) \not \subseteq R^{-1} 
\end{equation}
Let $J \subseteq R, J$ finite and arbitrary. 
Since $x \in \CL(K \cup H)$, and $J \cup L \subseteq R$ is finite there exists $x' \in K \cup H$ 
with $1-L(x-x') \subseteq  1 - (J \cup L) (x-x') \subseteq R^{-1}$. Since  $x' \in K \cup H$, 
we have that $x' \in K$ or $x' \in H$. 
But $x' \in H$ would contradict \eqref{xNotinCLH}, hence $x' \in K$. 
So we have, for the arbitrary finite $J \subseteq R$ that there exists $x' \in K$ with the property that
\begin{center}
$1 - J(x - x') \subseteq 1 - (J \cup L)(x - x') \subseteq R^{-1}$.
\end{center}  

\noindent So $x \in \CL(K)$, and so $\CL(K \cup H) \subseteq \CL(K) \cup \CL(H)$. Hence the spectral closure operation satisfies property K3.
\vskip 0.3cm

\noindent To show that K4 holds, notice that  it follows vacuously that $\CL(\emptyset)=\emptyset$ 
from definition $\CL(\emptyset)=\{ a \in A:\ \forall\ \text{finite}\ J \subseteq R\ \exists\ a' \in \emptyset:\ 1-J(a-a') \subseteq R^{-1} \}$.

\vskip 0.3cm

\noindent By Theorem \ref{thmKuratowski}, $\CL(\cdot)$ generates a topology on $R$ defined by 
\begin{center}
$\tau=\{K \subseteq R: \CL(R\setminus K)=R\setminus K\}$
\end{center} 
which we will call the {\sl spectral topology} on the ring $R$.

\qed

\begin{theorem} \label{TopologicalRing}
\normalfont
Let $R$ be a ring and $\tau$ be the spectral topology on $R$. Let $R \oplus R$ be endowed with the 
product topology induced by the spectral topology on $R$. Then
\begin{enumerate}[label=(\alph*)]
\item The map $\cdot : R \oplus R \rightarrow R$ is continuous.
\item The map $+ : R \oplus R \rightarrow R$ is continuous.
\item The map $- : R \rightarrow R$, that maps $r \in R$ to its additive inverse, is continuous.
\item $\langle R, \tau \rangle$ is $T_1 \iff R$ is semisimple.
\end{enumerate}
\end{theorem}

\proof 

\begin{enumerate}[label=(\alph*)]
\item To see that multiplication is continuous, we use part (d) of Theorem \ref{charContinuity}. 
So we let $E_1 \subseteq R$ and $E_2 \subseteq R$. We will show that
\begin{equation} \label{multContain}
\cdot \big[\CL_{R \oplus R} \big(E_1 \oplus E_2\big)\big] \subseteq \CL_R\big(\cdot \big[ E_1 \oplus E_2 \big]\big)
\end{equation}
\noindent First, from Proposition \ref{ClosureOfProducts} we have 
$\CL_{R \oplus R}(E_1 \oplus E_2) = \CL_R(E_1) \oplus \CL_R(E_2)$.
\vskip 0.3cm
\noindent Next, we note that
\begin{align*}
\cdot \big[E_1 \oplus E_2\big]
&= \{ \cdot ((e_1, e_2)) : (e_1, e_2) \in E_1 \oplus E_2 \}\\
&= \{ e_1 \cdot e_2 : e_1 \in E_1, e_2 \in E_2 \}\\
&= E_1 \cdot E_2
\end{align*}
\noindent So, to prove \eqref{multContain} is equivalent to proving:
\begin{center}
$\cdot \big[\CL_R(E_1) \oplus \CL_R(E_2)\big] \subseteq \CL_R\big(E_1 \cdot E_2 \big) $
\end{center}
or
\begin{center}
$\CL_R(E_1) \cdot \CL_R(E_2) \subseteq \CL_R\big(E_1 \cdot E_2 \big).$
\end{center}
This is what we proved in part (b) of Proposition \ref{ctyOfAddAndMult}. Hence multiplication 
is continuous.
\item To see that addition is continuous, we use part (d) of Theorem \ref{charContinuity}. 
So we let $E_1 \subseteq R$ and $E_2 \subseteq R$. We will show that
\begin{equation} \label{addContain}
+ \big[\CL_{R \oplus R} \big(E_1 \oplus E_2\big)\big] \subseteq \CL_R\big(+ \big[ E_1 \oplus E_2 \big]\big)
\end{equation}
\noindent First, from Proposition \ref{ClosureOfProducts} we have 
$\CL_{R \oplus R}(E_1 \oplus E_2) = \CL_R(E_1) \oplus \CL_R(E_2)$.
\vskip 0.3cm
\noindent Next, we note that
\begin{align*}
+ \big[E_1 \oplus E_2\big]
&= \{ + ((e_1, e_2)) : (e_1, e_2) \in E_1 \oplus E_2 \}\\
&= \{ e_1 + e_2 : e_1 \in E_1, e_2 \in E_2 \}\\
&= E_1 + E_2.
\end{align*}
\noindent So, to prove \eqref{addContain} is equivalent to proving:
\begin{center}
$+ \big[\CL_R(E_1) \oplus \CL_R(E_2)\big] \subseteq \CL_R\big[E_1 + E_2 \big] $
\end{center}
or
\begin{center}
$\CL_R(E_1) + \CL_R(E_2) \subseteq \CL_R\big(E_1 + E_2 \big).$
\end{center}
This is what we proved in part (a) of Proposition \ref{ctyOfAddAndMult}. Hence addition 
is continuous.
\item Next we show that the inversion map $- : R \to R$ that maps $a \in R$ to $-a$ 
is continuous. 
We use (c) of Theorem \ref{charContinuity}, to prove that for each closed set $V$ in $R$, 
we have that $-^{-1}[V]$ is closed in $R$ with respect to the spectral topology. i.e. 
$\CL_R(-^{-1}[V])=-^{-1}[V]$. Notice that:
\begin{center}
$-^{-1}[V]=\{ a \in R: -a \in V \} = -V$.
\end{center}
\noindent To verify this, consider arbitrary $a \in -^{-1}[V]$. Then $-a \in V$ giving that $a \in -V$ hence $-^{-1}[V] \subseteq -V$. 
For the reverse inclusion, consider arbitrary $a \in -V$ then $-a \in V$ giving $a \in -^{-1}[V]$. 
Thus, the reverse inclusion $-V \subseteq -^{-1}[V]$, holds, giving equality, as desired.
\vskip 0.3cm
\noindent Now, suppose that $V$ is closed in $R$. We must show that $-V$ is closed in $R$. i.e $\CL(V)=V \implies \CL(-V)=-V$.  Suppose that $\CL(V)=V$, then from part (b) of Lemma \ref{ngtnIsCts}, we have that $\CL(-V)=-\CL(V)=-V$, giving that the inversion map is continuous.
\item Suppose that $\langle R, \tau \rangle$ is a $T_1$ space. Then  from part (b) of Proposition 
\ref{SeparationEquiv}, we have that singletons are closed in $R$. 
This gives that $\CL(\{ 0 \}) = \{ 0 \}$. 
By Proposition \ref{closureOfSingleton}, we know for all $a \in R$ that $\CL(\{a\}) = a+\Rad R$. 
Therefore $\Rad R=0+\Rad R=\CL(\{ 0 \}) = \{ 0 \}$. Hence $R$ is semisimple.
\vskip 0.3cm
\noindent Conversely, suppose that $R$ is semisimple. Consider arbitrary, distinct $a,b \in R$. 
Since $R$ is semisimple, we have that $\CL(\{ a \})=a + \Rad R= a + \{ 0 \} = \{ a \}$ and 
$\CL(\{ b \})=b + \Rad R=\{ b \}$. So $\{ a \}$ and $\{ b \}$ are closed sets. 
Therefore $R \setminus \{ b \}$ is an open set containing $a$ and not $b$ and $R \setminus \{ a \}$ is 
an open set containing $b$ and not containing $a$. Hence, $\langle R, \tau \rangle$ is a $T_1$ space.
 \qed
\end{enumerate}


\begin{remark}
\normalfont
Parts (a) to (c) of Theorem \ref{TopologicalRing} means that the spectral topology is a ring topology as defined in Defintion \ref{dfnTopRing}.
\end{remark}

\begin{lemma} \label{MoreThan2}
\normalfont
\noindent 
Let $J$ is a subset of the ring $\mathbb{Z}$ containing more than two nonzero elements. 
Then the following implication holds: 
\begin{center}
$1-Jy \subseteq \mathbb{Z}^{-1} \implies y=0$.
\end{center}
\end{lemma}

\proof \space Suppose that $\{ a_1, a_2, a_3 \} \subseteq J$ with $a_1 \not = a_2 \not = a_3 \not = a_1$ and assume that $a_i$ is nonzero for each $i \in \{1,2,3\}$. The condition 
\begin{equation} \label{InvertsInZ}
1-Jy \subseteq \{ \pm 1 \}
\end{equation}
means that:

\medskip \qquad \qquad \quad \qquad \quad $1-a_1y=1$ \qquad \quad or \quad \qquad $1-a_1y=-1$ \hfill 
(a)

\smallskip \qquad \quad and \qquad \quad $1-a_2y=1$ \qquad \quad or \quad \qquad $1-a_2y=-1$ \hfill 
(b)

\smallskip \qquad \quad and \qquad \quad $1-a_3y=1$ \qquad \quad or \quad \qquad $1-a_3y=-1$ \hfill 
(c)

\vskip 0.3cm

\noindent In determining the set of values for $y$ that satisfy \eqref{InvertsInZ} we have the 
following options:
\vskip 0.3cm \qquad \space $1-a_1y=1$ \qquad and \qquad $1-a_2y=1$ \qquad and \qquad $1-a_3y=1$ \hfill (1)

\vskip 0.3cm or \quad $1-a_1y=-1$ \quad \space and \qquad $1-a_2y=-1$ \quad and \qquad $1-a_3y=-1$ \hfill $(2)$

\vskip 0.3cm or \quad $1-a_1y=-1$ \qquad and \qquad $1-a_2y=1$ \qquad and \qquad $1-a_3y=1$ \hfill $(3)$

\vskip 0.3cm or \quad $1-a_1y=1$ \qquad and \qquad $1-a_2y=-1$ \quad \space and \qquad $1-a_3y=1$ \hfill $(4)$

\vskip 0.3cm or \quad $1-a_1y=1$ \qquad and \qquad $1-a_2y=1$ \qquad and \qquad $1-a_3y=-1$ \hfill $(5)$

\vskip 0.3cm or \quad $1-a_1y=1$ \qquad and \qquad $1-a_2y=-1$ \quad \space and \qquad $1-a_3y=-1$ \hfill $(6)$

\vskip 0.3cm or \quad $1-a_1y=-1$ \quad \space and \qquad $1-a_2y=1$ \qquad 
and \qquad $1-a_3y=-1$ \hfill $(7)$

\vskip 0.3cm or \quad $1-a_1y=-1$ \quad \space and \qquad $1-a_2y=-1$ \quad \space 
and \qquad $1-a_3y=1$ \hfill $(8)$

\vskip 0.3cm
\noindent Consider the set of equations $(1)$. They imply that $-a_i y =0$ which means that $a_i y =0$ 
for each $i \in  \{ 1,2,3 \}$. Since we assumed that $a_i \not = 0$ for each $i \in \{ 1,2,3 \}$ and since 
we know that $\mathbb{Z}$ is an integral domain (Example \ref{ZIsAnIntegralDom}), we conclude that $y=0$.

\vskip 0.3cm 
\noindent Next, consider the set of equations $(2)$. They imply that $a_1 y =2$ and $a_2 y =2$ and 
$a_3 y =2$. 
Our options for values of $y$ are $y=1$, $y=-1$, $y=2$, $y=-2$. Suppose $y = 1$. Then we must have 
that $a_1 = a_2 = a_3 = 2$, contradicting our assumption that the $a_i$ are distinct. Suppose $y = -1$. 
Then we get $a_1 = a_2 = a_3 = -2$, contradicting the same assumption. Each of the other two options for 
a value for $y$ contradicts the same assumption. Hence the set of equations (2) has no solution.
\vskip 0.3cm 
\noindent Next, consider the set $(3)$. They imply that $a_1 y = 2$ and $a_2 y =0$ and $a_3 y =0$.
The first of these equations imply that $y \in  \{ -2,-1,1,2 \}$. The other two imply that $y=0$. 
Clearly there is no solution then. 
\vskip 0.3cm
\noindent The remaining sets of equations ((4) - (8)) all fail in a way similar to equations $(3)$ 
to produce a solution for $y$. The result follows. \qed


\begin{example}[\cite{CH3}, p. 270] \label{SpectralTopOnZ}
\normalfont
\noindent On $\mathbb{Z}$ the spectral closure generates the discrete topology.
\end{example}

\proof \space Suppose $K \subseteq \mathbb{Z}$. First we show that $\CL(K)=K$. 
From K1 of Theorem \ref{TheSpectralTopology} we have that $K \subseteq \CL(K)$. 
To see that $\CL(K) \subseteq K$, suppose that $a \in \CL(K)$. Let $J$ be an arbitrary finite subset of $\mathbb{Z}$. Then there exists $a' \in K$ such that $1-J(a-a') \subseteq \mathbb{Z}^{-1}$. 
Since this expression holds for all finite $J$, it must also hold when $J$ contains more than two nonzero elements. By Lemma \ref{MoreThan2}, this means that $a-a'=0$ which means that $a=a'$, hence $a \in K$, so that $\CL(K)=K$. This shows that every subset of $\mathbb{Z}$ is closed. 
\vskip 0.3cm
\noindent Again, let $K \subseteq \mathbb{Z}$, $K$ arbitrary. Then $\mathbb{Z} \setminus K$ is closed by the above reasoning, so $K$ is open. We have shown that every subset of $\mathbb{Z}$ is open, i.e. the spectral closure generates the discrete topology on $\mathbb{Z}$. \qed

\begin{example}[\cite{CH3}, p. 270] 
\normalfont
\noindent For a Boolean ring $R$, the spectral closure generates the discrete topology.

\end{example}

\proof Suppose $R$ is a Boolean ring. As was noted in Remark \ref{BooleanInvertibles}, $R^{-1}= \{ 1 \}$.
\vskip 0.3cm
\noindent Let $K$ be an arbitrary subset of $R$. Then from K1 of Theorem \ref{TheSpectralTopology} 
we know that $K \subseteq \CL(K)$. It remains to show that $\CL(K) \subseteq K$. 
Suppose that $a \in \CL(K)$. Then there exists $a' \in K$ 
such that $1- \{ 1 \} (a-a') \subseteq \{ 1 \}$, hence
\begin{center}
 $1-(a-a')=1 \implies -(a-a')=0 \implies a-a'=0 \implies a'=a$
\end{center}

\noindent Hence $a \in K$. The rest of the argument is the same as in Example \ref{SpectralTopOnZ}. 
We conclude that the spectral topology on $R$ is discrete. \qed

\begin{proposition}[\cite{CH3}, p. 270] \label{DivRing}
\normalfont
\noindent If $R$ is a division ring, the spectral closure gives the co-finite topology:
\begin{enumerate}[label=(\alph*)]
\item \quad if $K \subseteq R, K$ finite then $\CL(K)=K$,
\item \quad if $K \subseteq R, K$ infinite then $\CL(K)=R$.
\end{enumerate}
\end{proposition}

\proof \space From Definition \ref{dfnDivisionRing}, we can write $R = R^{-1} \cup \{0\}$. 
\vskip 0.3cm
\begin{enumerate}[label=(\alph*)]
\item Assume that $K \subseteq R$, and that $K$ is finite. 
%From K1 of Theorem \ref{TheSpectralTopology}, we have that for any 
%$K \subseteq R$ that $K \subseteq \CL(K)$. It remains to show that $\CL(K) \subseteq K$. 
Either $R$ is finite, or $R$ is infinite. Suppose $R$ is finite. Then either 
$K = R$ or $K \neq R$. 
\vskip 0.3cm
\noindent So first, suppose that $K = R$. By definition of the $\CL(\cdot)$ operation, 
$\CL(R) = R$, so 
$\CL(K) = \CL(R) = R = K$ as required. 
\vskip 0.3cm
\noindent Next, suppose that $K \neq R$. We show that $a \notin K \implies a \notin \CL(K)$, 
as follows. 
Suppose $a \not \in K$. 
This is possible since $K \subseteq R$ and $K \neq R$, hence $R \setminus K \not = \emptyset$. 
We will show that $a \not \in \CL(K)$. So let $a' \in K$, $a'$ arbitrary. Then $a \not = a'$ 
which means $a - a' \not = 0$. 
Since $R$ is a division ring, we have that $a - a' \in R^{-1}$. Consider  $J=\{(a - a')^{-1}\}$, 
a finite subset of $R$. Then the expression $1-J( a - a')$ simplifies to give 
$1-\{(a - a')^{-1}\}(a-a')=1-\{1\}=\{0\} \not \subseteq R^{-1}$. Hence $a \not \in \CL(K)$. 
Since $a$ was arbitrary, we have that $\CL(K) \subseteq K$, giving equality.

\item Next, we assume that $K$ is infinite. 
We always have that $\CL(K) \subseteq R$. It remains to show that $R \subseteq \CL(K)$. 
Let $a \in R$ and suppose that $a \not \in \CL(K)$. Then by definition of the spectral closure 
there exists $J$, a finite subset of $R$ such that for all $a' \in K$, 
we have 
\begin{equation} \label{LInv}
1-J(a-a') \not \subseteq R^{-1}. 
\end{equation}
\noindent This means that for each value of $a - a'$, there exists a value $j \in J$ such that 
\begin{equation}\label{LInv2}
1-j(a-a') \not \in R^{-1}. 
\end{equation}
\noindent Using Remark \ref{JacobsonRemV2}, we have that \eqref{LInv2} is equivalent to 
\begin{equation} \label{RInv}
1-(a-a')j \not \in R^{-1}.
\end{equation}
\noindent Consider \eqref{LInv2}. Since $R$ is a division ring it means that $1-j(a-a') = 0$, 
which means that $j(a-a') = 1$. This means that each $a-a'$ has a left inverse in $J$. 
Similarly, (2.8) implies that $j$ is also a right inverse for $a - a'$ in $J$. Hence $(a - a')$ is
 invertible. $J$ was assumed to be finite, and the set $\{a - a' : a' \in K \}$ is an infinite set 
 (since $K$ is infinite). This means we have a finite list of unique inverses for an infinite set, which is impossible, i.e. it contradicts the fact that inverses are unique (by \eqref{InversesAreUnique}). 
The contradiction stems from our assumption that $a \not \in \CL(K)$. Hence $a \in \CL(K)$, and 
part (b) is proved.
\end{enumerate}
\begin{example} \label{Previous}
\normalfont
\noindent The rings $\mathbb{R}$ and $\mathbb{C}$, under the usual addition and multiplication of 
real numbers and complex numbers respectively, are division rings, and so Proposition
\ref{DivRing} describes what the spectral topology looks like for these rings.
\end{example}

\begin{example} 
\normalfont
In this example we illustrate the fact that $\mathbb{R} (\mathbb{C})$, equipped with the spectral 
topology, is a $T_1$ space but not a $T_2$ space. To do so, we show that $\mathbb{R}$, 
equipped with the spectral topology, is a $T_1$ space but not a $T_2$ space. The argument for 
$\mathbb{C}$, equipped with the spectral topology, is exactly the same. 
\vskip 0.3cm
\noindent To show that $\mathbb{R}$ is a $T_1$ space, we note that $\mathbb{R}$ is a division ring, 
hence as discussed in Example \ref{Previous}, the spectral topology on $\mathbb{R}$ is the 
co-finite topology. Hence any finite subset of $\mathbb{R}$ is s closed set in the spectral topology. 
Now $\{0\}$ is a finite subset of $\mathbb{R}$, hence closed. Hence $\CL(\{0\}) = \{0\}$. 
From Proposition \ref{closureOfSingleton}, we have that $\Rad \mathbb{R} = \CL(\{0\}) = \{0\}.$ Hence
$\mathbb{R}$ is semisimple, and so by Proposition \ref{TopologicalRing}, we know that $\mathbb{R}$, 
endowed with the spectral topology is a $T_1$ space.
\vskip 0.3cm
\noindent Next, we prove that $\mathbb{R}$ equipped with the spectral topology, is not a $T_2$ space. 
To see this, we note that the spectral closure on $\mathbb{R}$ generates the co-finite topology, 
and hence by the second part of Remark \ref{T0T1T2} the topology cannot be $T_2$.  \qed

\end{example}


\subsection{The spectral topology on R / Rad R}
\begin{lemma}[\cite{A} - Theorem 3.1.5, p. 35] \label{InvertsInQuotient}
\normalfont
\noindent Let $R$ be a ring. Then $[x] \in (R/ \Rad R)^{-1}$ if and only if $ x \in R^{-1}$.
\qed
\end{lemma}
\begin{proposition} \label{aInClosureIffEqAInClosure}
\normalfont
\noindent Let $R$ be a ring, $K \subseteq R$. Then
\begin{center}
$a \in \CL_R(K) \iff [a] \in \CL_{R/\Rad R}(\pi(K))$.
\end{center}
\end{proposition}
\proof \space 
\noindent Suppose $a \in \CL_R(K)$. We will show that $[a] \in \CL_{R/\Rad R}(\pi(K))$. 
So let $\tilde{J}$ be any finite subset of $R/\Rad R$. 
By picking one representative from each equivalence class in $\tilde{J}$ we can construct a finite 
$J \subseteq R$ such that $\pi(J) = \tilde{J}$. 
Since $a \in \CL_R(K)$ there exists $a' \in K$ such that 
$1 - J(a - a') \subseteq R^{-1}$. By Lemma \ref{InvertsInQuotient} we have
\begin{center}
$[1] - \pi(J)([a] - [a']) \subseteq (R / \Rad R)^{-1}$
\end{center}
Next $\pi(J) = \tilde{J}$ and $[a'] \in \pi(K)$. Hence
\begin{center}
$[1] - \tilde{J}([a] - [a']) \subseteq (R / \Rad R)^{-1}$,
\end{center}
which means that $[a] \in \CL_{R / \Rad R}(\pi(K))$.
\vskip 0.3cm
\noindent Conversely, let $[a] \in \CL_{R / \Rad R}(\pi(K))$, for $K \subseteq R$. To see that 
$a \in \CL_R(K)$, let $J \subseteq R, J$ finite. Then $\pi(J)$ is a finite subset of $R / \Rad R$.
Since $[a] \in \CL_{R / \Rad R}(\pi(K))$ there exists $b \in R$ such that $[b] \in \pi(K)$ and 
\begin{center}
$[1] - \pi(J)([a] - [b]) \subseteq (R / \Rad R)^{-1}$.
\end{center}
\noindent Since $[b] \in \pi(K)$ we know there exists $a' \in K$ such that $[b] = [a']$. 
Hence we have 
\begin{center}
$[1] - \pi(J)([a] - [a']) \subseteq (R / \Rad R)^{-1}$.
\end{center} 
\noindent By Lemma \ref{InvertsInQuotient} we have $1 - J(a - a') \subseteq R^{-1}$. Since $a' \in K$ 
we have $a \in \CL_R(K)$.
\qed
\begin{proposition}\label{KClosedIffPiKClosed}
\normalfont 
Let $R$ be a ring, $K \subseteq R$. Then 
\begin{center}
$K$ is closed in $R \implies \pi(K)$ is closed in $R / \Rad R.$
\end{center}
\end{proposition}
\proof \space
Suppose that $K$ is closed in $R$. We show that $\pi(K)$ is closed in $R/\Rad R$. From Theorem
\ref{TheSpectralTopology} we know that $\pi(K) \subseteq \CL_{R / \Rad R}(\pi(K))$. 
It remains to show that  $\CL_{R / \Rad R}(\pi(K)) \subseteq \pi(K)$. So let $[a] \in \CL_{R / \Rad R}(\pi(K))$. 
By Proposition \ref{aInClosureIffEqAInClosure} we have that $a \in \CL_R(K).$ Since $K$ is closed this means that $a \in K.$ Hence $[a] \in \pi(K)$ as required. \qed
\vskip 0.3cm
\noindent Let $R$ be a ring. We will denote the spectral topology on $R$ by $\tau_{\CL_R}$. 
\vskip 0.3cm
\begin{proposition} \label{PiIsContinuous}
\normalfont
Let $R$ be a ring. The map
\begin{align*}
\pi : R &\rightarrow R / \Rad R\\
a &\mapsto [a]
\end{align*}
from $\langle R, \tau_{\CL_R} \rangle$ onto $\langle R / \Rad R, \tau_{\CL_{R / \Rad R}} \rangle$ 
is continuous.
\end{proposition}
\proof \space
\noindent We use the characterization of continuity - part (d) of Proposition \ref{charContinuity}. To see that $\pi$ is continuous, let $R$ be  ring and let $E \subseteq R$.
We show that 
\begin{center}
$\pi(\CL_R(E)) \subseteq \CL_{R / \Rad R} (\pi(E))$.
\end{center}
To see that this is the case, let $[a] \in \pi(\CL_R(E)).$ Then there exists $b \in \CL_R(E)$ 
such that $[a] = [b]$. Let $\tilde{J} \subseteq R / \Rad R, \tilde{J}$ finite. Since $\tilde{J}$ 
is finite there exists $J \subseteq R, J$ finite such that $\pi(J) = \tilde{J}$. Since $J$ is finite
and $b \in \CL_R(E)$ there exists $c \in E$ such that $1 - J(b - c) \subseteq R^{-1}.$ 
By Lemma \ref{InvertsInQuotient}, we know that 
\begin{center}
$[1] - \pi(J)([b] - [c]) \subseteq (R/\Rad R)^{-1}$.
\end{center}
Hence $[1] - \tilde J([a] - [c]) \subseteq (R/\Rad R)^{-1}$, and since $[c] \in \pi(E)$ 
we have that \\
$[a] \in \CL_{R / \Rad R} (\pi(E))$, as required. \qed
\begin{proposition}
\normalfont
Let $R$ be a ring. The spectral topology on $R / \Rad R, \tau_{\CL_{R / \Rad R}}$, equals the quotient topology generated by the natural homomorphism, 
\begin{center}
$\pi : R \rightarrow R / \Rad R$.
\end{center}
\end{proposition} 
\proof \space
By Proposition \ref{PiIsContinuous} we know that $\pi$ is continuous. By Proposition \ref{KClosedIffPiKClosed}, $\pi$ is closed. In view of Theorem \ref{QuotientTopology}, 
the result follows.
\qed 
 
 \vskip 0.3cm
\noindent We state the next result without proof.
\begin{proposition}[\cite{CH3}, p. 270] \label{finiteModuloTheRadical}
\normalfont
\noindent Let $R$ be a local ring, and $K \subset R$. Then $K$ is closed in $R$ if and only if 
it is finite modulo the radical.
\end{proposition}
\qed
\vskip 0.3cm
\noindent We can now apply Proposition \ref{finiteModuloTheRadical} to characterize the 
closed sets in 
$\mathbb{C}[[z]]$, as follows.
\begin{theorem}[\cite{CH3}, p. 270]
\normalfont
\noindent Let $R= \mathbb{C}[[z]]$ be the ring of all formal power series over the field $\mathbb{C}$ in
 the indeterminate $z$ and let $K \subseteq R$. Then necessary and sufficient for $K$ to be closed in $R$ is that the set 
\begin{center} 
 $K_0=\{ a_0 \in \mathbb{C}: a_0 + \sum_{n \geq 1} a_n z^n=f(z) \in K \}$ 
\end{center} 
 of all leading coefficients in $K$ be finite.
\end{theorem}
\qed
\vskip 1cm
\begin{center}
\maltese
\end{center}

\chapter{Properties of the spectral topology}

\section{Introduction}

In this chapter, we discuss some interesting properties of the spectral topology. We start by looking
at notions of continuity for the multiplication map defined on a ring with a topology defined on it. Then
we discuss the neighbourhood system around a point of the ring.

\section{Joint and separate continuity}

\noindent The multiplication operation on a ring $R$ is essentially a map 
$\cdot : R \times R \rightarrow R$. This map from the product space can be continuous in more than one 
way. We look at the two notions of continuity that are relevant for us. We start with the most general
definitions of the notions and then make the definitions specific for our needs. 

\begin{definition} [\cite{Namioka}, p. 517] \label{DefJointConty}
\normalfont
A map $f$ from the product $X \times Y$ of topological spaces $X$ and $Y$ into a topological space $Z$ 
is said to be {\sl separately continuous} if, for each $(x_0, y_0) \in X \times Y$, 
the maps $x \mapsto f(x,y_0)$ from $X$ to $Z$ and $y \mapsto f(x_0, y)$ from $Y$ to $Z$ are 
continuous. When $f$ is continuous at $(x_0, y_0)$ relative to the product topology, we shall say that 
$f$ is {\sl jointly continuous} at $(x_0, y_0)$.
\end{definition}

\noindent To make our discussions below easier we will introduce some terms, used by the authors in
 \cite{Henriksen}.
\vskip 0.3cm
\noindent Let $R$ be a ring and let $\cdot : R \times R \rightarrow R$ be the multiplication operation 
on $R$. Fix $r_0 \in R$. Then we define the maps $\cdot_{(r_0, \cdot)}$ and $\cdot_{(\cdot, r_0)}$ from $R$ to $R$ as:
\begin{enumerate}[label=(\alph*)]
\item $\cdot_{(r_0, \cdot)}(r) = r_0 \cdot r$, and
\item $\cdot_{(\cdot, r_0)}(r) = r \cdot r_0$.
\end{enumerate}
\noindent We will call $\cdot_{(r_0, \cdot)}$ the {\sl vertical section of $\cdot$ by $r_0$} and 
$\cdot_{(\cdot, r_0)}(r)$ the {\sl horizontal section of $\cdot$ by $r_0$}.
\vskip 0.3cm
\noindent It is now easy to see that Definition \ref{DefJointConty} above is equivalent to 
\begin{definition}
\normalfont
\noindent Let $R$ be a ring with a topology $\tau$ defined on it. The map 
\begin{center}
$\cdot : R \times R \rightarrow R$ 
\end{center}
is separately continuous w.r.t. $\tau$ if and only if for every $r_0 \in R$, 
the vertical and horizontal sections of $\cdot$ by $r_0$ are continuous w.r.t. $\tau$.
\end{definition}
\noindent Now we are ready to construct the working definition of separate continuity for our
 multiplication operation.
\begin{lemma}
\normalfont
\noindent Suppose $R$ is a ring with a topology $\tau$ defined on it. Suppose that the multiplication
operation $\cdot$ on $R$ is separately continuous w.r.t. $\tau$. Then for each element $a \in R$ and 
for each neighbourhood of zero, $U$, there exists a neighbourhood of zero, $V$, such that 
$a \cdot V \subseteq U$ and $V \cdot a \subseteq U$.
\end{lemma}
\proof \space Let $R, \tau$ and $\cdot$ be as described and let $a \in R, a$ arbitrary. By assumption the 
map $\cdot_{(a, \cdot)}: R \rightarrow R$ is continuous at $0$. 
Next, note that $\cdot_{(a, \cdot)}(0) = a\cdot 0 = 0$. 
So suppose that $U$ is a neighbourhood of $0 = \cdot_{(a, \cdot)}(0)$. By the continuity of 
$\cdot_{(a, \cdot)}$ we have that there exists a neighbourhood of $0$, $V$ say, such that 
$\cdot_{(a, \cdot)}[V] \subseteq U$. But $\cdot_{(a, \cdot)}[V] = \{ a \cdot v : v \in V \} = aV$. 
The second part of the statement is proved using the fact that $\cdot_{(\cdot, a)}$ is continuous at $0$
as well. The result follows. \qed

\section{Comparison of closures}

\begin{proposition}[\cite{CH3}, p. 269] \label{ContainmentOfClosures}
\normalfont
\noindent If $A$ is a Banach algebra with identity $1$ 
and invertible group $A^{-1}$ then for $K \subseteq A$ we have 
$\cl_{\| \cdot \|}(K)  \subseteq  \CL(K)  \subseteq  \cl_{\alg}(K)$.
\end{proposition}

\proof \space Let $A$ be a Banach algebra and let $K \subseteq A$. To prove that $\cl_{\| \cdot \|}(K) \subseteq \CL(K)$ let $a \in \cl_{\| \cdot \|}(K)$. 
Then $a \in K \cup \der_{\|\cdot\|}(K)$. If $a \in K$ then from part (a) of Theorem
 \ref{TheSpectralTopology} we 
have that $a \in \CL(K)$. If $a \in \der_{\|\cdot\|}(K)$ there exists a sequence of points $(b_n)$ 
from $K$ such that $b_n \xlongrightarrow[n]{\infty} a$. 
This means $\| a - b_n \| \xlongrightarrow[n]{\infty} 0$. Let $J$ be a finite subset of $A$. 
Then $J = \{0 \}$ or $J \not = \{0 \}$. If $J = \{0 \}$ then for every $k \in K$ we have that 
$1 - J(a - k) = \{1\} \subseteq A^{-1}$. Hence we have that $a \in \CL(K)$. If $J \not = \{0\}$ then 
$\displaystyle{\max_{j \in J} \| j \| \in \mathbb{R}^+}$. From  
$\| a-b_n \| \xlongrightarrow[n]{\infty} 0$ we have that there exists $N \in \mathbb{N}$ 
such that $n \geqslant N \implies \| a - b_n \| < \frac{1}{\max_{j \in J} \| j \| }$ giving 
$\displaystyle{\max_{j \in J} \| j \| \| a-b_n \| < 1}$. Let $j \in J$. 
Then $\| j(a-b_N) \| \leq \| j \|  \| a-b_N \| \leq \max_{j \in J} \| j \| \| a-b_N \| < 1$. 
Hence by Theorem \ref{oneMinXInvert} we have that $1-j(a-b_N) \in A^{-1}$. Since $j \in J$ was chosen 
arbitrarily, we have that $1 - J(a-b_N) \subseteq A^{-1}$, giving $a \in \CL(K)$. 
Thus $\cl_{\| \cdot \|}(K) \subseteq \CL(K)$ as required.
\vskip 0.3cm
\noindent To show that $\CL(K)  \subseteq  \cl_{\alg}(K)$, let $a \in \CL(K)$. 
Then for all finite  $J \subseteq A$ there exists $a' \in K$ such that 
$1-J(a-a') \subseteq A^{-1}$. Let $b \in A$. Then $\{ b \}$ is a finite subset of $A$ and by 
assumption there exists $a' \in K$ such that $1-b(a-a') \in A^{-1}$ from which the result follows. 
Hence $\CL(K) \subseteq \cl_{\alg}(K)$. \qed

\begin{corollary}
\normalfont
On a Banach algebra the spectral topology is coarser than the norm topology.
\end{corollary}
\proof \space  The proof follows as an application of Lemma \ref{2TopologiesOnOneSet} 
and Proposition \ref{ContainmentOfClosures}. \qed


\begin{theorem}[\cite{CH3}, p. 270] \label{RInverseOpenInSpectralT}
\normalfont
\noindent Let $R$ be a ring. Then $R^{-1}$ is open in the spectral topology.

\end{theorem}

\proof \space We show that $\CL(R \setminus R^{-1}) = R \setminus R^{-1}$. From the fact that the 
spectral closure satisfies K1 from Theorem \ref{TheSpectralTopology} we already have that 
$ R \setminus R^{-1} \subseteq \CL(R \setminus R^{-1})$, so it remains to show that $\CL(R \setminus R^{-1}) \subseteq R \setminus R^{-1}$. We prove this by showing that 
$a \not \in R \setminus R^{-1} \implies a \not \in \CL(R \setminus R^{-1})$. 
So suppose that $a \not \in R \setminus R^{-1}$. Then $a \in R^{-1}$. Let $J = \{ a^{-1} \}$ and pick arbitrary $a' \in R \setminus R^{-1}$. 
Then $1-J(a-a')=1-\{ a^{-1} \}(a-a')=1-(1-a^{-1}a')=a^{-1}a'$. Notice that with  $a^{-1} \in R^{-1}$ and $a' \not \in R^{-1}$, from part (d) of Lemma \ref{RInverseStructure1} we have that $a^{-1}a' \not \in R^{-1}$ so it follows that $a \not \in \CL(R \setminus R^{-1})$. \qed

\noindent The fact that in a ring $R$, the group of invertibles is open in the spectral topology 
enables us to construct neighbourhoods of 0.

\begin{theorem} \label{NbhoodsOf0}
\normalfont 
\noindent Let $R$ be a ring and $J$ a finite subset of $R$. The set
\begin{center}
$U_J = \{ a : 1 - Ja \subseteq R^{-1} \}$
\end{center}
is a neighbourhood of 0.
\end{theorem}
\proof \space
Let $R$ and $J$ be as described. Since $1 \in R^{-1}$ and $R^{-1}$ is open in the spectral topology,
there exists $V \in \mathcal{N}_1$ such that $V \subseteq R^{-1}$. By Theorem \ref{thmTranslationThm}, 
since $V \in \mathcal{N}_1$ we have that $1 - V \in \mathcal{N}_0$. 
By the separate continuity of multiplication, for each $j \in J$ there exists 
$\widetilde{U}_j \in \mathcal{N}_0$ with the property that $j \widetilde{U}_j \subseteq 1 - V.$ 
This means that 
\begin{equation} \label{Ujtilde}
1 - j \widetilde{U}_j \subseteq 1 - (1 - V) = V \subseteq R^{-1}.
\end{equation}
\noindent Let $\widetilde{U}_J = 
\displaystyle \bigcap_{j \in J} \widetilde{U}_j$. Since $\widetilde{U}_J$ is a finite intersection 
of neighbourhoods of 0, it is also a neighbourhood of 0.
\vskip 0.3cm
\noindent We show next that $\widetilde{U}_J \subseteq U_J$. To see this, let $a \in \widetilde{U}_J$.
Then $a \in \widetilde{U}_j$ for each $j \in J$.  By \eqref{Ujtilde} we have that $1 - ja \in R^{-1}$ 
for each $j \in J$. Hence $1 - Ja \subseteq R^{-1}$ and so $a \in U_J$. 
Hence $\widetilde{U}_J \subseteq U_J$ and so $U_J \in \mathcal{N}_0$.
\qed
\begin{theorem}[\cite{CH3}, Theorem 4 - p. 270] 
\normalfont
\noindent Let $R$ be a ring with topology $\rho$ for which the multiplication operation is 
separately continuous. For $K \subseteq R$, let $\cl_{\rho}(K)$ represent the closure of $K$ with 
respect to $\rho$. Then 
\begin{center}
$\cl_{\rho}(K) \subseteq \CL(K)$ for all $K \subseteq R \iff A^{-1} \in \rho$.
\end{center}
\end{theorem}

\proof  \space Suppose that $A^{-1} \in \rho.$ We show that 
$\cl_{\rho}(K) \subseteq \CL(K)$ for all 
$K \subseteq R.$ Let $x \in \cl_{\rho}(K)$. Since $A^{-1}$ is open and $1 \in A^{-1}$ there exists a 
$V \in \mathcal{N}_1$ with the property that $V \subseteq R^{-1}$. By Theorem \ref{thmTranslationThm}, 
we have that $V - 1 \in \mathcal{N}_0$. Let $J \subseteq R, J$ finite. 
Since multiplication is separately continuous, for every $j \in J$, 
there exists $U_j \in \mathcal{N}_0$ such that $j U_j \subseteq V - 1$. 
This gives us $1 + jU_j \subseteq V \subseteq A^{-1}$. Next, let $U_J = \bigcap\limits_{j \in J} U_j$. 
Then $U_J$ is a finite intersection of neighbourhoods of zero, hence $U_J \in \mathcal{N}_0$. By Lemma
\ref{MinVANbhood} $-U_J \in \mathcal{N}_0$ also. Again by Theorem \ref{thmTranslationThm}, 
$x - U_J \in \mathcal{N}_x$. 
Since $x \in \cl_{\rho}(K)$ there exists $x' \in K$ such that $x' \in x - U_J$. 
Hence $x' - x \in -U_J \implies x - x' \in U_J$. So $1 - J(x - x') \subseteq 1 - JU_J \subseteq A^{-1}$. 
Hence $x \in \CL(K)$.
\vskip 0.3cm
\noindent Conversely, suppose that $\cl_{\rho}(K) \subseteq \CL(K)$ for every $K \subseteq R$. 
Then by Lemma \ref{2TopologiesOnOneSet} the spectral topology is weaker than $\rho$. Since $A^{-1}$ is open in the spectral topology, we have that $A^{-1} \in \rho$.
\qed

\begin{comment}
\begin{definition}
\normalfont
\noindent Let $R$ be a ring. We will denote by Finite$(R)$ the set of all finite subsets of $R$. That is
\begin{center}
Finite$(R) = \{J \subseteq R : J$ is finite $\}$
\end{center}
\end{definition}
\end{comment}

\begin{definition}[\cite{CH3}, p. 271]
\normalfont
\noindent
Let $R$ be a ring. An element $x \in R$ is called:

\begin{enumerate}[label=(\alph*)]

\item {\sl Nearly left invertible} if $x \in \CL(R_l^{-1})$.

\item {\sl Nearly right invertible} if $x \in \CL(R_r^{-1})$.

\item {\sl Nearly invertible} if $x \in \CL(R^{-1})$.

\end{enumerate}

\end{definition}

\noindent In the spectral topology we have the following version of the well known property in the 
norm closure (Proposition \ref{IntersectLInvAndRInv}).

\begin{proposition}[\cite{CH3}, p. 271]
\normalfont
\noindent Let $R$ be a ring. Then 
\begin{center}
$R_l^{-1} \cap \CL(R_r^{-1}) = R^{-1} = \CL(R_l^{-1}) \cap R_r^{-1}.$
\end{center}
\end{proposition}

\proof \space 
We show $R_l^{-1} \cap \CL(R_r^{-1}) \subseteq R^{-1}$ and $R^{-1} \subseteq R_l^{-1} \cap \CL(R_r^{-1})$
\vskip 0.3cm
\noindent To see the first inclusion, let $a \in R^{-1}_l \cap \CL(R^{-1}_r)$. Then $a \in R^{-1}_l$ 
and $a \in \CL(R^{-1}_r)$. Since $a \in R^{-1}_l$ we know there exists $a' \in R^{-1}_r$ 
such that $a'a = 1$. Since $a \in \CL(R^{-1}_r)$ we know that there exists $a'' \in R^{-1}_r$ such that 
$1 - a'(a - a'') \in R^{-1}$. Hence $a'a'' \in R^{-1}$. So there exists $b \in R$ with the property that 
$(ba')a'' = b(a'a'') = (a'a'')b = 1$ which means that $a'' \in R^{-1}_l$, 
hence $a'' \in R^{-1}$. Since $a'' \in R^{-1}$ there exists $c \in R$ such that $ca'' = a''c = 1$.
\vskip 0.3cm
\noindent Next, we show that $a' \in R^{-1}$. Since we already have that $a' \in R^{-1}_r$ all we have 
to show is that $a' \in R^{-1}_l$. We know that $ba'a'' = 1$. Hence $ba'a''c = c$, which gives 
$ba' = c$ which gives us $a''ba'=a''c = 1$. Hence $a' \in R^{-1}_l$. So we have that $a'a = 1$ 
and $a''ba' = 1$, so that $a'$ is both left invertible and right invertible. From the paragraph 
preceding (\ref{InversesAreUnique}) we must have that $a''b = a$, so that $a'a = aa' = 1$. 
Hence $a \in R^{-1}$. Hence $R_l^{-1} \cap \CL(R_r^{-1}) \subseteq R^{-1}$.
\vskip 0.3cm
\noindent For the reverse inclusion, we use the fact that $R^{-1}_r \subseteq \CL(R^{-1}_r)$ 
(K1 from Theorem \ref{TheSpectralTopology}) and reason as follows:
\begin{center}
$R^{-1} = R^{-1}_l \cap R^{-1}_r \subseteq R^{-1}_l \cap \CL(R^{-1}_r)$
\end{center}
and the containment follows. Hence we have proved that both containments hold, and so the result follows.
\qed
\vskip 0.3cm
\noindent In words the above says that a nearly invertible element with one sided inverse has to be invertible.
\vskip 0.3cm
\noindent The fact that the collection of sets $\{U_J : J \subseteq R, J$ finite\} are all 
neighbourhoods of zero allows us to define a concept of convergence that applies to a general ring, 
as follows.  
\newpage
\begin{definition} \label{RingConvergence}
\normalfont
\noindent Let $R$ be a ring and let $(x_n)$ be a sequence with $x_n \in R$ for all 
$n \in \mathbb{N}$. Then we say $(x_n)$ converges to 0 if there exists a family $(N_J)$ of 
natural numbers, indexed by finite $J \subseteq R$, for which 
$n \geqslant N_J \implies 1+Jx_n \subseteq R^{-1}$.
\end{definition}
\vskip 2cm
\begin{center}
\maltese
\end{center}

\chapter{Quasinilpotents in general rings}

\section{Introduction}

In this chapter, we introduce concepts generally associated with Banach algebras, suitably extended 
to general rings by means of the spectral closure and we explore some consequences. 


\section{Quasinilpotent elements in a ring}


\begin{proposition}[\cite{H1}, p. 255] 
\normalfont
\noindent If $A$ is a normed algebra, then a necessary and sufficient condition for $a \in A$ 
to be quasinilpotent is the following ($n \in \mathbb{N}$):
\vskip 0.3cm
\noindent for all $m \in \mathbb{N}$ and for any $\{c,d\} \subseteq A^m $ 
the sequence $ c_m^n \cdots c_2^nc_1^na^nd_1^nd_2^n \cdots d_m^n \xlongrightarrow[n]{\infty} 0$.
\qed
\end{proposition}
\vspace{-0.5cm}
\noindent The above proposition relaxes the requirement for the structure $A$ to be a Banach algebra 
and merely stipulates that $A$ should have a topology defined by a norm $\| \cdot \|$.
\vskip 0.3cm
\noindent R. Harte and D. Cvetkovi{\'c-Ili{\'c}} generalize this a step further, by defining 
a concept of quasinilpotence on a general ring equipped with the spectral topology, so we take the following as definition of a quasinilpotent element in a general ring:
\begin{definition}[\cite{CH3}, p. 272]
\normalfont
\noindent Let $R$ be a ring and let $m, n \in \mathbb{N}$. Let $c = (c_1, c_2, ..., c_m) \in R^{m}$. We define
\begin{center}
$c^{(n)} = c^n_1 c^n_2 \cdots c^n_m$ \quad \text{and} \quad  $c_{(n)} = c^n_m \cdots c^n_2c^n_1$
\end{center}
\end{definition}

\begin{definition}[\cite{CH3}, p. 272] \label{DefnQN}
\normalfont
\noindent An element $a$ of a ring $R$ is said to be quasinilpotent if for all 
$m \in \mathbb{N}$ and for any $\{c,d\} \subseteq R^m $ the sequence 
$c_{(n)} a^n d^{(n)} \xlongrightarrow[n]{\infty} 0$. We use $QN(R)$ to represent the set 
of quasinilpotent elements of the ring $R$.

\end{definition}

\noindent It is easy to see that if $R$ is a Banach (more generally, a normed) algebra, then we recover the concept of quisinilpotence via Proposition 4.2.0.1.

\begin{theorem}[\cite{CH3}, p. 272] 
\normalfont
\noindent Let $R$ be a ring, $a, b \in R$. The set $\QN(R)$ satisfies:
\begin{enumerate}[label=(\alph*)]
\item $0 \in \{ a^k: k \in \mathbb{N} \} \implies a \in \QN(R)$,
\item $a \in \QN(R), ab = ba \implies ab \in \QN(R)$,
\item $1-\QN(R) \subseteq R^{-1}$,
\item $\Rad R \subseteq \QN(R)$.
\end{enumerate}
\end{theorem}

\proof \space To prove that (a) holds, suppose $a$ is nilpotent. Then $a^k=0$ for some 
$k \in \mathbb{N}$.
Let $J \subseteq R, J$ finite. Let $N_J = k$ and let $\{c,d\} \subseteq R^m$. Then, if $n \geqslant N_J$ 
we have that $1-J c^n_m \cdots c^n_2c^n_1a^nd^n_1d^n_2 \cdots d^n_m = \{1\} \subseteq R^{-1}$. 
Hence $c^n_m \cdots c^n_2c^n_1a^nd^n_1d^n_2 \cdots d^n_m \xlongrightarrow[n]{\infty} 0$, 
so that $a \in \QN(R)$.
\vskip 0.3cm
\noindent To prove (b), suppose that $ab=ba$ and $a \in \QN(R)$. It is easy to see that if $ab = ba$ 
then $(ab)^n = a^n b^n$. Let $\{c,d\} \subseteq R^m$. Then 
\begin{center}
$c_m^n \cdots c_2^nc_1^n(ab)^nd_1^nd_2^n \cdots d_m^n = c_m^n \cdots c_2^nc_1^n \cdot 1 \cdot a^n b^nd_1^nd_2^n \cdots d_m^n$
\end{center}
\noindent We have that $\{(c_1, \cdots, c_m, 1), (b, d_1, \cdots, d_m)\} \subseteq R^{m+1}$, 
and since $a \in \QN(R)$ we have that 
$c_m^n \cdots c_2^nc_1^n \cdot 1 \cdot a^n b^nd_1^nd_2^n \cdots d_m^n\xlongrightarrow[n]{\infty} 0$, 
hence $c_m^n \cdots c_2^nc_1^n(ab)^nd_1^nd_2^n \cdots d_m^n \xlongrightarrow[n]{\infty} 0$, 
and so $ab \in \QN(R)$.
\vskip 0.3cm
\noindent To prove (c), suppose that $a \in \QN(R)$, $a$ arbitrary. We show that $1-a \in R^{-1}$. 
We define $x_n = a^n$. Since $a \in \QN(R)$ and $(1, \cdots, 1) \in R^n$ 
we have that $a^n \xlongrightarrow[n]{\infty}0$. Let $J = \{1 \}$. By Definition \ref{RingConvergence}, 
we know that for the set $J$ there exists $N_J \in \mathbb{N}$ with the property that 
$n \geqslant N_J \implies 1 - Jx_n \subseteq A^{-1}$. So let $m \in \mathbb{N}, m > N_J$. 
Then $1 - Jx_m = 1 - \{1\}a^m \subseteq A^{-1}$, hence $1 - a^m \in A^{-1}$. Now we write:
\begin{center}
$(1-a)(1+a+a^2+ \dots + a^{m-1})=1-a^m=(1+a+a^2+ \dots + a^{m-1})(1-a)$.
\end{center}
Since $1 - a^m \in R^{-1}$, we know there exists $b \in R$ with the property that
$ (1 - a^m)b = 1 = b(1 - a^m)$. Hence, we have $(1-a)(1+a+a^2+ \dots + a^{m-1}) \cdot b = 1$, 
hence $1 - a \in R^{-1}_r$. Also, we have  $b \cdot (1+a+a^2+ \dots + a^{m-1})(1-a) = 1$, 
hence $1 - a \in R^{-1}_l$. Since $1 - a$ is then left and right invertible, it must be invertible, 
as discussed in the paragraph preceding equation \eqref{InversesAreUnique}.
\vskip 0.3cm
\noindent To prove (d), we show that $\Rad R \subseteq \QN(R)$:

\medskip \noindent To see that $\Rad R \subseteq \QN(R)$ suppose that $a \in \Rad R$. 
We show that $a \in \QN(R)$. So let $m \in \mathbb{N}, m$ arbitrary, and let 
$\{c, d\} \subseteq R^m, c, d$ also arbitrary. We show that 
$c_{(n)} a^n d^{(n)} \xlongrightarrow[n]{\infty} 0$. 
To show this, let $J \subset R, J$ finite and arbitrary. Let $J' = -J$. 
Then $J'$ is also finite. Let $j \in J', j$ also arbitrary. Then, since $a \in \Rad R$, for 
$n \in \mathbb{N} \setminus \{0\}$ we have that $1 - a[a^{n-1}d^{(n)}jc_{(n)}] \in R^{-1}$. 
From Lemma \ref{Jacobson} this means that $1 - jc_{(n)}a[a^{n-1}d^{(n)}] \in R^{-1}$, which means that
$1 - jc_{(n)}a^nd^{(n)} \in R^{-1}$. Hence $1 - J'c_{(n)}a^nd^{(n)} \subseteq R^{-1}$. 
Finally, this means that $1 + Jc_{(n)}a^nd^{(n)} \subseteq R^{-1}$, hence $a \in \QN(R)$.
\qed
\vskip 2cm
\begin{center}
\maltese
\end{center}


\chapter{Idempotents and \newline generalized inverses}

\section{Introduction}

In this chapter we discuss ring homomorphisms in connection with generalized inverses (relatively Fredholm \& relatively Weyl) as well as indempotent elements.


\section{Relatively Fredholm and relatively Weyl elements}


\begin{definition}[\cite{H1}, p. 246 \& \cite{CH3}, p. 272] 
\normalfont
\noindent Let $R$ be a ring. The {\sl relatively Fredholm (regular)} elements of $R$ are the 
elements of $R$ belonging to the set
\begin{center}
$R^{\cap}=\{ a \in R: a\in aRa \}$.
\end{center}
\end{definition}

\begin{definition}[\cite{H1}, p. 246 \& \cite{CH3}, p. 272] 
\normalfont
\noindent Let $R$ be a ring. The {\sl relatively Weyl (decomposably regular)} elements of $R$ are 
those elements belonging to the set
\begin{center}
$R^{\cup}=\{ a \in R: a\in aR^{-1}a \}$.
\end{center}
\end{definition}

\begin{definition}[\cite{H1}, p. 247 \& \cite{CH3}, p. 272] 
\normalfont
\noindent Let $R$ be a ring. The {\sl idempotent} elements $R^{\bullet}$ of $R$ are those elements of $R$ belonging to the set
\begin{center} 
$R^{\bullet}=\{ p \in R: p^2=p \}$.
\end{center}
\end{definition}


\pagebreak

\begin{lemma}[\cite{H1} - Theorem 7.3.4, p. 248] \label{CharRelWeyl}
\normalfont Let $R$ be a ring. Then
$R^{\cup}=R^{-1} R^{\bullet}$.

\end{lemma}

\proof \space We prove that $ R^{-1} R^{\bullet} \subseteq R^{\cup} $ and 
$R^{\cup} \subseteq R^{-1} R^{\bullet}$. To prove the first inclusion let $a \in R^{-1} R^{\bullet}$. 
Then there exists $b \in R^{-1}$ and $p \in R^{\bullet}$ such that $a=bp$. 
Now $a=bp=bpp=bpb^{-1}bp=ab^{-1}a \in R^{\cup}$. 
Hence $ R^{-1} R^{\bullet} \subseteq R^{\cup} $. Next, suppose $a \in R^{\cup}$. 
Then there exists $b \in R^{-1}$ such that $a = aba$. Then $ba = baba$, so that $ba \in R^{\bullet}$.
Then $a = b^{-1}ba \in R^{-1} R^{\bullet}$. Hence $R^{\cup} \subseteq R^{-1} R^{\bullet}$ and the 
proof is complete.\qed

\begin{remark} 
\normalfont
\noindent Let $R$ be a ring. We briefly discuss the following observations relating invertible,
idempotent, regular and decomposably regular elements in $R$:
\begin{enumerate}[label=(\alph*)]
\item $R^{-1} \subseteq R^{\cup}$,
\item $R^{\cup} \subseteq R^{\cap}$,
\item $R^{\bullet} \subseteq R^{\cup}.$
\end{enumerate}
\end{remark}

\proof \space To prove that (a) holds, let $a \in R^{-1}$. Then there exists $b \in R^{-1}$ 
such that $ab=1=ba$. So from $1 = ba$ we get $a = aba$, hence $a \in R^{\cup}$. 
\vskip 0.3cm
\noindent To prove that (b) holds, let $a \in R^{\cup}$. Then there exists $b \in R^{-1}$ such that 
$a = aba$. Since $A^{-1} \subseteq A$, we have that $a \in R^{\cap}$, and the result follows.
\vskip 0.3cm
\noindent To see that (c) holds, let $p \in R^{\bullet}$. 
Then $p=p \cdot 1 \in R^{\bullet} R^{-1}=R^{\cup}$ \qed

\begin{theorem}[\cite{CH3}, p. 272] 
\normalfont
\noindent Let $R$ be a ring. Nearly invertibles with generalized inverses in $R$ have invertible
generalized inverses:
\begin{enumerate}[label=(\alph*)]
\item There is inclusion $R^{\cap} \cap \CL(R^{-1}) \subseteq R^{\cup}$.
\item Necessary and sufficient for equality in (a) is that $R^{\bullet} \subseteq \CL(R^{-1})$.
\end{enumerate}

\end{theorem}

\proof \space Let $a \in R^{\cap} \cap \CL(R^{-1})$. We show that $a \in R^{\cup}$ by using Lemma 5.2.0.4. $a \in R^{\cap} \cap \CL(R^{-1})$ means that $a \in R^{\cap}$ and $a \in \CL(R^{-1})$. Since $a \in R^{\cap}$, there exists $a' \in R$ such that $a=aa'a$. 
Hence we have $a'a=a'aa'a=(a'a)^2$ hence $p=a'a \in R^{\bullet}$. Notice that $ap=aa'a=a$. Now, since $a \in \CL(R^{-1})$, there exists $b \in R^{-1}$ such that $1-(a-b)a' \in R^{-1}$. Let $c^{-1}=1-(a-b)a'$. 
Then $1=cc^{-1}=c+c(b-a)a'$, so left-multiplying by $a$ gives $a=ca+c(b-a)a'a$. But $p=a'a$ so $a=ca+c(b-a)p$. 
Hence $a=ca+cbp-cap$. Since $ap=a$ we have that $a=ca+cbp-ca$, hence $a=cbp$. 
Since $c,b \in R^{-1}$ we have that $cb \in R^{-1}$ (Lemma \ref{RInverseStructure1}) and since $p \in R^{\bullet}$, from Lemma 5.2.0.4, we have that $a \in R^{\cup}$ thus proving (a).
\vskip 0.3cm
\noindent To prove (b), suppose that $R^{\cup} \subseteq R^{\cap} \cap \CL(R^{-1})$ and let $a \in R^{\bullet}$. We show that $a \in \CL(R^{-1})$. 
From part 3 of Remark 5.2.0.5 we have that $R^{\bullet} \subseteq R^{\cup}$, so $a \in R^{\bullet} \implies a \in R^{\cup} \implies a \in \CL(R^{-1})$, by assumption.
\vskip 0.3cm
\noindent Conversely, suppose that $R^{\bullet} \subseteq \CL(R^{-1})$. We show that $R^{\cup} \subseteq R^{\cap} \cap \CL(R^{-1})$. Let $a \in R^{\cup}$. From part $b$ of Remark 5.2.0.5, we have that 
$a \in R^{\cap}$. We show that $a \in \CL(R^{-1})$, which will prove the statement. Since $a \in R^{\cup}$
 there exists $b \in R^{-1}$ such that $a=aba$. This gives $ab=abab=(ab)^2$, hence $ab \in R^{\bullet}$. 
 By assumption $R^{\bullet} \subseteq \CL(R^{-1})$ thus $ab \in \CL(R^{-1})$ which means that for all
  finite $J \subseteq R$, there exists $c \in R^{-1}$ such that $1-J(ab-c) \subseteq R^{-1}$. 
 Let $J=\{ b^{-1} \}$. Then $1-(ab-c)\{b^{-1}\} \subseteq R^{-1}$ equivalently 
 $1-(a-cb^{-1}) \in R^{-1}$. 
Since $c \in R^{-1}$ and $b \in R^{-1}$ we have (by part a, Lemma \ref{RInverseStructure1}) that 
$cb^{-1} \in R^{-1}$. Thus $a \in \CL(R^{-1})$ proving part $b$. \qed


\begin{theorem}[\cite{CH3}, p. 273]
\normalfont
\space Let $A$ be a Banach algebra. Then 
$0 \not \in \inter(\sigma(a)) \implies a \in \cl_{\|\cdot\|}(A^{-1}) \subseteq \CL(A^{-1})$.
\end{theorem}

\proof \space If $0 \notin \inter \sigma(a)$ then 
$0 \in A \setminus \sigma(a)$ or $0 \in \partial \sigma (a).$ 
\vskip 0.3cm
\noindent Suppose $0 \in A \setminus \sigma(a)$. 
Then 
$0 \not \in \sigma(a) \implies -a \in A^{-1} \implies a \in A^{-1} 
\subseteq \cl_{\|\cdot\|}(A^{-1})$.
\vskip 0.3cm
\noindent Suppose $0 \in \partial \sigma(a)$. From the fact that $\sigma(a)$ is compact 
(Theorem \ref{SpectrumCompact}), hence closed and bounded, there exists a sequence 
$(\lambda_n)$ with $\lambda_n \in \mathbb{C}$ for all $n \in \mathbb{N}$ with the properties that 
$\lambda_n \rightarrow 0$ and $a - \lambda_n \in A^{-1}.$ Since $\lambda_n \rightarrow 0$ 
we have that $a - \lambda_n \rightarrow a$. 
Hence every neighbourhood of $a$ (in the norm topology) will contain an element from $A^{-1}$, 
hence $a \in \cl_{\|\cdot\|}(A^{-1})$. The fact that $\cl_{\|\cdot\|}(A^{-1}) \subseteq \CL(A^{-1})$ 
follows from Proposition \ref{ContainmentOfClosures}. The result follows.

\qed
\vskip 2cm
\begin{center}
\maltese
\end{center}
\chapter{Fredholm and Weyl elements}

\section{Introduction}
\noindent Motivated by Atkinson's Theorem, Harte in \cite{H0}, uses the concept of a 
Fredholm operator to define for $a$ an element of a Banach algebra, what it means for $a$ to be {\sl Fredholm  relative to a Banach algebra homomorphism}. 
He then develops a theory called Fredholm Theory relative to a Banach algebra homorphism. In this chapter we illustrate how the spectral closure interfaces with that theory.

\section{T-Fredholm and T-Weyl elements}

\begin{definition}[\cite{H1}, p. 261] 
\normalfont
\noindent Let $R_1$ and $R_2$ be rings and $T: R_1 \to R_2$ be a ring homomorphism. An element $a$ of the 
ring $R_1$ is said to be {\sl T-Fredholm} if $Ta \in R_2^{-1}$.
\end{definition}

\begin{definition}[\cite{H1}, p. 261] 
\normalfont
\noindent Let $R_1$ and $R_2$ be rings and $T: R_1 \to R_2$ be a ring homomorphism. 
An element $a$ of the ring $R_1$ is said to be {\sl T-Weyl} if $a \in R_1^{-1}+T^{-1}(\{0\})$.
\end{definition}


\begin{definition}[\cite{H1}, p. 356] 
\normalfont
\noindent Let $R_1$ and $R_2$ be rings. \\
A homomorphism $T: R_1 \to R_2$ is said to have the 
{\sl Gelfand property} if for all $a \in R_1$, 
\begin{center}
$Ta \in R_2^{-1} \implies a \in R_1^{-1}$.
\end{center}
\end{definition}

\begin{definition}[\cite{CH3}, p. 273] 
\normalfont
\noindent Let $R_1$ and $R_2$ be rings. \\
A homomorphism $T: R_1 \to R_2$ is called {\sl relatively open} 
if and only if 
\begin{center}
$K \subseteq R_1 \implies T(R_1) \cap \CL_{R_2}(T(K)) \subseteq T(\CL_{R_1}(K))$.
\end{center}

\end{definition}


\begin{definition}[\cite{CH3}, p. 273] 
\normalfont
\noindent Let $R_1$ and $R_2$ be rings. A homomorphism $T: R_1 \to R_2$ has {\sl inverse closed range} 
if and only if $T(R_1) \cap R_2^{-1} \subseteq T(R_1)^{-1}$.
\end{definition}

\begin{remark}[\cite{H1} - Theorem 9.6.5, p. 358] 
\normalfont
\noindent Necessary and sufficient for the Gelfand homomorphism $T: R_1 \to R_2$ on $R_1$ to be 
one-to-one, is that $R_1$ is semisimple.
\end{remark}

\begin{theorem}[\cite{CH3}, p. 273]
\normalfont
\noindent Let $R_1, R_2$ be rings and let $T:R_1 \to R_2$ be a ring homomorphism from $R_1$ onto $R_2$. Then we have: 
\begin{equation} \label{FredholmAndWeyl}
 R_1^{-1} \subseteq R_1^{-1} + T^{-1}(\{0\}) \subseteq T^{-1}(R_2^{-1}).
\end{equation}
\end{theorem}


\proof \space Suppose $a \in R_1^{-1}$. Then we can write $a=a+0 \in R_1^{-1}+T^{-1}(\{0\})$. 
Since $a$ was arbitrary we have that $R_1^{-1} \subseteq R_1^{-1} +T^{-1}(\{0\})$, i.e. 
invertible elements are $T$-Weyl.
\vskip 0.3cm
\noindent Next, we show that $T$-Weyl elements are also $T$-Fredholm by showing 
$R_1^{-1}+T^{-1}(\{0\}) \subseteq T^{-1}(R_2^{-1})$. Suppose that $a \in R_1^{-1}+T^{-1}(\{0\})$. 
Then there exists $a_1 \in R_1^{-1}$ and $a_2 \in T^{-1}(0)$ such that $a=a_1+a_2$. 
Since $T$ is a ring homomorphism we have that 
\begin{center}
$Ta = T(a_1+a_2) = Ta_1+Ta_2=Ta_1+0=Ta_1$. 
\end{center}
\noindent Hence $a \in T^{-1}(R_2^{-1}) \iff a_1 \in T^{-1}(R^{-1}_2$). Since $a_1 \in R^{-1}_1$, 
there exists  $a_1^{-1} \in R_1$ with the property that $a_1 a_1^{-1} = a_1^{-1} a_1 = 1$. Since $T$ is onto we have 
(by Proposition \ref{RingHomomOnto}) that 
\begin{center}
$T(a_1)T(a^{-1}_1) = T(a_1a^{-1}_1) = T(1) = 1$ and $T(a^{-1}_1)T(a_1) = T(a^{-1}_1a_1) = T(1) = 1.$
\end{center}
Hence $a_1 \in T^{-1}(R^{-1}_2)$, which means that $a \in T^{-1}(R^{-1}_2)$, and so $a$ is $T$-Fredholm. \qed

\begin{lemma} \label{Ker+Ker}
\normalfont
\noindent Let $R_1, R_2$ be rings. If $T: R_1 \to R_2$ is a homomorphism, then $T^{-1}(0)+T^{-1}(0) \subseteq T^{-1}(0)$.
\end{lemma}

\proof \space Suppose that $a \in T^{-1}(\{0\})+T^{-1}(\{0\})$. Then there exist 
$a_1, a_2 \in T^{-1}(\{0\})$ such that $a=a_1+a_2$. Then
\begin{center}
$Ta = T(a_1 + a_2) = Ta_1 + Ta_2 = 0 + 0 = 0$
\end{center}
Hence $a \in T^{-1}(\{0\})$, and the result follows.
 \qed
\begin{lemma} \label{finiteL}
\normalfont
\noindent Let $R_1$ and $R_2$ be rings and let $T: R_1 \rightarrow R_2$ be a ring homomorphism onto 
$R_2$. Let $L \subseteq R_2, L$ finite. There exists $J \subseteq R_1, J$ finite such that $T(J) = L.$
\end{lemma}
\proof \space
\noindent Let $T, R_1, R_2$ and $L$ be as described, and let $l \in L$. Since $T$ is onto the set 
$T^{-1}(\{l\})$ is not empty. The set $J$ is constructed as follows. For each $l \in L$, pick one 
element from $T^{-1}(\{l\})$ to go into $J$. Then it is clear that $T(J) = L$ and that $J$ is finite.
\qed 
\begin{theorem}[\cite{CH3}, p. 273] \label{TContinuous}
\normalfont
\noindent Let $R_1$ and $R_2$ be rings and $T: R_1 \rightarrow R_2$ a homomorphism 
from $R_1$ onto $R_2$. Then $T$ is continuous with respect to the spectral topology.
\end{theorem}
\proof \space We use again part (d) of Proposition \ref{charContinuity}. So we will show that 
 $K \subseteq R_1 \implies T(\CL_{R_1}(K)) \subseteq \CL_{R_2}(T(K)).$ 
So let $y \in T(\CL_{R_1}(K))$ and let $L \subseteq R_2, L$ finite and arbitrary. Then, by Lemma \ref{finiteL} there exists  $J \subseteq R_1, J$ finite such that $L = T(J)$. 
Since $y \in T(\CL_{R_1}(K))$ there exists 
$x \in \CL_{R_1}(K)$ such that $y = Tx$. Hence there exists $x' \in K$ with the property that
$1_{R_1} - J(x - x') \subseteq R_1^{-1}.$ Next we show that $a \in R_1^{-1} \implies Ta \in R_2^{-1}$. 
To see this, let $a \in R^{-1}_1$. Then 
\begin{center}
$T(a)T(a^{-1}) = T(aa^{-1}) = T(1) = 1$ and $T(a^{-1})T(a) = T(a^{-1}a) = T(1) = 1$
\end{center}
prove the point. Hence 
$1_{R_1} - J(x - x') \subseteq R_1^{-1} \implies T(1_{R_1} - J(x - x')) \subseteq R_2^{-1}$. 
By the linearity of $T$, we have
\begin{center}
$T(1_{R_1} - J(x - x')) = T(1_{R_1}) - T(J)(Tx - Tx')) = 1_{R_2} - L(y - y') \subseteq R_2^{-1}$
\end{center}
\noindent Since $L$ was finite and arbitrary, the last line proves that $y \in \CL_{R_2}(T(K))$ 
as required.
\qed

\begin{theorem}[\cite{CH3}, p. 273] \label{GelfandP}
\normalfont
\noindent Let $R_1, R_2$ be rings and let $T: R_1 \rightarrow R_2$ be a homomorphism. 
If $T$ has the Gelfand property then:
\begin{center}
$K \subseteq R_1 \implies T(R_1) \cap \CL_{R_2}(T(K)) \subseteq T(\CL_{R_1}(K)).$
\end{center} 
\end{theorem}
\proof \space Suppose $K \subseteq R_1$. Let $y \in T(R_1) \cap \CL_{R_2}(T(K)).$ We show 
that $y \in T(\CL_{R_1}(K))$, i.e. we show that there exists $x \in \CL_{R_1}(K)$ with the 
property that $y = Tx$. First, $y \in T(R_1)$ means there exists $x \in R_1$ such that $y = Tx$. 
We will show that $x \in \CL_{R_1}(K)$. To do so, let $J \subset R_1, J$ finite and arbitrary. 
Since $J$ is finite, we have that $L = T(J)$ is also finite and $y \in \CL_{R_2}(T(K))$ means 
that there exists $y' \in T(K)$ such that $1_{R_2} - L(y - y') \subseteq R_2^{-1}.$ 
Since $y' \in T(K)$ this means there exists $x' \in K$ with the property that $y' = T(x')$. 
So we have that 
\begin{center}
$T(1_{R_1}) - T(J)(Tx - Tx') \subseteq R_2^{-1}$ or $T(1_{R_1} - J(x - x')) \subseteq R_2^{-1}.$
\end{center}
Since $T$ has the Gelfand property we have $1_{R_1} - J(x - x') \subseteq R^{-1}_1$
and so $x \in \CL_{R_1}(K)$ as required. \qed

\vskip 0.3cm

\begin{theorem} [\cite{CH3}, p. 273] \label{InverseClosedRange}
\normalfont
\noindent Let $R_1, R_2$ be rings and $T : R_1 \rightarrow R_2$ a ring homomorphism onto $R_2$. 
If $T$ has inverse closed range then:
\begin{center}
$K \subseteq R_1 \implies \CL_{R_1}(K) \cap T^{-1}(R_2^{-1}) \subseteq T^{-1}(\{0\}) + R_1^{-1}K. $
\end{center}
\end{theorem}
\proof \space Assume that $T: R_1 \to R_2$ is a ring homomorphism and suppose 
that $T$ has inverse closed range. Let $K \subseteq R_1$ and let 
$a \in \CL_{R_1}(K) \cap T^{-1}(R_2^{-1})$. We show that $a \in T^{-1}(\{0\})+R_1^{-1}K$.
\vskip 0.3cm
\noindent We have that $a \in \CL_{R_1}(K) \cap T^{-1}(R_2^{-1})$ so $a \in \CL_{R_1}(K)$ and $a \in T^{-1}(R_2^{-1})$. Since $a \in T^{-1}(R_2^{-1})$ we know that $Ta \in R_2^{-1}$. Since $T$ has 
inverse closed range we have that there exists $d \in R_1$ such that $(Ta)^{-1} = Td$.  Since $T$ is onto we have that
\begin{center}
$Ta Td = 1 \implies T(ad) = T(1) \implies T(1 - ad) = 0$
\end{center}
and 
\begin{center}
$Td Ta = 1 \implies T(da) = T(1) \implies T(1 - da) = 0$.
\end{center}
Hence $\{1 - ad, 1 - da \} \subseteq T^{-1}(\{0\}).$
\noindent Since $a \in \CL_{R_1}(K)$ there exists $c \in K$ such that $1-d(a-c) \in R_1^{-1}$, or equivalently, by Lemma \ref{Jacobson}, $1-(a-c)d \in R_1^{-1}$.
\vskip 0.3cm
\noindent Let $e^{-1}=1-(a-c)d$. Then we have
\begin{align*}
a=ee^{-1}a
&=e(1-(a-c)d)a\\
&=e(1-ad+cd)a\\
&=e(1-ad)a+ecda\\
&=e(1-ad)a+ec(da-1+1)\\
&=e(1-ad)a+ec(da-1)+ec\\ 
&\in T^{-1}(\{0\})+T^{-1}(\{0\})+R_1^{-1}K\\ 
&\subseteq T^{-1}(\{0\})+R_1^{-1}K \qquad \text{(from Lemma \ref{Ker+Ker})} 
\end{align*}
\qed

\begin{example}
\normalfont
Let $R$ be a ring and let $J$ be a two sided ideal in $R$. The canonical map 
$\pi_J : R \rightarrow R/J$ is a homomorphism onto. Hence Theorem \ref{TContinuous} says that 
the spectral topology for $R/J$ is weaker than or equal to the quotient of the spectral topology of $R$.
\qed
\end{example}

\smallskip

\begin{example}
\normalfont
\noindent  Let $R_1$ be a subring of a ring $R_2$ and $T : R_1 \rightarrow R_2$ be a homomorphism 
such that $T^{-1}(R_2^{-1}) \subseteq R_1^{-1}$. Then Theorem \ref{GelfandP} tells us that the spectral topology on $R_1$ is weaker than or equal to the restriction of the spectral topology of $R_2$.
\qed
\end{example}


%\chapter{Further properties of homomorphisms}

%\section{Introduction}

%In this section we discuss the relationship between ring homomorphism and nearly invertible Fredholm, %Weyl and weakly Riesz elements. 

\section{Nearly invertible Fredholm, Weyl \newline and weakly Riesz elements}

\begin{definition}
\normalfont
\noindent Let $R_1, R_2$ be rings and $T:R_1 \to R_2$ be a homomorphism. Then $a \in R_1$ is a 
{\sl nearly invertible Fredholm} element if  $a \in \CL(R_1^{-1}) \cap T^{-1}(R_2^{-1})$.
\end{definition}

\begin{definition}[{\cite{DH}, p. 14}] 
\normalfont
\noindent Let $R$ be a ring and $I$ a two sided ideal of $R$. Then $I$ is {\sl weakly Riesz} 
if  $1+I \subseteq \CL(R^{-1})$.

\end{definition}

\begin{lemma} 
\normalfont
\noindent Let $R_1, R_2$ be rings and  $T : R_1 \to R_2$ be a ring homomorphism. Then
\begin{center} 
 $R_1^{-1} + T^{-1}(\{0\}) = R_1^{-1}(1+T^{-1}(\{0\}))$.
\end{center}
\end{lemma}

\proof \space Suppose that $a \in R_1^{-1}+T^{-1}(\{0\})$. Then $a=a_1+a_2$ with $a_1 \in R_1^{-1}$ 
and $a_2 \in T^{-1}(\{0\})$. So $a=a_1(1+a_1^{-1}a_2)$ showing, using Lemma \ref{KerIsAnIdeal}, 
that $a \in R_1^{-1}(1+T^{-1}(\{0\}))$.  Hence 
$R_1^{-1} + T^{-1}(\{0\}) \subseteq R_1^{-1}(1+T^{-1}(\{0\}))$
\vskip 0.3cm
\noindent Conversely, suppose that $a \in R_1^{-1}(1+T^{-1}(\{0\}))$. Then $a=a_1(1+a_2)$ with 
$a_1 \in R_1^{-1}$ and $a_2 \in T^{-1}(\{0\})$. Then $a=a_1+a_1a_2$. Since $a_2 \in T^{-1}(\{0\})$, 
using Lemma \ref{KerIsAnIdeal}, we have that $a_1a_2 \in T^{-1}(\{0\}) $ also.  Hence we have that 
$a \in R_1^{-1}+T^{-1}(\{0\})$. Hence $R_1^{-1}(1+T^{-1}(\{0\})) \subseteq R_1^{-1} + T^{-1}(\{0\})$, 
and the result follows. \qed

\begin{theorem}[\cite{CH3}, p. 274] 
\normalfont
\noindent Let $R_1, R_2$ be rings and let $T:R_1 \to R_2$ be a ring homomorphism. Then
\begin{enumerate}[label=(\alph*)]
\item If $T$ has inverse closed range then
\begin{center}
$\CL(R_1^{-1}) \cap T^{-1}(R_2^{-1}) \subseteq R_1^{-1}+T^{-1}(\{0\})$.
\end{center}
\item  $R_1^{-1}+T^{-1}(\{0\}) \subseteq \CL(R_1^{-1}) \cap T^{-1}(R_2^{-1})$ if and only if $T^{-1}(\{0\})$ is weakly Riesz.

\end{enumerate}
\end{theorem}

\proof \space To prove part (a), we apply Theorem \ref{InverseClosedRange} with $K=R_1^{-1}$ to get
\begin{align*}
R_1^{-1} \subseteq R_1 \implies \CL_{R_1}(R_1^{-1}) \cap T^{-1}(R_2^{-1}) 
&\subseteq T^{-1}(\{0\}) + R_1^{-1}R_1^{-1} \\
&\subseteq T^{-1}(\{0\}) + R_1^{-1}.
\end{align*}
The last inclusion above follows from part (a) of Lemma \ref{RInverseStructure1}.
\vskip 0.3cm

\noindent To prove part (b), suppose that 
\begin{center}
$R_1^{-1}+T^{-1}(\{0\}) \subseteq \CL(R_1^{-1}) \cap T^{-1}(R_2^{-1})$.
\end{center}
Then
\begin{align*}
1+T^{-1}(\{0\}) 
&\subseteq R_1^{-1}+T^{-1}(\{0\})\\
&\subseteq \CL(R_1^{-1}) \cap T^{-1}(R_2^{-1}) \qquad \text{(by assumption)}\\
&\subseteq \CL(R_1^{-1}),
\end{align*}
hence $T^{-1}(\{0\})$ is weakly Riesz, as required.
\vskip 0.3cm
\noindent Conversely, suppose that $T^{-1}(\{0\})$ is weakly Riesz. Then
\begin{align*}
R_1^{-1}+T^{-1}(\{0\})
&= R_1^{-1}(1+T^{-1}(\{0\})) \qquad \text{from lemma 7.2.0.4}\\
&\subseteq R_1^{-1} \CL(R_1^{-1}) \ \qquad \qquad \text{by assumption}\\
&= \CL(R_1^{-1}) \qquad \qquad \qquad \ \text{from Proposition \ref{InvertiblesTimesClosure}.}
\end{align*}

\medskip \noindent From \eqref{FredholmAndWeyl} we have that 
$R_1^{-1}+T^{-1}(\{0\}) \subseteq T^{-1}(R_2^{-1})$. Combined with the above containment, 
$R_1^{-1}+T^{-1}(\{0\}) \subseteq \CL(R_1^{-1})$ 
gives $R_1^{-1}+T^{-1}(\{0\}) \subseteq \CL(R_1^{-1}) \cap T^{-1}(R_2^{-1})$ \qed
\vskip 2cm
\begin{center}
\maltese
\end{center}

\chapter{Closure operations and Bass stable rank} \label{chapter 8}

\section{Introduction}

In this chapter we look at two additional operators. The first of these is just a simple extension 
of the spectral closure operation to n tuples. 
The second is a different idea proposed and studied by a different team of authors - 
Ara, Pedersen and Perera in \cite{APP1, APP2}. Both of these operations appear to be useful in the 
study of the concepts of Bass stable rank of a ring. 

\section{The spectral closure in n dimensions}

We extend the notion of spectral closure to tuples as follows.

\begin{definition}[{\cite{CH3}, p. 275}]
\normalfont
Let $R$ be a ring and $n \in \mathbb{N}$. Let $a, a' \in R^n$. We define $a' \ast a$ as
\begin{center}
$a' \ast a = \displaystyle \sum_{j = 1}^n a'_j  a_j \in R$.
\end{center}
\end{definition}
We use this dot product type operation to define the spectral closure of a subset of $R^n$.
\begin{definition} [{\cite{CH3}, p. 275}]
\normalfont
Let $R$ be a ring and $n \in \mathbb{N}$. Let $K \subseteq R^n$. We define:
\begin{center}
$\CL^{(n)}_{\Left}(K) = \{ x \in R^n : \forall J \subseteq R^n \ \exists \ x' \in K, 1 - J\ast(x - x') \subseteq A^{-1} \}$,
\end{center}
and similarly
\begin{center}
$\CL^{(n)}_{\Right}(K) = \{ x \in R^n : \forall J \subseteq R^n \ \exists \ x' \in K, 1 - (x - x')\ast J \subseteq A^{-1} \}$.
\end{center}
\end{definition} \label{InvertibleTuples}
\noindent We also define left and right invertible tuples as follows.
\begin{definition}[{\cite{CH3}, p. 275}]
\normalfont
Let $R$ be a ring, $n \in \mathbb{N}$. We define the set of {\sl left invertible n-tuples} as
\begin{center}
$R^{-n}_{\Left} = \{ a \in R^n : 1 \in R^n \ast a = \displaystyle \sum_{j = 1}^n R a_j$,
\end{center}
and the set of {\sl right invertible n-tuples} as
\begin{center}
$R^{-n}_{\Right} = \{ a \in R^n : 1 \in a \ast R^n = \displaystyle \sum_{j = 1}^n a_jR$.
\end{center}
\end{definition}
\begin{definition} [{\cite{CH3}, p. 276}] \label{DefnBSR}
\normalfont
Let $R$ be a ring and $n \in \mathbb{N}.$ We say that $R$ has {\sl left (Bass) stable rank 
$\leqslant n$} provided that the following condition holds:
\vskip 0.3cm
\noindent If $(a, b) \in R^n \times R$ and $(a, b) \in R^{-n-1}_{\Left}$ then there exists 
$c \in R^n$ with the property that $a - cb \in R^{-n}_{\Left}$.  
\end{definition}
\begin{remark}
\normalfont
The operations $\CL^{n}_{\Right}$ and $\CL^{n}_{\Left}$ are Kuratowski closure operations - 
see \cite{CH3}, p. 275. We do not include the proof that they are, because the arguments are 
largely a repeat of those in Theorem \ref{TheSpectralTopology}.
\qed
\end{remark}
\noindent Using the closure operation defined in Definition \ref{InvertibleTuples}, 
Harte is able to construct a sufficient condition for a ring to have Bass stable rank $\leqslant n$. 
We state the result without proof.
\begin{theorem} [{\cite{CH3}, p. 276}]
\normalfont
Let $R$ be a ring and suppose that
\begin{center}
$R^n \subseteq \CL_{\Right}^{(n)}(R^{-1}_{\Right})$.
\end{center}
Then $R$ has stable rank $\leqslant n.$
\qed
\end{theorem}
\begin{comment}
\proof \space
\noindent Suppose $(a, b) \in R^n \times R$ and $(a, b) \in R^{-n-1}$. 
We show that there exists $c \in R^n$ such that $a - cb \in A^{-n-1}_{\Left}$. 
Since $(a, b) \in R^{-n-1}_{\Left}$ we have that 
there exists $(a', b') \in R^n \times R$ such that $a' \ast a + b'b = 1$, with 
$a' \in \CL_{\Right}^{(n)}(R^{-n}_{\Right})$. This means that $b'b = 1 - a'a = d - a''\ast a.$
\vskip 0.3cm
\noindent Since $\{a\}$ is a finite subset of $R^n$, we know there exists 
$a'' \in R^{-n}_{\Right}$ such that $1 - (a' - a'')\ast a \in R^{-1}$ 
or $1 - (a' - a'')\ast a = d$ for some $d \in R^{-1}$. 
Then $1 - a' \ast a = d - a'' \ast a = b'b$. Since $a'' \in R^{-n}_{\Right}$ 
there exists $a''' \in R^n$ such that $a'' \ast a''' = 1$. Then
\begin{align*}
d^{-1}a'' \ast (a + a'''b'b) 
&= d^{-1}(a'' \ast a +  a''\ast a'''b'b)\\
&= d^{-1}(a'' \ast a + d - a''\ast a)\\
&= 1
\end{align*}
Hence $a'' \ast (a + a'''b'b) = d \implies a''' \ast a'' \ast (a + a'''b'b) = a'''d$.
\end{comment}
\section{A second operation}
In \cite{APP1, APP2} the authors construct and analyse a different closure operator which is 
of interest to us. Below, we first define the operation and discuss and prove some of its interesting properties. Then we look at how it is connected to the main closure operator we have been discussing so far.
\begin{definition} \label{def1}
\normalfont
Let $R$ be a unital ring, and let $K \subseteq R, a \in R$. \\
Then $a \in \Cl_1(K)$ if and only if the following condition holds:
\begin{center}
If $xa + b = 1$ for some $x, b \in R$ then there exists $y \in R$ such that 
$a + yb \in K$.
\end{center}
\end{definition}
\noindent As in the case of the spectral closure, we have an alternative definition, 
which we give next.
\begin{definition}\label{def2}
\normalfont Let $R$ be a unital ring, and let $K \subseteq R, a \in R$. \\
Then $a \in \Cl_2(K)$
if and only if the following condition holds:
\begin{center}
$Ra + Rb = R \implies (a + Rb) \cap K \neq \emptyset.$
\end{center}
\end{definition}
\noindent Next we show that the two definitions above are in fact equivalent. 
\begin{lemma} \label{Equivalent}
\normalfont
Let $R$ be a ring and $K \subseteq R$. Then $\Cl_1(K) = \Cl_2(K)$.
\end{lemma}
\proof \space
\noindent We show that $\Cl_1(K) \subseteq \Cl_2(K)$ 
and $\Cl_2(K) \subseteq \Cl_1(K)$.
\vskip 0.3cm
\noindent To see the first inclusion, suppose that $a \in \Cl_1(K)$, and that $b \in R$ 
is such that 
\begin{equation} \label{Ra+Rb}
Ra + Rb = R. 
\end{equation}
\noindent We show that $(a + Rb) \cap K \neq \emptyset.$ From \eqref{Ra+Rb} we have that 
there exist $r_1, r_2 \in R$ such that $r_1a + r_2b = 1$. By assumption there exists 
$y \in R$ such that $a + yr_2b \in K.$ Hence $(a + Rb) \cap K \neq \emptyset.$ 
Hence $a \in \Cl_2(K)$, which shows that $\Cl_1(K) \subseteq \Cl_2(K)$.
\vskip 0.3cm
\noindent To see the second containment, let $a \in \Cl_2(K)$. Also, suppose that 
there exists $x, b \in R$ with the property that 
\begin{equation} \label{xa+b}
xa + b = 1. 
\end{equation}
We will show that $Ra + Rb = R.$ Since $R$ is a ring we always have $Ra + Rb \subseteq R$. 
Let $r \in R$. Then from \eqref{xa+b} we have that $rxa + rb = r$, hence $r \in Ra + Rb$. 
This means that $R \subseteq Ra + Rb$. By assumption this means that 
$(a + Rb) \cap K \neq \emptyset.$ So there exists $y \in R$ such that $a + yb \in A$, as required. 
This means that $\Cl_2(K) \subseteq \Cl_1(K)$ and the result follows.
\qed
\begin{remark}
\normalfont
\noindent Lemma \ref{Equivalent} means we can now refer to the closure of a set with a 
single symbol. For $K \subseteq R$, $R$ a ring, we will simply write $\Cl(K)$. 
In what follows we will alternate between the two definitions.
\qed
\end{remark}
\noindent The operation $\Cl(\cdot)$ has some interesting properties, which we list and prove 
in the following proposition. 
\begin{proposition}
\normalfont
Let $R$ be a ring, $A, B \subseteq R$. Then
\begin{enumerate}[label=(\alph*)]
\item $\Cl(\emptyset) = \emptyset$,
\item $A \subseteq B \implies \Cl(A) \subseteq \Cl(B)$,
\item $A \subseteq \Cl(A)$,
\item $\Cl(R) = R$,
\item $\Cl(A) = \Cl(\Cl(A))$,
\item If $A \neq \emptyset$ then $\Rad R \subseteq \Cl(A)$,
\item $\Cl(\{0\}) = \Rad R$.
\end{enumerate}
\end{proposition}
\proof \space
\noindent Property (a) is trivial to see. 
\vskip 0.3cm
\noindent To see that (b) holds, suppose $A \subseteq B$ and that $a \in \Cl(A)$. To see that
$a \in \Cl(B)$, suppose that $b \in R$ is such that $Ra + Rb = R$. Since $a \in \Cl(A)$ we have 
$(a + Rb) \cap A \neq \emptyset.$ Since $A \subseteq B$, 
we also have $(a + Rb) \cap B \neq \emptyset$. Hence $a \in \Cl(B)$, and so 
$\Cl(A) \subseteq \Cl(B)$.
\vskip 0.3cm
\noindent To see that (c) holds, suppose that $A \subseteq R$, and that $a \in A$. 
Suppose that $b \in R$ is such that $Ra + Rb = R$. Notice that 
$a = a + 0b \in (a + Rb) \cap A$. Hence $a \in \Cl(A)$. Hence $A \subseteq \Cl(A)$.
\vskip 0.3cm
\noindent From (c) we have that $R \subset \Cl(R)$. By definition of $\Cl(\cdot)$ we have 
that $\Cl(R) \subseteq R$. Hence (d) holds.
\vskip 0.3cm
\noindent To see that (e) holds, we first notice that from (b) and (c) we have that 
$\Cl(A) \subseteq \Cl(\Cl(A))$. To see the reverse inclusion, let $a \in \Cl(\Cl(A))$, and 
suppose that $xa + b = 1$ for some $x, b \in R$. By definition of $\Cl(\cdot)$ there exists 
$y \in R$ with the property that $a + yb \in \Cl(A).$ Next we have
\begin{align*}
x(a + yb) + (1-xy)b
&= xa + xyb + b -xyb\\
&= xa + b = 1
\end{align*}
Hence $a + yb + z(1 -xy)b \in R$ for some $z \in R$. Hence $a + (y + z - zxy)b \in A$. 
This means that $a \in \Cl(A)$.
\vskip 0.3cm
\noindent To see that (f) holds, suppose that $A \neq \emptyset$, and let $z \in \Rad R$. 
We show that $z \in \Cl(A)$. Suppose that $xz + b = 1$ for some $x, b \in R$. 
Then $xz = 1 - b \in \Rad R$. Hence $b \in R^{-1}$. Then $(a - z)b^{-1} \in R$. Let $a \in R$. 
Then
\begin{center}
$a = z + (a - z)b^{-1}b \in z + Rb$.
\end{center}
Hence $z \in \Cl(A).$ Hence $\Rad R \subseteq \Cl(A)$ and (f) holds.
\vskip 0.3cm
\noindent Finally, to see that (g) holds, notice first that from (f) we have that 
$\Rad R \subset \Cl(\{0\})$. To see that $\Cl(\{0\}) \subseteq \Rad R$, suppose that 
$a \notin \Rad R$. Then there exists a maximal ideal $L$ such that $a \notin L$. 
Then $Ra + L$ is also a left ideal and $L \subseteq Ra + L$. Since $L$ is maximal, we have that 
$Ra + L = R$. So there exists $x \in R$, $l \in L$ such that $xa + l = 1$. 
Suppose that $a \in \Cl(\{0\})$. Then $a + yl = 0$ for some $y \in R$, so that $a = -yl$. 
But this would mean that $a \in L$, contradicting our initial assumption. Hence we have proved that $a \notin \Rad R \implies a \notin \Cl(\{0\})$. This means that $\Cl(\{0\}) \subseteq \Rad R.$ The result follows. \qed
\begin{remark} \label{NotKuratowski}
\normalfont
\noindent Let $R$ be a ring, $E, F \subseteq R$. For non-commutative rings the equation 
$\Cl(E \cup F) = \Cl(E) \cup \Cl(F)$ is not true in general (see Example 1.10 in \cite{APP2}), 
so that $\Cl(\cdot)$ is not a closure operation in the sense of Kuratowski, even though the
conditions $(a), (c)$ and $(e)$ strongly suggest that. 
\qed
\end{remark}

\vskip 2cm
\begin{center}
\maltese
\end{center}


\chapter{Conclusion}
%\addcontentsline{toc}{chapter}{\numberline{}Conclusion}%
\section{Introduction}
\noindent In this final chapter we summarize what was discussed and briefly indicate some 
questions raised by the study. 
\section{Overview}
\noindent Suppose $A$ is a Banach algebra, and $a \in A$. A cornerstone of the theory 
of Banach algebras is the fact that
\begin{center}
$\|a\| < 1 \implies 1 - a \in A^{-1}$.
\end{center}
\noindent This is a deep result in Banach algebra theory since it connects the algebraic 
and analytic foundations of the subject. A similarly deep result in spectral theory is the 
fact that if $a, b \in A$, then
\begin{center}
$1 - ab \in A^{-1} \iff 1 - ba \in A^{-1}$.
\end{center}
\noindent This last fact holds in a general ring as well. Using mainly these two facts 
as motivation, the authors of \cite{CH3}, in the same article, define a set valued mapping that is 
to generate a topology on a ring. 
\vskip 0.3cm
\noindent In Chapter 2 we focused on proving that the set valued mapping is a Kuratowski closure 
operation. We discussed some basic properties of the closure operation. We also looked at how the 
closure of a product of sets is related to the product of the closure of the sets. 
These relationships were needed to prove that the closure operation is in fact a Kuratowski closure
operation, hence generates a topology, called the spectral topology, on the ring. We also proved that a ring with the spectral topology is a topological ring. In this chapter we also looked at some examples
of the spectral topology on different types of rings.
\vskip 0.3cm
\noindent In Chapter 3 we compare the closure of a set in a Banach algebra relative to the spectral
topology with its closure relative to the norm topology. From this we see that the spectral topology
on a Banach algebra is coarser than the norm topology. In this chapter we also showed that the set 
of invertibles in the ring is open in the spectral topology. This fact enabled us to analyse what neighbourhoods of 0 look like in the spectral topology.
\vskip 0.3cm
\noindent In Chapter 4 we look at how the spectral closure and its topology allows us to define a concept 
of quasinilpotent that applies to a general ring.
\vskip 0.3cm
\noindent In Chapter 5 we looked at how the spectral closure intervenes in concepts of generalized
 invertibility.
\vskip 0.3cm
\noindent In Chapter 6 we look at how the spectral closure intervenes with Fredholm Theory relative to 
a Banach algebra homomorphism.
\vskip 0.3cm
\noindent Finally, in Chapter 7, we look at how the spectral closure intervenes in the concepts of 
Bass Stable Rank of a ring. In this chapter we discuss a variant of the main closure operator,
specifically defined on n tuples of elements from a ring. We also discussed an alternative operation
studied by a different team of researchers. 
\section{Open Quesions}
We briefly list some questions that are either described in \cite{CH3} as being unknown, or seem to us 
to be unknown facts, and could potentially lead to research questions or simply a deeper understanding 
of the spectral topology.
\vskip 0.3cm
\noindent In Theorem \ref{NbhoodsOf0} we proved that if $R$ is a ring and $J$ is a finite subset of $R$
then the set
\begin{center}
$U_J = \{a \in R : 1 - Ja \in R^{-1} \}$
\end{center}
is a neighbourhood of 0. We notice that each such set is a superset of $\Rad R$, 
ie $\Rad R \subset U_J$ for each $J$ a finite subset of $R$. This raises the question as to whether 
there are rings $R$ for which $\Rad R$ is an open set in the spectral topology. 
If such rings do exist, what would be the consequences of this fact? For example, if the ring was also 
semisimple, then the spectral topology would necessarily be discrete.
\vskip 0.3cm
\noindent In Theorem \ref{TopologicalRing}, we see that the spectral topology on a ring $R$ is $T_1$ 
if and only if the ring is semisimple. From Example \ref{SpectralTopOnZ} we have that on $\mathbb{Z}$
the spectral topology is discrete, hence $T_2$. This raises the question as to whether the following 
implication always holds. If $R$ is a ring, let $\tau$ be the spectral topology. Is it the case that:
\begin{center}
$\langle R, \tau \rangle$ is $T_2 \implies \langle R, \tau \rangle$ is discrete?
\end{center}
\vskip 0.3cm
\noindent In Definition \ref{DefnQN}, we defined a general notion of quasinilpotent, one that can be 
applied to any ring with the spectral topology. In \cite{CH3}, the authors mention that it is not known whether the following implication holds, for $R$ a ring:
\begin{center}
$a, b \in \QN(R), ab = ba \implies a + b \in \QN(R).$
\end{center}
\vskip 0.3cm
As is mentioned in section 7.3 above, the operation defined by Ara, Pedersen and Perera in 
Definitions \ref{def1} and \ref{def2}, do not generate a topology on a ring. In \cite{CH3}, p. 277,
the authors make the following suggested modification to the concept of Ara, Pedersen and Perera:
\vskip 0.3cm
\noindent The concept by Ara, Pedersen and Perera is essentially:
\vskip 0.3cm
\noindent Let $R$ be a ring and $K \subseteq R$. Then $a \in \cl^{\bullet}_{\Left}(K)$ if and only if 
for arbitrary $b \in R$ 
there is implication:
\begin{center}
$(a, b) \in R^{-2}_{\Left} \subseteq R^2 \implies (a - Rb) \cap K \neq \emptyset$.
\end{center} 
\noindent As mentioned in Remark \ref{NotKuratowski}, the operation does not satisfy all of the
Kuratowski closure conditions. Consider the following definition of an operator:
\newpage
\begin{definition} \label{NewOperator}
\normalfont
\noindent Let $R$ be a ring, $K \subseteq R^n$. Then $a \in \Cl^{\bullet}_{\Left}(K) \subseteq R^n$ 
if and only if
for every finite $J \subseteq R$
\begin{center}
 $(a, J) \subseteq R^{-n-1}_{\Left} \implies \displaystyle \bigcap_{b \in J} (a - R^nb) \cap K \neq \emptyset.$
\end{center}
\end{definition}
\noindent Harte et al in \cite{CH3} say that Definition \ref{NewOperator} makes a `cosmetic' change to the concept in Definition \ref{DefnBSR} which may or may not actually alter it. The first question here 
is whether the change does alter the definition. Harte et al also mention that they believe that 
the modified operator will satisfy all the properties of a Kuratowski closure operator. 
The second question is whether the new operator does satisfy the conditions and of course, how to prove that it does, or does not.

\vskip 2cm
\begin{center}
\maltese
\end{center}

\chapter*{List of Symbols}
\addcontentsline{toc}{chapter}{\numberline{}List of Symbols}%




\begin{changemargin}{-2cm}{-2cm} 

\begin{minipage}[t]{0.6\textwidth}

$\emptyset$ \hfill empty set\

\medskip

$\powerset(A)$ \hfill power set of $A$\

\medskip

$\mathbb{N}$ \hfill set of natural numbers\

\medskip

$\mathbb{Z}$ \hfill set of integers\

\medskip

$\mathbb{Q}$ \hfill set of rationals\

\medskip

$\mathbb{R}$ \hfill real number field\

\medskip

$\mathbb{C}$ \hfill complex number field\

\medskip

$\mathbb{R^+}$ \hfill positive real numbers\

\medskip

$[n]$, $I$ \hfill (finite) index set

\medskip

$G$, $R$, $A$ \hfill group/ring/algebra\

\medskip

$\prod_{i \in I} R_i$ \hfill direct sum (product) of rings\

\medskip

$R^{-1}$, ($R_l^{-1}$/$R_r^{-1}$) \hfill set of (left/right) invertibles\

\medskip

$\mathcal{M}_l$, $\mathcal{M}_r$ \hfill set of all left/right maximal ideals\

\medskip

$R/J$ \hfill quotient ring modulo ideal $J$\

\medskip

$\Rad R$ \hfill Jacobson radical of $R$\

\medskip

$(z)$ \hfill principal ideal generated by $z$\

\medskip

$R[[z]]$ \hfill formla power series ring\

\medskip

$K_0$ \hfill set of leading coefficients of polynomials\

\medskip

$a^{-1}$, $a$ \hfill (invertible) element\

\medskip

$[a]$ \hfill equivalence class of $a$\

\medskip


\end{minipage}
\begin{minipage}[t]{0.05\textwidth}

${}$\\

\end{minipage}
\begin{minipage}[t]{0.6\textwidth}

$\not =$, $=$ \hfill (non-)equality relation\

\medskip

$\not \in$, $\in$ \hfill (non-)membership relation\

\medskip

$<$, $>$ \hfill less/greater-than relation\

\medskip

$\leq$, $\geq$ \hfill less/greater-than-or-equal-to relation\

\medskip

$\subset$, $\subseteq$ \hfill subset relation\

\medskip

$\supset$, $\supseteq$ \hfill superset relation\ 

\medskip

$\sim$ \hfill equivalent/equinumerous\

\medskip

$\#$ \hfill cardinality\

\medskip

$\cap$ \hfill intersection\

\medskip

$\cup$ \hfill union\

\medskip

$\forall$ \hfill universal quantifier\

\medskip 

$\exists$ \hfill existential quantifier\

\medskip

$\implies$ \hfill  (one-way) implication\

\medskip

$\iff$ \hfill bi-implication (if and only if)\

\medskip

$\xLongrightarrow[\textup{\mbox{?`}}]{?}$ \hfill unverified implication\

\medskip

$\infty$ \hfill absolutus infinitus\

\medskip

$\epsilon$, $\delta$ \hfill infinitesimally small numbers\

\medskip

$0$ \hfill additive identity\

\medskip

$1$ \hfill multiplicative identity\

\medskip

$\lambda$, $\alpha$, $\beta$ \hfill scalars / eigenvalues\


\end{minipage}

\end{changemargin}


\pagebreak

\begin{changemargin}{-2cm}{-2cm}

\noindent
\begin{minipage}[t]{0.6\textwidth}

$G+H$ \hfill sum set\

\medskip

$G \cdot H$ \hfill product set\

\medskip

$K^{-1}$ \hfill inverses of elements in $K$\

\medskip

$\tau$, $\sigma$ \hfill topology\

\medskip

$\tau^{\times}$ \hfill product topology\

\medskip

$\langle X, \tau \rangle$ \hfill topological space\

\medskip

$\langle X, \tau^{\times} \rangle$ \hfill product (topological) space\

\medskip

$\langle X, d \rangle$ \hfill metric space\

\medskip

$\langle X, \| \cdot \| \rangle$ \hfill metric space\

\medskip

$X \setminus A$ \hfill complement of a set\

\medskip

$\mathcal{N}_x$ \hfill neighbourhood system at $x$\

\medskip

$\mathscr{D}$, $\mathscr{B}$ \hfill (sub)base for a given topology\

\medskip

$B(x_0, \epsilon)$ \hfill $\epsilon$-neighbourhood about $x_0$\

\medskip

$\overline{A}$ \hfill (familiar) closure of a set\

\medskip

$\partial A$ \hfill boundary of a set\

\medskip

$\sigma(a)$ \hfill spectrum of Banach algebra element $a$\

\medskip

$\rho(a)$ \hfill Gelfand's spectral radius\

\medskip

$T$ \hfill linear operator\

\medskip

$T|_A$ \hfill restriction of a linear operator\

\medskip

$\tilde{T}$ \hfill extension of a linear operator\

\medskip

$T^{\times}$ \hfill adjoint of a linear operator\

\medskip

$\mathscr{D}(T)$, $\mathscr{R}(T)$ \hfill domain/range of a linear operator\

\medskip

$\mathcal{G}(T)$ \hfill graph of a linear operator\

\medskip

$\frac{\partial}{\partial x}$, $\frac{d}{dx}$ \hfill (partial) derivative operator w.r.t. $x$\

\medskip

$\nabla$ \hfill nabla/del operator\

\medskip

$\int$ \hfill (Lebesgue) integral operator\

\medskip

$V$, $V^{\alpha}$ \hfill Voltera/Riemann-Liouville operator\





\end{minipage}
\begin{minipage}[t]{0.05\textwidth}

${}$\\

\end{minipage}
\begin{minipage}[t]{0.6\textwidth}

$\mapsto$ \hfill function definition\

\medskip

$f^{-1}(\cdot)$, $f(\cdot)$ \hfill (inverse) function/functional\

\medskip

$f(\cdot,\cdot)$ \hfill multivariable (two-argument) function\

\medskip

$\circ$ \hfill infix function/operator composition\

\medskip

$+$ \hfill infix binary addition operation\

\medskip

$\cdot$ \hfill infix binary multiplication operation\

\medskip

$*$ \hfill general infix binary operation\

\medskip

$\alpha(\cdot,\cdot)$ \hfill prefix binary addition operation\

\medskip

$\mu(\cdot, \cdot)$ \hfill prefix binay multiplication operation\

\medskip

$\beta(\cdot, \cdot)$ \hfill general prefix binary relation\

\medskip

$p_j(\cdot)$ \hfill $j^{\textup{th}}$ projection map\

\medskip

$I(\cdot)$ \hfill identity map\

\medskip

\textbf{0}$(\cdot)$ \hfill zero map\

\medskip

$\Gamma(\cdot)$ \hfill gamma function\

\medskip

$d(\cdot, \cdot)$ \hfill metric\

\medskip

$\| \cdot \|$ \hfill norm\

\medskip

$\dist(\cdot,K)$ \hfill distance from a point to a set $K$\

\medskip

$\inter(\cdot)$ \hfill interior operation\

\medskip

$\der(\cdot)$ \hfill derived set operation\

\medskip

$\cl(\cdot)$ \hfill topological closure operation\

\medskip

$\cl_{\tau}(\cdot)$ \hfill closure operation w.r.t. $\tau$\

\medskip 

$\cl_X(\cdot)$ \hfill closure operation w.r.t. $X$\

\medskip

$\cl_{\textup{alg}}(\cdot)$ \hfill algebraic closure operation\

\medskip

$\cl_{\| \cdot \|}(\cdot)$ \hfill norm closure operation\

\medskip

$\CL(\cdot)$ \hfill spectral closure operation\

\medskip

$\CL_{\tau}(\cdot)$ \hfill spectral closure operation w.r.t. $\tau$\

\medskip

$\CL_A(\cdot)$ \hfill spectral closure operation w.r.t. $A$\


\end{minipage}


\end{changemargin}


\pagebreak


\begin{changemargin}{-2cm}{-2cm}

\noindent
\begin{minipage}[t]{0.6\textwidth}


$\CL(A^{-1})$ \hfill set of nearly invertibles in $A$\

\medskip

$\QN_{\| \cdot \|}(A)$ \hfill norm quasinilpotent elements of $A$\

\medskip

$\QN(A)$ \hfill ring quasinilpotent elements of $A$\


\end{minipage}
\begin{minipage}[t]{0.05\textwidth}

${}$\\

\end{minipage}
\begin{minipage}[t]{0.6\textwidth}


$\bsr(R)$ \hfill bass stable rank of $R$\

\medskip

$(x_n)$ \hfill sequence\

\medskip

$\displaystyle{\lim_{n \to \infty}}$, $\xlongrightarrow[\infty]{n}$ \hfill  limit (shorthand)\

\medskip

$\inf$, $\sup$ \hfill infimum/supremum of a set\

\medskip

$\max$ \hfill maximum value\\

\end{minipage}


\bigskip

$\dim(V)$ \hfill dimension of a vector space\

\medskip

$T(\cdot)$, $\phi(\cdot)$ \hfill ring/vectorspace homo/epi/isomorphism\

\medskip

$T^{-1}(0)$, $\ker(\phi)$ \hfill  kernel/null space of a vector space homomorphism\

\medskip

$\coker(\phi)$ \hfill cokernel of a vector space homomorphism\

\medskip

$\alpha(T)$, $\alpha(\phi)$ \hfill kernel index of a vector space homomorphism\

\medskip

$\beta(T)$, $\beta(\phi)$ \hfill deficiency index of a vector space homomorphism\

\medskip

$\iota(T)$, $\iota(\phi)$ \hfill index of a vector space homomorphism\

\medskip

$\mathcal{BL}(X,Y)$ \hfill set of bounded linear operators from $X$ to $Y$\

\medskip

$\Phi(X,Y)$ \hfill set of Fredholm elements from $X$ to $Y$\

\medskip

$\Omega(X,Y)$ \hfill set of Weyl operators from $X$ to $Y$\

\medskip

$R^{\bullet}$ \hfill set of indempotent elements in $R$\

\medskip

$R^{\cap}$ \hfill regular (relatively Fredholm) elements in $R$\

\medskip

$R^{\cup}$ \hfill decomposably regular (relatively Weyl) elements in $R$\

\medskip



\end{changemargin}

\begin{thebibliography}{99}

\bibitem{APP1} P. Ara, G. Pedersen, F. Perera \textit{An infinite analogue of rings with 
stable rank one}, J. Algebra 230 (2000), 608 - 655.

\bibitem{APP2} P. Ara, G. Pedersen, F. Perera \textit{A closure operation in rings}, Int. J. 
Mat. 12 (2001), 791 - 812.


\bibitem{A} B. Aupetit, \textit{A Primer on spectral theory}, Springer Science \& Business 
Media, 2012.

\bibitem{BMSW} B. Barnes, G. Murphy, M. Smyth and T. West, \textit{Riesz and Fredholm theory in 
Banach algebras}, Pitman Publishing Company, 1982. 

\bibitem{B} H. Bass, \textit{K - Theory and stable algebra}, Publ. IHES 22 (1964), 5 - 60.

\bibitem{ALC} \href{https://gallica.bnf.fr/ark:/12148/bpt6k90201q/f177}{A.-L. Cauchy, \textit{Sur l'équation à l'aide de laquelle on détermine les inégalités séculaires des mouvements des planétes}, Exercises de mathématiques 4, in Œuvres complètes d'Augustin Cauchy, Paris: Gauthier-Villars et fils, 2 No. 9, 174-95}

\bibitem{CH1} D. Cvetkovic-Ili\'{c}, and R. E. Harte, \textit{On Jacobson’s Lemma and Drazin Invertibility}, Applied mathematics letters 23 (4) (2010), 417-420.

\bibitem{CH2} D. Cvetkovic-Ili\'{c}, and R. E. Harte, \textit{On the Algebraic Closure in Rings}, Proceedings of the American Mathematical Society, 135 (11) (2007), 3547-3552.

\bibitem{CH3} D. Cvetkovic-Ili\'{c}, and R. E. Harte, \textit{The Spectral Topology in Rings}, Studia Mathematica 200 (3) (2010), 267-278.

\bibitem{D} H. A. M. Dzinotyiweyi, \textit{The analogue of the group algebra for topological semigroups}, pitman (1984).

\bibitem{DH} D. S. Djordjevi\'{c}, R. E. Harte, C. M. Stack, \textit{On left-right consistency in Rings}, Math. Proc. Roy. Irish Acad. 106A (2006), 11 - 17.

\bibitem{E} K. J. Engel \& R. Nagel, \textit{A short course on operator semigroups}, Springer 256s (2006).

\bibitem{F} J. B. Fraleigh, \textit{A First Course in Abstract Algebra}, Addison-Wesley Publishing Company Inc. 1989.

\bibitem{G} J. A. Gallian, \textit{Contemporary Abstract Algebra}, 6th Edition, Houghton Mifflin Company, 2006.

\bibitem{G1} L. J. Goldstein, \textit{Abstract Algebra - a first course}, Prentice-Hall, New Jersey, 1973.

\bibitem{G2} S. Goldberg, \textit{Unbounded Linear Operators - Theory and Applications}, McGraw-Hill Book Company. 1966.

\bibitem{H0} R. E. Harte, \textit{Fredholm theory relative to a Banach algebra homomorphism},
 Mathematische Zeitschrift 179, (1982), 431 - 436.

\bibitem{H2} R. E. Harte, \textit{Regular Boundary Elements}, Proceedings of the American Mathematical
 Society 99 (2) (1987), 328 - 330.

\bibitem{H3} R. E. Harte, \textit{Arens-Royden and the Spectral Landscape}, Filomat (Niš) 16 (2002), 31-42.

\bibitem{H1} R. E. Harte, \textit{Invertibility and Singularity for Bounded Linear Operators}, Courier Dover Publications, 2016.



\bibitem{Henriksen} M. Henriksen, R. G. Woods, \textit{Separate versus Joint Continuity}, Topology and its Applications, 97 (1999), 175 - 205.

\bibitem{DH1} \href{http://www.digizeitschriften.de/dms/img/?PID=GDZPPN002499967}{D. Hilbert, \textit{Grundzüge einer allgemeinen Theorie der linearen Integralgleichungen}, Nachrichten von d. Königl. Ges. d. Wissensch. zu Göttingen (Math.-physik. Kl.), (1904), 49-91}.

\bibitem{JMW} P. W. Jones, D. Marshall and T. Wolff, \textit{Stable rank of the disc algebra}, Proceedings of the American Mathematical Society, 96 (4) (1986), 603-604.

\bibitem{K} E. Kreyszig, \textit{Introductory Functional Analysis with Applications}, Vol. 1. Wiley \& Sons. Inc. New York, 1978.

\bibitem{KK} \href{http://matwbn.icm.edu.pl/ksiazki/fm/fm3/fm3121.pdf}{K. Kuratowski,  \textit{Sur l'opération Ā de l'Analysis Situs}, Fundamenta Mathematicae (3) (1922), 182–199}.

\bibitem{L1} T. Y. Lam, \textit{A First Course in Noncummutative Rings - Second Edition}, Springer, New York, 2001.

\bibitem{L2} S. Lipschutz, \textit{Theory and Problems of General Topology}, Schaum's Outline Series (1965).

\bibitem{Muller} V. M\"{u}ller, \textit{Spectral Theory of Linear Operators and Spectral Systems in 
Banach Algebras}, Operator Theory Advances and Applications, Vol. 139, Birkh\"{a}user, (2003).

\bibitem{Namioka} I. Namioka, \textit{Separate continuity and joint continuity}, 
Pacific Journal of Mathematics, Vol. 51, No. 2, 1974.

\bibitem{P} Z. Piotrowski, \textit{The Genesis of Separate versus Joint Continuity}, Tatra Mountains Mathematical Publications, (8) (1996), 113-126.

\bibitem{R} W. Rudin, \textit{Functional Analysis}, McGraw-Hill, New York, 1973.

\bibitem{R1} J. Rotman, \textit{Advanced Modern Algebra}, Prentice Halls, (2003).

\bibitem{S} R. Y. Sharpe, \textit{Steps in Commutative Algebra}, Cambridge University Press, 1990.

\bibitem{TL} A. E. Taylor and D. C. Lay,  \textit{Introduction to functional analysis}, Robert, E. Krieger Publishing Company Inc. 1980.

\bibitem{W1} S. Warner, \textit{Topological Rings}, Elsevier Science Publishers B.V. Amsterdam, 1993.

\bibitem{W2} S. Willard, \textit{General Topology}, Addison-Wesley Publishing Company Inc, 1970.

\bibitem{X} Y. Xue, \textit{A note about a theorem of R. Harte}, Filomat, 22 (2) (2008), 95-98.


\end{thebibliography}

\end{document}

\begin{enumerate}[label=(\alph*)]
\begin{enumerate}[label=(\roman*)]
%...
% Arabic numbers
\begin{enumerate}[label=\arabic*)]
%...
% Alphabetical
\begin{enumerate}[label=\alph*)]
{\color{red} text }
\vspace{-1cm}
%%%%%%%%%%%%%%%%%%%%%%%%%%%
\begin{comment}
\begin{example}[\cite{G1}, p. 98] \label{DivOfZero}
\normalfont
\noindent Let $R$ denote the set of functions $f: \mathbb{R} \to \mathbb{R}$. 
\vskip 0.3cm
\noindent For $f,g \in R$, we define the sum $f+g$ and the product $f \cdot g$ 
for $x \in \mathbb{R}$ by
\begin{center}
$(f + g)(x)=f(x)+g(x)$ \qquad and \qquad $(f \cdot g)(x)=f(x) \cdot g(x)$.
\end{center}
%and
%\begin{center}
%$(f \cdot g)(x)=f(x) \cdot g(x)$.
%\end{center}
\noindent With the operations defined above, $R$ becomes a commutative ring with identity. The identity element in this ring is the element $I$, defined by:
\begin{center}
$I(x) = 1$ for all $x \in \mathbb{R}$
\end{center}
and the zero element is the element $0$, defined as:
\begin{center}
$0(x) = 0$ for all $x \in \mathbb{R}$.
\end{center}
\noindent Now consider $f, g \in R$ defined by
\[ f(x) = \begin{cases} 
      0 & x \neq 0 \\
      1 & x = 0 
   \end{cases}
\]
and
\[ g(x) = \begin{cases} 
      1 & x \neq 0 \\
      0 & x = 1 
   \end{cases}.
\]
\noindent Then $f \not = 0$, $g \not = 0$, but $f \cdot g = 0$. \qed
\end{example}

\noindent Example \ref{DivOfZero} shows that if $R$ is a ring, then it is possible to have 
$a, b \in R, a \neq 0, b \neq 0$, but $ab = 0.$
\end{comment}
%%%%%%%%%%%%%%%%%%%%%%%%%%%%%%%
\begin{comment}
\proof \space Suppose that $1-ab \in R_l^{-1}$. Then there exists $c \in R$ such that $c(1-ab)=1$. 
We show that $1-ba \in R_l^{-1}$. To see that this is the case, we show that $1+bca$ is a left inverse 
for $1-ba$:
\begin{align*}
(1+bca)(1-ba)&=1(1-ba)+bca(1-ba)\\
             &=(1-ba)+bc(a-aba)\\
             &=(1-ba)+bc(1-ab)a\\
			 &=(1-ba)+b1a\\
	         &=1-ba+ba = 1
\end{align*}

\noindent Similarly, suppose that $1-ab \in R_r^{-1}$.  
Then there exists $c \in R$ such that $(1-ab)c=1$ and direct computation shows that $1+bca$ is a 
right inverse for $1-ba$, hence $1-ba \in R_r^{-1}$. Since $R^{-1}=R_l^{-1} \cap R_r^{-1}$ the 
result follows. \qed
\end{comment}
%%%%%%%%%%%%%%%%%%%%%%%%%%%%%%%%%%
\begin{comment}
\begin{definition}[\cite{L2}, p. 32 to 34] 
\normalfont
\noindent Two sets $A$ and $B$ are said to be {\sl equinumerous} or {\sl equivalent}, denoted by $A \sim B$, if and only if there exists a bijective function $f:A \rightarrow B$ between them. 
If $A \sim B$ we say that $A$ and $B$ have the same {\sl cardinality}, written $\#A=\#B$.
\end{definition}
\end{comment}
%%%%%%%%%%%%%%%%%%%%%%%%%%%%%%%%%%%
\begin{comment}
\begin{corollary}
\normalfont
\noindent Let $R$ and $S$ be rings, and let $f$ be a homomorphism from $R$ onto $S$. 
If $a \in R^{-1}$ the $f(a) \in S^{-1}$. 
\end{corollary}
\proof \space Suppose that $R, S$ and $f$ are as described, and suppose $a \in R^{-1}$. Then since 
$f$ is a homomorphism onto, we have:

\end{comment}
\begin{comment}
\begin{definition}[\cite{W1}, p. 32] \hfill \textcolor{red}{remove if not used later}

\medskip \noindent \quad {\sl (1)} \space \textup{A homomorphism which is onto is said to be an {\sl epimorphism}.}

\medskip \noindent \quad {\sl (2)} \space \textup{A homomorphism which is one to one and onto is an {\sl isomorphism}.}

\end{definition}


\begin{lemma}[\cite{F}, p. 310] \textup{Suppose that $R$ and $R'$ are rings and let $\phi:R \to R'$ be a ring homomorphism. If $M$ is a subring of $R$, then $\phi(M)$ is a subring of $R'$. Conversely, if $N$ is a subring of $R'$, then $\phi^{-1}(N)$ is a subring of $R$. In particular:}

\medskip \noindent \text{\space} {\sl (1)} \quad \textup{The ring homomorphism preserves the additive identity: $\phi(0)=0'$}

\medskip \noindent \text{\space} {\sl (2)} \quad \textup{The ring homomorphism preserves the multiplicative unit: $\phi(1)=1'$}

\medskip \noindent \text{\space} {\sl (3)} \quad \textup{The ring homomorphism preserves multiplicative inverses: $\phi(a^{-1})=(\phi(a))^{-1}$}

\end{lemma}
\end{comment}
%%%%%%%%%%%%%%%%%%%%%%%%%%%%
\begin{comment}
\noindent A subset $M$ of a topological space $X$ is said to be {\sl dense} in $X$ if $\cl(M)=X$ sometimes written $\overline{M}=X$.
\end{comment}
%%%%%%%%%%%%%%%%%%%%%%%%%%%%%
\begin{comment}
\begin{proposition} 
\normalfont
The topology $\tau$ of a topological group $\langle G, \tau, \beta \rangle$ is determined by the
collection of neighbourhoods of the identity element $e$ of the group. In other words, we only need to
know the neighbourhoods of the identity element $e$ of $G$ to know all neighbourhoods of all points of 
$G$.

\end{proposition}

\proof \space Suppose $\langle G, \alpha, \tau \rangle$ is a topological group and consider arbitrary element $g \in G$. Define the translation $t_g:G \rightarrow G$ by $t_g(x)=\alpha(g,x)=g+x$. 
We also need the inverse map $\gamma_g:G \rightarrow G \times G$ defined by $\gamma_g(x)=(g,x)$. 
We have by definition that the group operations $\alpha$ is continuous with respect to the product topology $\tau^{\times}$ on $G \times G$. 
First we show that the inverse map $\gamma_g$ is continuous with respect to the topology $\tau$ on $G$, by observing that $\gamma_g^{-1}(g,V) = V$ is open in $G$ for arbitrary open set $(g,V) \subseteq G \times G$, since $\{ U \times V : U, V\ \text{open in}\ \tau \}$ is a base for $\tau^{\times}$ as described in definition 1.3.0.13. 
Since the composition of continuous functions is again continuous, we have that $t_g=\alpha\ \circ \gamma_g$ must also be continuous. Since $t_g$ and $t_{g^{-1}}$ are each others' continuous inverses $t_{g^{-1}}=t_g^{-1}$, we have that translations are homeomorphisms of $G$. i.e.

\medskip \qquad \qquad \qquad \qquad \quad \space $U \in \mathcal{N}_g \quad \text{if and only if}\ \quad t_{g^{-1}}(U) \in \mathcal{N}_e$ \qed

\end{comment}
%%%%%%%%%%%%%%%%%%%%%%%%%%%%%
\begin{comment}

\begin{definition}[\cite{W1}, p. 33] \textup{Let $J$ be an ideal of a topological ring $A$, 
then the quotient topology of $A / J$ is the strongest topology on $A / J$ for which the 
canonical epimorphism $\phi_J : A \to A / J$ defined by $\phi_J(x)=[x]=x+J $ for all $x \in J$ 
is continuous;}

\medskip \qquad \qquad \qquad \qquad \textup{$\tilde{\tau} = \{ E \subseteq A / J: \phi_J^{-1}(E)\ \text{open in}\ A \}$}

\medskip \noindent \textup{The quotient topological space $\langle A / J , \tilde{\tau} \rangle$ constitutes the quotient set $A / J$ endowed with the quotient topology $\tilde{\tau}$. 
The operations of addition and multiplication on $A / J$ are those described in definition 1.2.0.13 when $A / J$ is considered as a ring.}
\end{definition}
\end{comment}
%%%%%%%%%%%%%%%%%%%%%%%%%%%%%%%%%%
\begin{comment}
\noindent This enables us to define the {\sl norm closure} of a subset $M \subseteq X$ by

\begin{center}
$\cl_{|| \cdot ||}(M)=M \cup \der(M)$
\end{center}

\noindent the set of those points $x \in X$ for which the sequence $(x_n)$ converges to $x$, together with all its terms, where $x \in \der(M) \text{\space if and only if \space} B(x_o,\epsilon) \cap \left(M \setminus \{x_o\}\right) \not = \emptyset$.
\vskip 0.3cm
\noindent In a normed space we shall write $\dist(x,M)= \inf \{ \| x-y \|: y \in M \}$ alternatively we can express the closure as $\cl(M)=\{ x \in M: \dist(x,M)=0 \}$.

\vskip 0.3cm
\noindent The {\sl boundary} $\partial M$ of a subset $M$ of a topological space $X$ is the set of all points in the closure $\cl(M)$ of $M$ not belonging to the interior $\inter(M)$ of $M$, symbolically: $$\partial M = \cl(M) \setminus \inter(M)$$
\end{comment}
%%%%%%%%%%%%%%%%%%%%%%%%%%%%%%%%%%%%%%%
\begin{comment}
\begin{proof}
Let $||x|| = r < 1$ and let $s_n = \displaystyle \sum_{k=0}^n x^k$. For $n < m$, we have
$$||s_m - s_n|| \leqslant \displaystyle \sum_{k = n+1}^m ||x||^k \leqslant \dfrac{r^{n+1}}{1 - r},$$
so $(s_n)$, the sequence of partial sums, is a Cauchy sequence converging to some element 
$a = \sum_{k=0}^{\infty} x^k$. 
\vskip 0.3cm
\noindent Next, since $x s_n = s_{n+1} - 1$, we have:
\begin{align*}
\lim_{n \rightarrow \infty} x s_n &= \lim_{n \rightarrow \infty} (s_{n + 1} - 1) \\
\therefore xa &= a - 1\\
\end{align*}
Hence 
\begin{equation}\label{rInvertible}
(1 - x) a = 1
\end{equation}
\noindent Similarly, since $s_n x = s_{n+1} - 1$, we have:
\begin{align*}
\lim_{n \rightarrow \infty} s_n x &= \lim_{n \rightarrow \infty} (s_{n + 1} - 1) \\
\therefore ax &= a - 1\\
\end{align*}
Hence 
\begin{equation}\label{lInvertible}
a(1 - x) = 1
\end{equation}
The equations \eqref{rInvertible} and \eqref{lInvertible} complete the proof.
\end{proof}

\end{comment}

%%%%%%%%%%%%%%%%%%%%%%%%%%%%%%%%%%%%%

\begin{comment}
\begin{proof}
\noindent
We have that $x = a + x - a = a(1 + a^{-1}(x - a))$. Hence
\begin{align*}
||a^{-1}(a - x)|| 
&= ||a^{-1}(x - a)||\\
&\leqslant ||x - a||\cdot ||a^{-1}|| < 1
\end{align*}
\noindent Hence by Theorem \ref{oneMinXInvert}, $1 + a^{-1}(x - a)$ is invertible, and so $x$ 
is invertible by part (a) of Lemma \ref{RInverseStructure1}. Also,
\begin{center}
$\|x^{-1} - a^{-1}\| \leqslant \|a^{-1}\|^2\|x - a\|$
\end{center}
\end{proof}
\end{comment}
%%%%%%%%%%%%%%%%%%%%%%%%%%%%

\begin{comment}
\begin{lemma}\label{NormClosureOfZero}
\normalfont
Let $A$ be a Banach algebra. Then $\cl_{||\cdot||}{\{0\}} = \{0\}.$
\end{lemma}
\proof \space From Theorem \ref{ClosureITODerivedSet} We prove the lemma by showing that if 
$a \in A, a \neq 0$, then $a \notin \der(\{0\})$. If $a \not = 0$ then $\|a\|$ is the distance 
from $a$ to 0. So, $B(a, \dfrac{\|a\|}{2})$ is a neighbourhood of $a$ that contains no point 
from $\{0\}$, hence $a \notin \der \{0\}$.
\end{comment}
%%%%%%%%%%%%%%%%%%%%%%%%%
\begin{comment}
\proof \space We show that  $A^{-1} \subseteq \cl_{\| \cdot \|}(A_l^{-1}) \cap A_r^{-1}$ and that 
$\cl_{\| \cdot \|} ( A^{-1}_l) \cap A_r^{-1} \subseteq A^{-1}$. 
The proof of the version involving the right invertibles is a mirror image of the one for left invertibles. To see the first inclusion we note that $A^{-1}_l \subseteq \cl_{\| \cdot \|}(A_l^{-1})$, and so
\begin{center}
$A^{-1} = A^{-1}_l \cap A^{-1}_r \subseteq \cl_{\| \cdot \|}(A_l^{-1}) \cap A^{-1}_r $.
\end{center}
\noindent For the second inclusion, let $a \in \cl_{\| \cdot \|}(A_l^{-1}) \cap A^{-1}_r.$ 
Then $a \in A^{-1}_l$ or $a \in \der A^{-1}_l$. If $a \in A^{-1}_l$ then $a \in A^{-1}_l \cap A^{-1}_r$, hence $a \in A^{-1}$ and the proof is done. If $a \in \der(A^{-1}_l)$ then there exists a 
sequence $(a_n)$ such that $a_n \in A^{-1}_l$ and $(a_n)$ converges to $a$. 
Since $a_n \in A^{-1}_l$ there exists $b_n \in A$ such that $b_n a_n = 1$ for all $n$. Now we have:
\begin{center}
$1 = \displaystyle \lim_{n \rightarrow \infty} 1 = \displaystyle \lim_{n \rightarrow \infty} b_n a_n
= \displaystyle \lim_{n \rightarrow \infty} b_n \displaystyle \lim_{n \rightarrow \infty} a_n
= (\displaystyle \lim_{n \rightarrow \infty} b_n) \cdot a$
\end{center}
\noindent The above shows that the sequence $(a_n)$ converges to an element in $A^{-1}_l$. 
Hence $a \in A^{-1}_l$, which means that $a \in A^{-1}_l \cap A^{-1}_r = A^{-1}$ and the result follows.
\qed
\end{comment}
%%%%%%%%%%%%%%%%%%%%%%%%%%%

\begin{comment}
\proof \space Suppose that $a \in \QN_{\| \cdot \|}(A)$. Then $\| a^n \|^{\frac{1}{n}} \xlongrightarrow[n]{\infty} 0$. From lemma 1.5.1.3, we have that $\rho(a)=0$ giving $\sigma(a)=\{ 0 \}$.

\vskip 0.3cm

\noindent Conversely suppose that $\sigma(a)=\{ 0 \}$ for arbitrary $a \in A$, then $\rho(a)=0$ by definition. From lemma 1.5.1.3, Gelfand's spectral radius formula gives $\displaystyle{\lim_{n \to \infty} \| a^n \|^{\frac{1}{n}}} \xlongrightarrow[n]{\infty} 0$, exactly the definition for quasinilpotence of $a \in \QN_{\| \cdot \|}(A)$. \qed
\end{comment}

%%%%%%%%%%%%%%%%%%%%%%%%%%%%%

\begin{comment}
\begin{remark} 
\normalfont
\noindent Recall that $A^{-1}$ is an open set in a Banach algebra $A$, therefore $a \in \partial A^{-1}$ if and only if $a \in \cl(A^{-1})$ and $a \not \in \inter(A^{-1})$. 
To elaborate even further, $a \in \cl(A^{-1})$ if and only if it is the limit of a sequence of points $(a_n)$ in $A^{-1}$ and $a \not \in \inter(A^{-1})$ if and only if there is no open set $U$ such that $a \in U \subseteq A^{-1}$ giving that $a \not \in A^{-1}$ because $A^{-1}=\inter(A^{-1})$, since $A^{-1}$ is open. \qed
\end{remark}
\end{comment}

%%%%%%%%%%%%%%%%%%%%%%%%%

\begin{comment}
\proof
\begin{enumerate}[label=(\alph*)]
\item Suppose that $x,y \in R^{-1}$. Then $x^{-1}$ and $y^{-1}$ exist and 
\begin{align*}
(xy)(y^{-1}x^{-1})&=x(yy^{-1})x^{-1}=x1x^{-1}=xx^{-1}=1 \quad \text{and} \\
(y^{-1}x^{-1})(xy)&=y^{-1}(x^{-1}x)y=y^{-1}1y=y^{-1}y=1.
\end{align*}

\noindent The above show that $(xy)^{-1}=y^{-1}x^{-1}$. Hence $xy \in R^{-1}$.

\item If $xy \in R^{-1}$ there exists $z \in R$ such that

\begin{center} 
$(xy)z=x(yz)=1$, hence $x \in R^{-1}_r$ and $z(xy)=(zx)y=1$, hence $y \in R^{-1}_l$.
\end{center}
%and 
%\begin{center}
%$z(xy)=(zx)y=1$ \qquad \text{hence $y \in R^{-1}_r$.}
%\end{center}

\item If $xy,y \in R^{-1}$ then $y^{-1}$ exists and $y^{-1} \in R^{-1}$. Hence 
\begin{center}
$x=(xy)y^{-1} \in R^{-1}$ \quad (using (a)).
\end{center}

\item Suppose that $x \in R^{-1}$ and $y \not \in R^{-1}$ and suppose that $xy \in R^{-1}$. 
\vskip 0.3cm
Then there exists $z \in R^{-1}$ such that 
\begin{center}
$z(xy)=1=(xy)z$ 
\end{center}
which means
\begin{center}
 $z(xy)=1 \implies (zx)y=1$ \hfill (1)
\end{center}
and 
\begin{center}
$(xy)z=1 \implies x^{-1}(xy)z=x^{-1} \implies yz=x^{-1} \implies y(zx)=1$ \hfill (2)
\end{center}

Combining (1) and (2) gives $y \in R^{-1}$, contradicting our assumption that $y \not \in R^{-1}$. 
Hence $xy \not \in R^{-1}$, as desired. \qed
\end{enumerate}
\end{comment}

%%%%%%%%%%%%%%%%%%%%%%%%%%%%%%%%%%

\begin{comment}
\vskip 0.3cm

$\therefore\ 1-h \cdot l (a-a') \in R^{-1}$ for all $h\in H, l \in L$

$\therefore\ 1- l (a-a')h \in R^{-1}$ for all $h\in H, l \in L$ \hfill from Lemma \ref{Jacobson}.

$\therefore\ 1-L(a-a')H \subseteq R^{-1}$

$\therefore\ a \in \CL_2(K)$.
\end{comment}

%%%%%%%%%%%%%%%%%%%%%%%%%%%%%%%

\begin{comment}
\begin{corollary} 
\normalfont
$\Rad(\mathbb{R}) = \{ 0 \}$, i.e. $\mathbb{R}$ is semisimple, considered as a ring.

\end{corollary}

\proof  This follows immediately from  the fact that $\mathbb{R}$, equipped with the spectral topology is a $T_1$ space and part {\sl (3)} of theorem 2.3.0.5.
\end{comment}

%%%%%%%%%%%%%%%%%%%%%%%%%%%%%%%%%

\begin{comment}
\item An element which is both nearly left invertible and nearly right invertible is said to 
be {\sl nearly invertible}: $x \in \CL(R_l^{-1}) \cap \CL(R_r^{-1})$ if and only if $x \in \CL(R^{-1})$.
\end{comment}

%%%%%%%%%%%%%%%%%%%%%%%%%%%%

\begin{comment}
\begin{lemma}[Identity (4.8) from \cite{CH3}, p. 271] \textup{For sequence $(x_n)$ to converge to zero, there must be a family $(N_J)$ of natural numbers, indexed by finite $J \subseteq R$:}

\medskip {\sl (4.8)} \space $n \geq N \implies 1+J x_n \subseteq R^{-1}$

\end{lemma}
\proof \space Suppose that the sequence $(x_n)$ of points in arbitrary $K \subseteq R$ converges to $x$. Consider $x_n \in K \subseteq \CL(K)$ for all $n \in \mathbb{N}$. $x_n \in \CL(K)$ if and only if for every finite $J \subseteq R$, there exists $x \in K$ such that $1-J(x_n-x)\subseteq R^{-1}$. 
If we take $K=\{0\}$, we obtain $x_n \xlongrightarrow[n]{\infty} 0$ in terms of $\CL(\cdot)$. So, $x_n \in \CL(\{ 0 \})$ if and only if for every finite $J \subseteq R$, there exists $x \in \{ 0 \}$ such that $1-J(x_n-x) \subseteq R^{-1}$, reducing to $x=0$ and the expression simplifies to $1-J x_n \subseteq R^{-1}$ as desired. \qed

\end{comment}

%%%%%%%%%%%%%%%%%%%%%%%%%%%%%%%

\begin{comment}

\begin{remark}[Speculation]

\textup{We would like to investigate whether the Riemann-Liouville operator from example 1.5.1.10 is quasinilpotent in the generalized sense of a ring (at least in a semisimple ring so that the topological ring equipped with the spectral topology is a $T_1$ space, by part {\sl (3)} of theorem 2.3.0.5), rather than on a normed space and for values of the parameter $\alpha$ outside of the natural numbers $\mathbb{N}$.}

\end{remark}


\begin{remark}[\cite{CH3}, p 272] \hfill

\medskip \noindent \textup{\textbf{(Unverified summation property for commutative quasinilpotent elements)}}

\medskip \noindent \quad \textup{"We have not been able to settle whether or not there is implication"}

\medskip \noindent \quad \quad {\sl (5.7)} \space 
$a,b \in \QN(R); ab=ba \xLongrightarrow[\textup{\mbox{?`}}]{?} a+b \in \QN(R)$

\end{remark}

\end{comment}

%%%%%%%%%%%%%%%%%%%%%%%%%%%%%%%%%%%%%%

\begin{comment}
{\color{red} Everything below in red still to be reviewed
\medskip \noindent "The spectral closure also intervenes in abstract Fredholm theory. Classically, Fredholm  operators on Banach spaces always have generalized inverses (\cite{?}, p. ?)."

\medskip \noindent "Fredholm  operators on Banach spaces have invertible generalized inverses iff they are of index zero; thus theorem 6 says that nearly invertible Fredholm operators have index zero."

\vskip 0.3cm
To keep the result below, we will keep all the Basic Fredholm Theory in Chapter 1, but we will assume that the basic Functional Analysis it is built on is known.
\vskip 0.3cm

\begin{proposition}[\cite{CH3}, p. 273] \textup{Let $T$ be a Fredholm operator on a Banach space 
$A$, i.e. $T \in \Phi(A)$. Then there always exists a pseudo (generalized) inverse $S \in A^{\cap}$ 
such that $T=TST$. Furthermore $S \in A^{\cup}$, i.e. $S$ is also invertible if and only if $T$ 
has index zero.}

\end{proposition}


\proof \space The first statement is exactly theorem 1.5.2.26. The second statement is 
exactly {\sl (b)} of theorem 1.5.2.27 where a pseudoinverse operator is invertible if and only if it 
is bijective. \qed

\vskip 0.3cm

\noindent "This survives in an abstract context, where

\smallskip \qquad \qquad \quad homomorphisms $T:R_1 \rightarrow R_2$ between rings

\smallskip \noindent \qquad \qquad \qquad \qquad \qquad \quad bring with them Weyl and Fredholm [10],[11],[18] elements":
}
\end{comment}

%%%%%%%%%%%%%%%%%%%%%%%%%%%%%%%%%%%
\begin{comment}
\textup{"{\sl (7.1)} applies in particular when $R_2=R_1 / J$ is the quotient of $R_1$ by a two sided ideal $J \subseteq R_1$ and {\sl (7.2)} tells us that the spectral topology for $R_1 / J$ is weaker than or equal to the quotient of the spectral topology for $R_1$."}


\medskip \noindent \textup{We elaborate for sake of clarity that, since the canonical epimorphism $\phi: R_1 \to R_1 / J $ defined by $\phi (x)=[x]=x+J$, where $J$ is a two sided ideal of $R_1$, is indeed onto, hence satisfying condition {\sl (7.1)} $R_1 / J \subseteq \phi (R_1)$. So from the previous theorem 6.2.0.11, we get from {\sl (7.2)} that \space $\phi (\CL_{R_1}(K)) \subseteq \CL_{R_1 / J}(\phi(K))$.}
\end{comment}

%%%%%%%%%%%%%%%%%%%%%%%%%%%%%%

\begin{comment}
We check the details below, using the definition in \cite{F}, p 40:
\begin{itemize}
\item [G0.] \quad $R^{-1}$ is closed under $\cdot$. The proof of this fact is essentially Lemma \ref{RInverseStructure1}, part (a). 
\item [G1.] \quad By the associativity of $\cdot$ in $R$, we have that $\cdot$ is associative on $R^{-1}$.
\item [G2.] \quad $1 \in R^{-1}$ and if $a \in R^{-1}$ then
\begin{center}
$1 \cdot a = a \cdot 1 = a$
\end{center}
shows that $1$ is the identity element in $R^{-1}$.
\item [G3.] \quad If $a \in R^{-1}$ then by definition of $R^{-1}$ there exists $a^{-1}$ such that
\begin{center}
$a^{-1} \cdot a = a \cdot a^{-1} = 1$
\end{center}
\end{itemize}
\end{comment}

%%%%%%%%%%%%%%%%%%%%%%%%%%%%%%%%%%%

\begin{comment}
\bigskip \noindent Classical operator theory is animated in the setting of normed spaces.


\begin{definition}[\cite{K}, p. 82] \textup{A {\sl linear operator} $T$ is an operator such that.}

\begin{enumerate}

\item[(i)] \textup{the domain $\mathscr{D}(T)$ of $T$ is a vector space and the range $\mathscr{R}(T)$ lies in a vector space over the same field,}

\item[(ii)] \textup{for all $x, y \in \mathscr{D}(T)$ and scalars $\alpha$, we have $T(x+y)=Tx+Ty$ and $T(\alpha x)= \alpha Tx$}


\end{enumerate}


\end{definition}


\begin{definition}[\cite{K}, p. 99] \textup{Two operators $T_1$ and $T_2$ are defined to be {\sl equal}, written $T_1=T_2$ if they have the same domain $\mathscr{D}(T_1)=\mathscr{D}(T_2)$ and if $T_1 x = T_2 x$ for all $x \in \mathscr{D}(T_1)=\mathscr{D}(T_2)$. The {\sl restriction} of an operator $T: \mathscr{D}(T) \to Y$ to a subset $B \subset \mathscr{D}(T)$ is denoted by $T \big |_B$ and is the operator $T \big |_B : B \to Y$ defined by $T \big |_B x = T x$ for all $x \in B$. An {\sl extension} of $T$ to a set $\mathscr{D}(T) \subset M$ is an operator $\tilde{T}: M \to Y$ such that $T= \tilde{T} \big |_{\mathscr{D}(T)}$, that is, $\tilde{T}x = Tx$ for all $x \in \mathscr{D}(T)$ [hence $T$ is the restriction of $\tilde{T}$ to $\mathscr{D}(T)$].}

\end{definition}



\begin{definition}[\cite{G2}, p. 43] \textup{The {\sl graph} $\mathcal{G}(T)$ of a linear operator $T$ from normed space $X$ to normed space $Y$ is the set $\{ (x,Tx): x \in \mathscr{D}(T) \}$ which is a subspace of $X \times Y$. If the graph of $T$ is closed in $X \times Y$, then $T$ is said to be (a) {\sl closed} (operator) in $X$.}

\end{definition}


\begin{definition}[\cite{K}, p. 91] \textup{Let $X$ and $Y$ be normed spaces and $T: \mathscr{D}(T) \to Y$ a linear operator, where $\mathscr{D}(T) \subset X$. The operator $T$ is said to be {\sl bounded} if there exists a positive real number $c$ such that for all $x \in \mathscr{D}(T)$ we have that $$\| Tx \|_Y \leq c \| x \|_X $$ }

\end{definition}


\begin{proposition}[\cite{K}, p. 97] \textup{Let $T: \mathscr{D}(T) \to Y$ be a linear operator, where $\mathscr{D}(T) \subset X$ and $X$ and $Y$ are normed space. Then $T$ is continuous if and only if $T$ is bounded.} \qed

\end{proposition}


\begin{definition}[\cite{K}, p. 41] \textup{ Let $(X,d)$ and $(\tilde{X},\tilde{d})$ be metric spaces. Then:}


\begin{enumerate}

\item[\textup{(a)}] \textup{A mapping $T$ of $X$ into $\tilde{X}$ is said to be {\sl isometric} or an {\sl isometry} if $T$ preserves distances, that is, if  $\tilde{d}(Tx,Ty)=d(x,y)$ for all $x, y \in X$, where $Tx$ and $Ty$ are the images of $x$ and $y$, respectively.}

\item[\textup{(b)}] \textup{The space $X$ is said to be {\sl isometric} with the space $\tilde{X}$ if there exists a bijective isometry of $X$ onto $\tilde{X}$. The spaces $X$ and $\tilde{X}$ are then called {\sl isometric} spaces.}


\end{enumerate}


\end{definition}


\begin{lemma}[\cite{G2}, p. 15] \textup{If normed linear space $Y$ is isometric to a Banach space, then $Y$ is also a Banach space.} \qed

\end{lemma}

\begin{lemma}[\cite{G2}, p. 9] \textup{If $X$ is a Banach space and $M$ is a closed subspace of $X$, then $X/M$ is a Banach space.} \qed

\end{lemma}


\begin{lemma}[\cite{K}, p. 67] \textup{A subspace $Y$ of a Banach space $X$ is complete if and only if the set $Y$ is closed in $X$.} \qed

\end{lemma}

\begin{proposition}[\cite{K}, p. 292 \& \cite{G2}, 45] \textup{Let $X$ and $Y$ be Banach spaces and $T: \mathscr{D}(T) \to Y$ be a closed linear operator where $\mathscr{D}(T) \subset X$. Then if $\mathscr{D}(T)$ is closed in $X$, the operator $T$ is bounded.} \qed

\end{proposition}

\begin{proposition}[\cite{G2}, p. 45 - theorem {\sl \textrm{II} 1.9}] \textup{A closed linear operator mapping a Banach space into a Banach space is continuous.} \qed

\end{proposition}
\end{comment}

%%%%%%%%%%%%%%%%%%%%%%%%%%%%%%%%%%

\begin{comment}
\begin{construction}[\cite{G2}, p. 11 \& \cite{K}, p. 118 - 119] \hfill

\bigskip

\noindent \textup{The collection of all bounded linear operators $\mathcal{BL}(X,Y)$ which map $X$ into $Y$ with the usual component-wise definition of vector addition $(T + S ) x = T x + S x$ and scalar multiplication $( \alpha T ) x = \alpha  T x  $ for all $x \in X$ and $\alpha \in \mathbb{K}$ where $T,S \in \mathcal{BL}(X,Y)$, forms a vector space when equipped with the norm derived from boundedness:}

\bigskip \qquad \qquad \qquad \qquad \qquad $\displaystyle{  \| T \|=\sup_{\underset{x \not = 0}{x \in \mathscr{D}(T)}} \frac{\| Tx \|}{\| x \|}= \sup_{\underset{\| x \| = 1}{x \in \mathscr{D}(T)}} \| Tx \| }$

\medskip

\noindent \textup{Restricting $Y=\mathbb{K}$ to the underlying scalar field, we define the {\sl dual} or {\sl conjugate} space of normed space $X$ in a similar way, denoted $X'$ constituting all bounded linear functionals $f: X \to \mathbb{K}$ with the usual component-wise definition of vector addition $(f+g)(x)=f(x)+g(x)$ and scalar multiplication $( \alpha f ) x = \alpha f(x)$ for all $x \in X$ and $\alpha \in \mathbb{K}$ where $f,g \in \mathcal{BL}(X,\mathbb{K})$, equipped with the norm defined on $\mathcal{BL}(X,\mathbb{K})$ by:}

\bigskip \qquad \qquad \qquad \qquad \qquad $\displaystyle{  \| f \|=\sup_{\underset{x \not = 0}{x \in \mathscr{D}(f)}} \frac{ | f(x) | }{\| x \|}= \sup_{\underset{\| x \| = 1}{x \in \mathscr{D}(f)}} | f(x) | }$

\end{construction}


\begin{proposition}[\cite{K}, p. 100] \textup{Let $T: \mathscr{D}(T) \to Y$ be a bounded linear operator, where $\mathscr{D}(T)$ lies in a normed space $X$ and $Y$ is a Banach space. Then $T$ has an extension $\tilde{T}: \overline{\mathscr{D}(T)} \to Y$ where $\tilde{T}$ is a bounded linear operator of norm $\| \tilde{T} \|=\| T \|$.} \qed
\end{proposition}



\begin{remark} \textup{Closed operators $T: X \to Y$ having domain dense in $X$ (i.e. $\cl(\mathscr{D}(T))=X$) is commonly of interest, since in this case the domain is closed $\mathscr{D}(T)=X=\cl(X)=\cl(\mathscr{D}(T))$ guaranteeing for Banach spaces $X$ and $Y$, that $T$ is bounded and that there exists a unique continuous Linear extension $\tilde{T}$ of $T$ to all of $X$, so we may restrict our attention to operators satisfying this desired criteria.}
\end{remark}



\begin{proposition}[\cite{G2}, p. 43 - Remark {\sl \textrm{(II)} 1.3 \textrm{(iv)}}] \hfill

\smallskip \noindent \textup{If $\mathscr{D}(T)$ is closed and $T$ is continuous, then $T$ is closed.} \qed

\end{proposition}







\begin{remark} \hfill 

\begin{enumerate}

\item[(1)] \textup{Care should be taken to discern the difference between the use of the phrases {\sl closed operator} and {\sl closed range operator}, the former is definition 1.5.2.3 and the latter means that $\mathscr{R}(T)$ is closed in $Y$, for some operator $T : X \to Y$.}


\item[(2)] \textup{The adjoint operators $T^{\times} : Y \to X$ has elegant properties when $X$ and $Y$ are Banach, as theorem 1.5.2.19 illustrates, allowing us to disregard the closed range criteria for an operator to be classified as Fredholm, which was historically required by definition.}

\end{enumerate} 

\end{remark}


\begin{definition}[\cite{F}, p. 311, \cite{G2}, p. 102, \cite{H1}, p. 29] \hfill

\medskip \noindent \textup{Let $\phi : R \to R'$ be a homomorphism between vector spaces $R$ and $R'$, then:}

\medskip \noindent \textup{The subspace $\phi^{-1}(\{ 0' \})=\{ a \in R: \phi(a)=0' \}=\ker(\phi)$ is called the kernel or null space of $\phi$ and its dimension $\alpha(\phi)= \dim(\ker(\phi))$ is called the kernel index of $\phi$.}

\medskip \noindent \textup{The cokernel of $\phi$ is the quotient space $\coker(\phi)=R' / \mathscr{R}(\phi)$ of the codomain $R'$ by the range $\mathscr{R}(\phi)$ and its dimension $\beta(\phi)= \dim(R' / \mathscr{R}(\phi))$ is the deficiency index of $\phi$ -  \cite{K}, p. 57.}

\end{definition}





\begin{theorem}

\textup{If $T$ is a bounded linear operator between Banach spaces $X$ and $Y$ with finite deficiency index $\beta(T)=\dim(Y / \mathscr{R}(T)) < \infty$,  then $\mathscr{R}(T)$ is a closed subspace of $Y$, i.e. $\cl(\mathscr{R}(T))=\mathscr{R}(T)$.}

\end{theorem}




\proof Let $\pi:Y \to Y / \mathscr{R}(T)$ be a quotient mapping defined by $\pi(x)=[x]$ for $x \in X$. \hfill

\bigskip

Suppose that the linearly independent set $\{z_1,z_2, \dots , z_n\}$ forms a Hamel basis which spans $Y / \mathscr{R}(T)$ such that any $z \in Y / \mathscr{R}(T)$ can be expressed as a finite linear combination of basis elements and coefficients in the scalar field $\mathbb{K}$. Assign/identify $\pi(y_i)=z_i$ corresponding to each $y_1, y_2, \dots , y_n \in Y$. Then $V=\spn{\{ y_1, y_2, \dots , y_n \}}$ is the topological complement of $\mathscr{R}(T)$, satisfying $V + \mathscr{R}(T)=Y$ and $V \cap \mathscr{R}(T) =\{0\}$.

\bigskip

Notice that finite dimensional $V$ is a closed subspace of Banach space $Y$, so from lemma 1.5.2.8 we have that the quotient $Y / V$ is a Banach space and the quotient mapping $\eta: Y \to Y / V$ is continuous/bounded, by definition.



\bigskip


Consider the mapping $L: X / \ker(T) \to Y$ defined by $L([x])=T(x)$. It is easy to verify that $L$ is bounded with $\| L \| \leq \| T \|$ and also a bijection onto $L(X)=T(X)$.


\bigskip


The composition $X / \ker(T) \xlongrightarrow{L} Y \xlongrightarrow{ \eta } Y / V$ is a bijection, therefore an isomorphism between Banach spaces, hence $L \circ (\eta \circ L)^{-1} : Y / V \to \mathscr{R}(T) $ is an isomorphism from Banach space $Y / V$ 
to normed space $\mathscr{R}(T)$, thus from lemma 1.5.2.7 we have that $\mathscr{R}(T)$ is also a Banach space, hence closed in $Y$ according to lemma 1.5.2.9. \qed

\begin{lemma}[\cite{G2}, p. 102 - Theorem {\sl \textrm{IV}} 2.3 {\sl \textrm{(i)}}] \textup{If $T \in \mathcal{BL}(X,Y)$ has dense domain $\cl(\mathscr{D}(T))=X$ and closed range $\cl(\mathscr{R}(T))=\mathscr{R}(T)$, then $\alpha(T^{\times})=\beta(T)$.} \qed

\end{lemma}
\end{comment}

%%%%%%%%%%%%%%%%%%%%%%%%%%%%%%%%%%%%%%%%%

\begin{comment}
\begin{definition}[\cite{H1}, p. 192 \& Theorem 6.11.5 p. 223] \textup{A Fredholm operator $T: X \to Y$ is a bounded linear operator $T \in \mathcal{BL}(X,Y)$ between Banach spaces $X$ and $Y$, having finite integer index defined $i(T)=\alpha(T)-\beta(T) \in \mathbb{Z}$. We denote the class of Fredholm operators acting between Banach spaces $X$ to $Y$ by $\Phi(X,Y)$ and if $X=Y$ we simply write $\Phi(X)$.}

\end{definition}


\begin{remark} \textup{By using lemma 1.5.2.20, the index of a Fredholm operator in definition 1.5.2.21 may be equivalently expressed in terms of the more aesthetic formula $i(T)=\alpha(T)-\alpha(T^{\times})$.}

\end{remark}
\end{comment}

%%%%%%%%%%%%%%%%%%%%%%%%%%%%%%%%

\begin{comment}

\begin{definition}[\cite{H1}, p. 192] \textup{A Fredholm operator for which $\alpha(T)=\alpha(T^{\times})$ has zero index and is called {\sl Weyl}. We denote the class of Weyl operators acting between Banach spaces $X$ to $Y$ by $\Omega(X,Y)$ and if $X=Y$ we simply write $\Omega(X)$.}

\end{definition}


\begin{remark} 

\textup{From definition 1.5.2.21 and 1.5.2.23, clearly $\Omega(X,Y) \subseteq \Phi(X,Y)$.}

\end{remark}
\end{comment}

%%%%%%%%%%%%%%%%%%%%%%%%%%%%%%%%%%%%%%%

\begin{comment}
\begin{remark} \textup{We are interested in the class of epimorphisms over a semisimple ring equipped with the spectral topology, since this gives us many useful properties:}
\bigskip \noindent \quad \textup{The space is a $T_1$ space (see theorem 2.3.0.5) }
\medskip \noindent \quad \textup{The invertible group is open (see theorem 3.2.0.5) }
\medskip \noindent \quad \textup{The maps are continuous (see theorem 6.2.0.11) }
\medskip \noindent \quad \textup{Quasinilpotents are nearly invertible (see theorem 4.2.2.4) }
\end{remark}
\end{comment}

%%%%%%%%%%%%%%%%%%%%%%%%%%%%%%%%%%%%%%%


\begin{comment}
\proof \space Suppose that $1+J \subseteq \CL_A(A^{-1})$, i.e. $J$ is weakly Riesz. We show that $1+J \subseteq \CL_A(1+J)^{-1}$. 
Consider arbitrary $a \in 1+J$ then for arbitrary finite $L \subseteq A$ there exists $c \in A^{-1}$ for which, changing sign, $1+(L \cup L a L) (a-c)=U^{-1}=A^{-1}$. 
Now, with $U'=cU$, $U' \subseteq 1+J$ and $1-L(a-U') \subseteq 1-L(a-cU) \subseteq ((1-La)(1+La-Lc)+Lc)U=1+LaL(a-c) \subseteq A^{-1}$

\medskip Conversely, suppose that $1+J \subseteq \CL_A(1+J)^{-1}$. Notice that $(1+J)^{-1} \subseteq A^{-1}$, so if we apply the Kuratowski closure property from part {\sl (i)} of theorem 1.3.0.15, we obtain $\CL_A(1+J)^{-1} \subseteq \CL_A(A^{-1})$. But $1+J \subseteq \CL_A(1+J)^{-1}$ therefore $1+J \subseteq \CL_A(A^{-1})$.

\bigskip Finally, suppose that $1+J \subseteq \CL_A(1+J)^{-1}$ then $1+J \subseteq (1+J) \cap \CL_A(1+J)^{-1}$. From proposition 1.3.0.16 we have that $(1+J) \subseteq \CL_{1+J}(1+J)^{-1}$ i.e. $J$ is weakly Riesz relative to the subspace $1+J$. \qed
\end{comment}

%%%%%%%%%%%%%%%%%%%%%%%%%%%%%%%%%%%%%%

\begin{comment}
\begin{remark}
\textup{It may be useful to keep in mind that $K_0$ is a subset of both the power series ring 
$\mathbb{C}[[z]]$ and the underlying scalar field $ \mathbb{C}$.}
\end{remark}


\proof Suppose that $K$ is closed in $R=\mathbb{C}[[z]]$ then $\CL(K)=K$. Consider arbitrary formal power series $f(z)=(a_0,a_1,a_2, \dots) \in K$ then $f(z) \in \CL(K)$ which means that for every finite subset $J$ of $R$ there exists $g(z) =(b_0,b_1,b_2, \dots) \in K$ such that $1-J(f(z)-g(z)) \subseteq R^{-1}$. To see that $K_0$ is finite, combine lemma 1.2.1.8 with the observation that the leading coefficients in $K_0$ is limited to the finite criteria for $J$.


\medskip

\noindent Conversely, suppose that $K$ is not closed in $R$ then $\CL(K) \not \subseteq K$. So there exists $f(z) = (a_0,a_1,a_2, \dots) \in  \CL(K)$ such that $f(z) \not \in K$. We must show that $K_0$ is not finite. Suppose by way of contradiction that $K_0$ is instead a finite subset of $R$. Then there exists $g(z) = (b_0,b_1,b_2, \dots) \in K$ such that $1-K_0(f(z)-g(z)) \subseteq R^{-1}$. However $f(z) \not \in K$ but $g(z) \in K$ so $a_0 \not \in K_0$ but $b_0 \in K_0$ giving that $a_0-b_0 \not = 0$ therefore $\displaystyle{f(z)-g(z) \not \in \Rad R=\{ h(z)=\sum_{n \geq 0} c_n z^n : a_0 = 0 \}}$ thus from proposition 1.2.1.2 we have that $1-R(f(z)-g(z)) \not \subseteq R^{-1}$ which contradicts that $1-K_0(f(z)-g(z)) \subseteq R^{-1}$. The assumption that $K_0$ is finite leads to an absurdity and by the contrapositive of this argument we have that if $K_0$ is finite then $K$ is closed, as desired. \qed
\end{comment}
\begin{comment}
\section{Generalized exponentials, subsemigroups \newline \& stable rank}


\begin{definition}[\cite{H1}, p. 292]
\normalfont
\noindent Let $A$ is a Banach algebra and $a \in A$. Then
\begin{center} $\displaystyle{e^{a}=\exp(a)=\sum_{n=0}^{\infty} \frac{a^n}{n!}}=1+a+ \frac{a^2}{2!} + \frac{a^3}{3!} \dots$
\end{center}
and
\begin{center}
$\Exp(X)=\{ e^{a_1} \cdot e^{a_2} \dots  e^{a_k}: k \in \mathbb{N}, \{ a_1,a_2, \dots, a_k \} \subseteq A \}$ 
\end{center}

\end{definition}

\begin{definition}[\cite{D}, p. 1] 
\normalfont A nonempty set $S$ with an associative binary operation is called a semigroup. 
A nonempty subset $H$ of a semigroup $S$ is called a {\sl subsemigroup} if $H\cdot H = H$. 
\end{definition}

\noindent The study of semigroups and related concepts rooted in this rich theory, such as (infinitesimal) semigroup generators, generalized exponentials, Hille-Yoshida-type results and rank stability, arise naturally in other scientific disciplines, such as applied mathematics and physics. The starting point of this formalism is due to Jacques Hadamard (1865 - 1963) who suggested a mathematical framework for studying the {\sl well-posedness} of mathematical models, by having the properties that:

\begin{enumerate}[label=(\arabic*)]
\item \quad a solution exists
\item \quad the solution is unique
\item \quad the solution depends continuously on the data and parameters
\end{enumerate}

\noindent Being mindful of the existence of a solution to an equation subject to certain constraints is
 crucial to the solution-modelling process, since this tells us whether the construction of a solution is
 possible or whether our search is in vain, hinting at reformulating the problem with 
 improved constraints.
\vskip 0.3cm
\noindent The uniqueness criteria is important for many reasons. If too many solutions exist for an equation with restrictions, it may be indication that the constraints are to weak and should be made stronger. It may also be that the solution is unique only within a certain class of operators, for example, a problem may have several solutions, only one of which is bounded.
\vskip 0.3cm
{\color{red} Below still to be reviewed
\vskip 0.3cm
\noindent Consider the generic problem $Au=f$ where $A:X \to Y$ with $u \in X, f \in Y$.  To say that a solutions depends continuously on the data and parameters, may be characterized by requiring that there exists $c > 0$ such that $\| u \|_X \leq c \| f \|_Y$ for all $f \in Y$, i.e. small variations in $X$ lead to small variations in $Y$. The widely-used and wildly popular energy-method in mathematical physics is an efficient method to test for this criteria.
\vskip 0.3cm
In practice, the (abstract) Cauchy (initial value) problem is of interest:

$$
  \left.
    \begin{array}{l}
      u'(t)=Au(t) \qquad \textup{for \space} t \geq 0 \\[0.5cm]
      u(0)=h
    \end{array}
  \right\} \quad \cdots  \quad \left(IVP\right)\\[0.5cm]
$$



\noindent It is an elementary exercise to verify that the product form $u(t)=T(t)h$ offers a solution to {\sl IVP} and suggests that $T(t)=e^{At}$ since by direct computation we have that $T'(t)=Ae^{At}=AT(t)$ and $T'(0)=A$.

\bigskip If for each initial value $h \in X$, a unique solution $u(\cdot,h)$ exists, then $u(t,h)=T(t)h$ defines an operator semigroup $\{ T(t) \}_{t \geq0}$, the solution to {\sl IVP} where $A$ is a linear operator on Banach space $X$, called the (infinitesimal) generator while $T(\cdot):\mathbb{R}^+ \to B(X)$ operating on $u$ according to the rule $T(t):u(s) \to u(t+s)$.

\bigskip The solution $u(t+s)$ at instance $t+s$ can be computed as $u(t+s)=T(t+s)h$ or {\sl alternatively} we can solve for $u(s)=T(s)h$ then set this as initial data and then compute the solution at instance $t$ giving $u(t+s)=T(t)(T(s)h)$. From the uniqueness criteria, our two approaches must be identical so we get equality, giving us the {\sl semigroup (composition) property} for operators:




\bigskip



\qquad \qquad \qquad \qquad \qquad \qquad $T(t+s)=T(t)(T(s))$

\bigskip

\qquad \qquad \qquad \qquad \qquad \qquad \qquad by induction

\bigskip

\qquad \qquad \qquad \qquad \quad $\displaystyle{T(nt)=T \big (\sum_{j=1}^n t \big )= \prod_{j=1}^n T(t)=T(t)^n}$

\bigskip

\qquad \qquad \qquad \qquad \qquad $T(0)$ is the identity operator $I$




\bigskip The semigroup (composition) property serves as affirmation whether a given structure indeed constitutes a semigroup. $T(t)$ has inverse $T(t)^{-1}=T(-t)$ for all $t \geq 0$ since $T(t)T(-t)=T(t-t)=T(0)=I=T(0)=T(-t+t)=T(-t)T(t)$, i.e. addition in the domain corresponds to multiplication in the semigroup of operators.




\begin{proposition}
\medskip \textup{Suppose $X$ is a Banach space and that all parameters of $A$, $u$ and $T$ are as described for {\sl IVP} If the generator $A$ is bounded, then $T(t)=e^{At}$ is uniformly continuous.}
\end{proposition}


\proof
Suppose $A$ is a bounded operator on $X$. Then $\| A \| < \infty$ thus $\sum_{n=0}^{\infty} \frac{(tA)^n}{n!}$ converges for each $t \geq 0$ to the bounded linear operator $T(t)$ and the semigroup (composition) property holds, since $\big (  \sum_{m=0}^{\infty} \frac{t^m}{m!} \big ) \big (  \sum_{n=0}^{\infty} \frac{s^n}{n!} \big )=\sum_{k=0}^{\infty} \frac{(t+s)^k}{k!}$ and clearly $T(0)=1$. To see that $T(t)$ is uniformly continuous, notice that $\| T(t)-I \|=\| \sum_{n=1}^{\infty} \frac{t^n \|A \|^n}{n!} \| \leq e^{t \| A \|} -1$ and $e^{t \| A \|} -1 \xlongrightarrow[t]{0} 0$. \qed



\begin{remark}
\textup{The celebrated Hille-Yoshida theorem deals with the case of constructing a semigroup $T(\cdot)$ when $A$ is an unbounded operator on Banach space $X$.}
\end{remark}



\begin{example}

\textup{For the Riemann-Liouville integral operator in example 1.5.1.10 $$(V^{\alpha}f)(x)= \frac{1}{\Gamma(\alpha)} \int_{0}^{x} (x-t)^{\alpha-1} f(t) dt$$ where $\alpha$ is a complex number with positive real part.}


\medskip

\noindent \textup{It can easily be verified that fundamental relations hold:}

\medskip \noindent \textup{$$\frac{d}{dx}V^{\alpha+1}f(x)=V^{\alpha}f(x) \qquad V^{\alpha}(V^{\beta}f)=V^{\alpha+\beta}f$$ the latter of which is the semigroup (composition) property.}

\end{example}
}

\begin{lemma}
\normalfont
\noindent Let $A, B$ be Banach algebras and let $T:A \to B$ be a bounded homomorphism. 
Then $1+T^{-1}(0)$ is a subsemigroup of the ring $A$ considered as a semigroup.
\end{lemma}
\proof \space
Let $H = 1 + T^{-1}(\{0\})$. We show that $HH = H$. 
\vskip 0.3cm
\noindent To see that $HH \subseteq H$, let $a_1, a_2 \in H$. Then there exist 
$b_1, b_2 \in T^{-1}(\{0\})$ such that 
\begin{center}
$a_1 = 1 + b_1$ \quad \quad and \quad \quad $a_2 = 1 + b_2$
\end{center}
Then
\begin{center}
$a_1 \cdot a_2 = (1+b_1)(1+b_2) = 1 + b_1 + b_2 + b_1b_2$
\end{center}
and since $T(b_1 + b_2 + b_1b_2) = T b_1 + Tb_2 + Tb_1 Tb_2 = 0$, 
we have $b_1 + b_2 + b_1b_2 \in T^{-1}(\{0\})$.
\vskip 0.3cm
\noindent Hence we have that $a_1 a_2 \in H$, so that $HH \subseteq H$. 
\vskip 0.3cm
\noindent To see that $H \subseteq HH$, suppose that $a \in H$. Then
\begin{center}
$a = a \cdot 1 = a \cdot (1 + 0) \in HH$.
\end{center}
\noindent Hence $H \subseteq HH$. The result follows.
\qed
\begin{theorem}[\cite{CH3}, p. 275] 
\normalfont
\noindent If $T: A \to B$ is a homomorphism, then  we have the containment 
\begin{center}
$(1+T^{-1}(\{0\})) \cap A^{\cap} \subseteq (1+T^{-1}(\{0\}))^{\cap}.$
\end{center}
\end{theorem}

\proof \space Let $a \in (1+T^{-1}(\{0\})) \cap A^{\cap}$. Then $a = 1 + b$ for some 
$b \in T^{-1}(\{0\})$ and there exists $a' \in A$ such that $a = aa'a$. Then 
\begin{center}
$Ta = T(1+b) = T(1) + Tb = T(1) = 1.$
\end{center}
and
\begin{center}
$Ta' = TaTa'Ta = T(aa'a) = Ta = 1.$
\end{center}
Hence $a' = a' + 0 \in 1 + T^{-1}(\{0\})$ and since $a = aa'a$, we have that $a \in (1+T^{-1}(\{0\}))^{\cap}.$
\qed
\noindent We state the following result without proof.

\begin{theorem}[\cite{CH3}, p. 275] 
\normalfont
\noindent Let $A, B$ be Banach algebras and let $T:A \to B$ be a homomorphism. Let $J=T^{-1}(\{0\})$.
We have:
\begin{align*}
1+J \subseteq \CL_A(A^{-1}) 
&\iff 1+J \subseteq \CL_A(1+J)^{-1} \\
&\implies 1+J \subseteq \CL_{1+J}(1+J)^{-1}
\end{align*}
\end{theorem}

{\color{red} Below still to be reviewed

\begin{remark}[\cite{CH3}, p. 275] \hfill

\smallskip \noindent \textup{\textbf{(Unverified converse of weakly Riesz "stable rank one" property)}}

\smallskip \noindent \textup{"We have not settled whether we can reverse the second implication of {\sl (8.7)}" i.e.} $$1+J \subseteq \CL_{1+J}(1+J)^{-1} \xLongrightarrow[\textup{\mbox{?`}}]{?} 1+J \subseteq \CL_A(1+J)^{-1}$$

\smallskip \noindent \textup{"which is, of course a triviality for the norm closure" i.e.} $$ 1+J \subseteq \cl_{{\| \cdot \|}_{1+J}} (1+J)^{-1} \implies1+J \subseteq \cl_{{\| \cdot \|}_A} (1+J)^{-1}$$

\end{remark}
}


\begin{definition}[\cite{JMW}, p. 603]
\normalfont
\noindent If $R$ is any ring, then its {\sl (Bass) stable rank}, denoted by $\bsr(R)$ is 
by definition the least integer $n$ such that whenever $r_1, \dots, r_{n+1} \in R\ \textup{and}\ \{ r_j \}$ generate $R$ as a left 
(right, respectively) ideal, there are $b_1, \dots, b_n \in R$ such that $r_1+b_1r_{n+1} \dots  r_n+b_nr_{n+1}$ ($r_1+r_{n+1}b_1 \dots  r_n+r_{n+1}b_n$, respectively)  
generate $R$ as a left (right, respectively) ideal.

\end{definition}


\section{Arens and Royden properties \newline and connected components}

\begin{definition}[\cite{H1}, p. 281] 
\normalfont
Let $X$ be a topological space, $K \subseteq X$. By a {\sl disconnection} of $K$ in $X$ we mean a pair $(G_1, G_2)$ of open subsets of $X$ with the properties that:
\begin{center}
$K \subseteq G_1 \cup G_2$ \quad and \quad $K \cap G_1 \cap G_2 = \emptyset$ \quad 
and \quad $K \cap G_1 \not = \emptyset \not = K \cap G_2$.
\end{center}
\noindent A subset $K$ of $X$ is called {\sl connected} if there are no disconnections of $K$ 
in $X$. 
\vskip 0.3cm
\noindent If $t \in K \subseteq X$ then we also define
\begin{center}
$\Comp_K(t)= \bigcup \{ H \subseteq K: t \in H\ \textup{connected in}\ X \}.$ 
\end{center}
\noindent Then $\Comp_K(t) \subseteq K$. Also $\Comp_K(t)$ is connected in $X$ and is called the {\sl connected component} of $t$ in $K$.
\end{definition}
\noindent Of interest to us is a particular connected component, the connected component of 1 in 
$A^{-1}$.

\begin{definition}[\cite{H1}, p. 290] 
\normalfont
\noindent Let $A$ be a normed algebra. Then 
\begin{center}
$A_0^{-1} = \Comp_{A^{-1}}(1) \subseteq A^{-1}.$
\end{center}
%\medskip \textup{If $A^{-1}=A_o^{-1}$, then we say that $A^{-1}$ is connected.}
\end{definition}

\begin{theorem}[\cite{H1}, p. 290] 
\normalfont
Let $A$ be a normed algebra. Then
\begin{enumerate}[label=(\alph*)]
\item $A_0^{-1}$ is a subgroup of $A^{-1}$.
\item $A_0^{-1} = A^{-1} \cap \cl_{\|\cdot\|}(A_0^{-1})$.
\item If $A^{-1}$ is an open subset of $A$, then $A_0^{-1}$ and $A^{-1} \setminus A_0^{-1}$ 
are open in $A$.
\end{enumerate}
\end{theorem}

\begin{definition}[\cite{X}, p. 96] 
\normalfont
Let $A$ and $B$ be unital Banach algebras and $T:A \to B$ a unital (i.e. $T1 = 1)$ 
continuous homomorphism. We say that $T$ has {\sl property (F)} if for every $b \in T(A)$ 
with $\| 1-b \| \leq 1$, then $b^{-1} \in T(A)$.
\end{definition}


\begin{definition}[\cite{H3}, p. 33] 
\normalfont
A bounded linear homomorphism $T:A \to B$ is said to have the {\sl Arens property} 
provided $A^{-1} \cap T^{-1}(\Exp(B)) \subseteq \Exp(A)$.
\end{definition}


\begin{definition}[\cite{H3}, p. 33] 
\normalfont
A bounded linear homomorphism $T:A \to B$ is said to have the {\sl Royden property} 
provided $B^{-1} \subseteq T(A^{-1}) \cdot \Exp(B)$.
\end{definition}

%{\color{red} Below still to be reviewed

\begin{remark}\textup{It is a standard result that for a Banach algebra $A$, the invertible group $A^{-1}$ is open, hence $A_o^{-1}$ and $A^{-1} \setminus A_o^{-1}$ are also open.}

\end{remark}

\begin{theorem}[\cite{H1}, p. 292] \textup{For a Banach algebra $A$, we have that:}

\medskip {\sl (1)} \space $A_o^{-1} = \Exp(A)$.

\medskip {\sl (2)} \space $ab=ba \implies e^{(a+b)}=e^a \cdot e^b$.

\medskip {\sl (3)} \space $\| x \| < 1 \implies 1-x = e^a$ \textup{with} $\displaystyle{a= - \sum_{n=1}^{\infty} \frac{x^n}{n}}$.

\end{theorem}


\begin{theorem}[\cite{H1}, p. 292] \textup{If $T: A \to B$ is a bounded homomorphism between normed algebras, then $T(A_o^{-1}) \subseteq B_o^{-1}$. If $A$ and $B$ are complete and $T$ is onto, then $T(A_o^{-1}) = B_o^{-1}$ and $T^{-1} (B_o^{-1}) \subseteq A^{-1} + T^{-1}(0)$. If, in addition, $A^{-1}$ is connected, tehn equality holds  $T^{-1} (B_o^{-1}) = A^{-1} + T^{-1}(0)$}.

\end{theorem}


\begin{theorem}[\cite{CH3}, p. 274] \textup{When $T: A \to B$ is a bounded homomorphism of Banach algebras then the condition {\sl (8.1)} implies the corresponding condition with norm closure, which in turn follows from {\sl property (F)} of Xue, which is equivalent to the analogue of {\sl (7.5)} in which invertible groups are replaced by the generalized exponentials, the connected components of their identities:" $\Exp(B) \cap T(A) \subseteq T(\Exp(A))$}
\end{theorem}



\begin{remark} \textup{This theorem is most conveniently broken into two part: }

\medskip \noindent \textup{If $T:A \to B$ is a bounded Homomorphism of Banach algebras, then:}

\medskip \noindent {\sl (1)} \space $\CL(A^{-1}) \cap T^{-1}(B^{-1}) \subseteq A^{-1} + T^{-1}(0)$ \space $\implies$ \space $\cl_{\| \cdot \|}(A^{-1}) \cap T^{-1}(B^{-1}) \subseteq A^{-1} + T^{-1}(0)$

\medskip \noindent {\sl (2)} \space $\cl_{\| \cdot \|}(A^{-1}) \cap T^{-1}(B^{-1}) \subseteq A^{-1} + T^{-1}(0) \iff \Exp(B) \cap T(A) \subseteq T(\Exp(A))$

\medskip \textup{Part {\sl (1)} can be phrased in words by saying that if nearly invertible $T$-Fredholm elements are $T$-Weyl with respect to the spectral topology, then the same inclusion holds with respect to the norm topology.}

\medskip \textup{Part {\sl (2)} can be phrased in words by saying that $T$-Fredholm elements are $T$-Weyl with respect to the norm topology if and only if $T$ has the Arens property.}

\end{remark}


\proof \space \textup{To prove part {\sl (1)}, suppose that $\CL(A^{-1}) \cap T^{-1}(B) \subseteq A^{-1} + T^{-1}(0)$. From proposition 3.2.0.1 with $K=A^{-1}$, we have that $\cl_{\| \cdot \|}(A^{-1}) \subseteq \CL(A^{-1})$. Therefore $\cl_{\| \cdot \|}(A^{-1}) \cap T^{-1}(B) \subseteq \CL(A^{-1}) \cap T^{-1}(B)$. Hence $\cl_{\| \cdot \|}(A^{-1}) \cap T^{-1}(B \subseteq A^{-1}+T^{-1}(0)$ as desired. Part {\sl (2)} follows from property (F)}. \qed

\bigskip

\noindent Observation:
\medskip

\noindent Combining {\sl (2)} from theorem 7.4.0.6 and theorem 6.2.0.9 we get the chain:


\bigskip




\noindent $A_o^{-1} = A^{-1} \cap \cl(A_o^{-1})$

\medskip $\subseteq  A^{-1} \cap \cl(A^{-1})$

\medskip $\subseteq (A^{-1} + T^{-1}(0)) \cap \cl(A^{-1})$

\medskip $\subseteq T^{-1}(B^{-1}) \cap \cl(A^{-1})$

\end{comment}
